%*******************************************************
% Abstract
%*******************************************************
\begingroup
\let\clearpage\relax
\let\cleardoublepage\relax
\let\cleardoublepage\relax

\pdfbookmark[1]{Abstract}{Abstract}
\chapter*{Abstract}

In this work, we define the basics of singular homology theory, starting with the necessary geometric constructions
and the definition of the singular homology groups. Following the definitions and a proof of homotopy invariance,
we prove the baricentric subdivision theorem, the central result of this work, that leads us inmediatly to the
Mayer-Vietoris sequence. This and other results are applied to obtain the homology groups of the spheres and a number
of classical theorems, including the Brower fixed-point theorem and the Jordan-Brower separation theorem.

On the second part, we introduce the maximum clique problem, a well studied combinatorial optimization problem, known
to be NP-hard. Due to this, the use of metaheuristics is necessary to solve it in reasonable time. Several heuristics
are considered, explained in detail and applied to DIMACS instances. We discuss their behaviour and compare them,
showing which of the proposed algorithms can yield satisfactory results.


\paragraph{keywords} Homology theory, Homotopy invariance, Baricentric subdivision theorem, Mayer-Vietoris sequence,
Maximum clique, Combinatorial optimization problem, Metaheuristics, NP-hard, DIMACS.

\vfill

\endgroup

\vfill
