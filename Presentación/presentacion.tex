%%%
% Plantilla de Presentación
% Modificación de una plantilla de Latex de LaTeXTemplates para adaptarla
% al castellano y a las necesidades de escribir informática y matemáticas.
%
% Tomada de https://github.com/M42/plantillas/blob/master/presentacion/presentacion.tex
% Créditos a MRoman42 por la adaptación.
% Más adaptaciones a mi gusto.
%
% License:
% CC BY-NC-SA 3.0 (http://creativecommons.org/licenses/by-nc-sa/3.0/)
%%%

%%%%%%%%%%%%%%%%%%%%%%%%%%%%%%%%%%%%%%%%%
% Beamer Presentation
% LaTeX Template
% Version 1.0 (10/11/12)
%
% This template has been downloaded from:
% http://www.LaTeXTemplates.com
%%%%%%%%%%%%%%%%%%%%%%%%%%%%%%%%%%%%%%%%%

%----------------------------------------------------------------------------------------
%	PAQUETES Y CONFIGURACIÓN DEL DOCUMENTO
%----------------------------------------------------------------------------------------

\documentclass{beamer}

%% Configuración de la presentación
\mode<presentation> {
  %%% Selección de estilo
  % The Beamer class comes with a number of default slide themes
  % which change the colors and layouts of slides. Below this is a list
  % of all the themes, uncomment each in turn to see what they look like.

  %\usetheme{default}
  %\usetheme{AnnArbor}
  %\usetheme{Antibes}
  %\usetheme{Bergen}
  %\usetheme{Berkeley}
  %\usetheme{Berlin}
  %\usetheme{Boadilla}
  %\usetheme{CambridgeUS}
  %\usetheme{Copenhagen}
  %\usetheme{Darmstadt}
  %\usetheme{Dresden}
  %\usetheme{Frankfurt}
  %\usetheme{Goettingen}
  %\usetheme{Hannover}
  %\usetheme{Ilmenau}
  %\usetheme{JuanLesPins}
  %\usetheme{Luebeck}
  %\usetheme{Madrid}
  %\usetheme{Malmoe}
  %\usetheme{Marburg}
  %\usetheme{Montpellier}
  %\usetheme{PaloAlto}
  %\usetheme{Pittsburgh}
  %\usetheme{Rochester}
  %\usetheme{Singapore}
  \usetheme{Szeged}
  %\usetheme{Warsaw}

  %% Selección de color
  % As well as themes, the Beamer class has a number of color themes
  % for any slide theme. Uncomment each of these in turn to see how it
  % changes the colors of your current slide theme.

  %\usecolortheme{albatross}
  %\usecolortheme{beaver}
  %\usecolortheme{beetle}
  %\usecolortheme{crane}
  %\usecolortheme{dolphin}
  %\usecolortheme{dove}
  %\usecolortheme{fly}
  %\usecolortheme{lily}
  %\usecolortheme{orchid}
  %\usecolortheme{rose}
  %\usecolortheme{seagull}
  %\usecolortheme{seahorse}
  \usecolortheme{whale}
  %\usecolortheme{wolverine}

  %% Configuración del pie de línea
  %\setbeamertemplate{footline} % To remove the footer line in all slides uncomment this line
  %\setbeamertemplate{footline}[page number] % To replace the footer line in all slides with a simple slide count uncomment this line
  %\setbeamertemplate{navigation symbols}{} % To remove the navigation symbols from the bottom of all slides uncomment this line
}

%% Fuentes de tamaño arbitrario
\usepackage{lmodern}

%% Gráficos
\usepackage{graphicx} % Allows including images
\usepackage{booktabs} % Allows the use of \toprule, \midrule and \bottomrule in tables

%%% Castellano.
% noquoting: Permite uso de comillas no españolas.
% lcroman: Permite la enumeración con numerales romanos en minúscula.
% fontenc: Usa la fuente completa para que pueda copiarse correctamente del pdf.
\usepackage[english,spanish,es-noquoting,es-lcroman]{babel}
\usepackage[utf8]{inputenc}
\usepackage[T1]{fontenc}
\selectlanguage{spanish}

%-------------------------------------------
% Paquetes añadidos por JC.
%-------------------------------------------
% Tikz
\usepackage{tikz}
\usepackage{tikz-3dplot}

\usepackage{verbatim}
\usetikzlibrary{arrows,shapes}

% Sections
\AtBeginSection{\frame{\sectionpage}}
\newtranslation[to=spanish]{Section}{Sección}

\defbeamertemplate{section page}{mine}[1][]{%
  \begin{centering}
    {\usebeamerfont{section name}\usebeamercolor[fg]{section name}#1}
    \vskip1em\par
    \begin{beamercolorbox}[sep=12pt,center]{part title}
      \usebeamerfont{section title}\insertsection\par
    \end{beamercolorbox}
  \end{centering}
}

%----------------------------------------------------------------------------------------
%	TÍTULO
%----------------------------------------------------------------------------------------

\title[Homología Singular. Problema del clíque máximo.]{Introducción a la homología singular. Análisis y resolución del problema del clique máximo.} % The short title appears at the bottom of every slide, the full title is only on the title page

\author{José Carlos Entrena Jiménez} % Your name
\institute[UGR] % Your institution as it will appear on the bottom of every slide, may be shorthand to save space
{
  Universidad de Granada \\ % Your institution for the title page
  \medskip
  \textit{jentrena@correo.ugr.es} % Your email address
}
\date{\today} % Date, can be changed to a custom date



\begin{document}
% Spanish
\selectlanguage{spanish}
% Para tikz
\pgfdeclarelayer{background}
\pgfsetlayers{background,main}
% Secciones
\setbeamertemplate{section page}[mine]

%% Diapositiva de título.
\frame{\titlepage}

%% Diapositiva de contenidos.
% Throughout your presentation, if you choose to use \section{} and \subsection{} commands,
% these will automatically be printed on this slide as an overview of your presentation
\begin{frame}
  \frametitle{Contenidos} % Table of contents slide, comment this block out to remove it
  \tableofcontents
\end{frame}


%----------------------------------------------------------------------------------------
%	PRESENTACIÓN
%----------------------------------------------------------------------------------------

%------------------------------------------------
\section{Homología Singular} % Sections can be created in order to organize your presentation into discrete blocks, all sections and subsections are automatically printed in the table of contents as an overview of the talk
%------------------------------------------------

\subsection{Introducción} % A subsection can be created just before a set of slides with a common theme to further break down your presentation into chunks

\begin{frame}
\frametitle{Ideas}

Símplices y símplices singulares. Construcción de los grupos y definición de borde. Complejo de cadenas.
Ciclos y bordes. Homología.

\end{frame}


\begin{frame}
  \frametitle{Motivación}
  \begin{itemize}
    \item Construir la homología singular.
    \item Obtener herramientas para calcularla.
    \item Obtener resultados a partir de estas herramientas.
  \end{itemize}

\end{frame}



%------------------------------------------------

\begin{frame}
  \frametitle{Objetivos}
  Los objetivos que se plantearon antes del inicio del trabajo fueron los siguientes:
  \begin{itemize}
    \item Construcción de la homología singular y estudio de sus propiedades.
    \item Cálculo de la homología singular de las esferas y obtención de aplicaciones.
  \end{itemize}
  Ambos objetivos se han cumplido de forma satisfactoria.
\end{frame}


%------------------------------------------------
\subsection{Definiciones}

\begin{frame}
  \frametitle{Definiciones}
  \begin{columns}[c]

    \column{.45\textwidth}
    Un $p$-símplice es la envolvente convexa de $p-1$ puntos afínmente independientes.
    Lo llamaremos estándar si sus puntos son de la forma $(0, \dots, 1, \dots, 0)$.

    \column{.5\textwidth}
    \begin{figure}
      \tdplotsetmaincoords{70}{130}
      \begin{tikzpicture}[tdplot_main_coords]
        \def\laxis{3}
        \def\ltriangle{1}
        \def\ltick{.2}
        %%% axes
        \draw [->] (0,0,0) -- (\laxis,0,0) node [below] {$x$};
        \draw [->] (0,0,0) -- (0,\laxis,0) node [right] {$y$};
        \draw [->] (0,0,0) -- (0,0,\laxis) node [left] {$z$};
        %%% axes ticks
        \pgfmathtruncatemacro{\nticks}{floor(\laxis)-1}
        \begin{scope}[
          help lines,
          every node/.style={inner sep=1pt,text=black}
          ]
          \foreach \coord in {1,...,\nticks} {
            \draw (\coord,\ltick,0) -- ++(0,-\ltick,0) -- ++(0,0,\ltick)
            node [pos=1,left] {\coord};
            \draw (\ltick,\coord,0) -- ++(-\ltick,0,0) -- ++(0,0,\ltick)
            node [pos=1,right] {\coord};
            \draw (\ltick,0,\coord) -- ++(-\ltick,0,0) -- ++(0,\ltick,0)
            node [at start,above right] {\coord};
          }
        \end{scope}
        %%% figure
        \filldraw [opacity=.33,red] (\ltriangle,0,0) -- (0,\ltriangle,0)
        -- (0,0,\ltriangle) -- cycle;
      \end{tikzpicture}
      \caption{$2$-símplice estándar}
    \end{figure}

  \end{columns}

\end{frame}

%------------------------

\begin{frame}
  \frametitle{Definiciones}

  Un p-símplice singular será una aplicación continua de un p-símplice a un espacio topológico.
  Tomando un $\mathbb{Z}$-módulo sobre el conjunto de p-símplices singulares creamos el grupo
  de p-cadenas singulares, $S_p(\mathbb{X})$.
  \\~\\
  Sobre el grupo de p-cadenas singulares podemos definir un homomorfismo de grupos borde $\partial$,
  de forma que el borde de una p-cadena sea una (p-1)-cadena sin borde. Por tanto, el borde verifica
  que $\partial^2$ es la aplicación cero.

\end{frame}

%---------------------------

\begin{frame}
  \frametitle{Definiciones}
  Consideramos los ciclos y los bordes:
  \begin{itemize}
    \item $Z_p(\mathbb{X}) = \{c \in S_p(\mathbb{X}) \mid \partial(c) = 0\}$ p-ciclos.
    \item $B_p(\mathbb{X}) = \{c \in S_p(\mathbb{X}) \mid \exists d \colon \partial(d) = c\}$ p-bordes.
  \end{itemize}

  Como $B_p \subseteq Z_p$, podemos tomar el cociente, el p-ésimo grupo de homología: \\~\\
  \centerline{$H_p(\mathbb{X}) = \frac{Z_p(\mathbb{X})}{B_p(\mathbb{X})}$}
\end{frame}

%------------------------------------------------

\begin{frame}
\frametitle{Multiple Columns}
\begin{columns}[c] % The "c" option specifies centered vertical alignment while the "t" option is used for top vertical alignment

\column{.45\textwidth} % Left column and width
\textbf{Heading}
\begin{enumerate}
\item Statement
\item Explanation
\item Example
\end{enumerate}

\column{.5\textwidth} % Right column and width
Lorem ipsum dolor sit amet, consectetur adipiscing elit. Integer lectus nisl, ultricies in feugiat rutrum, porttitor sit amet augue. Aliquam ut tortor mauris. Sed volutpat ante purus, quis accumsan dolor.

\end{columns}
\end{frame}



\begin{frame}
\frametitle{Theorem}
\begin{theorem}[Mass--energy equivalence]
$E = mc^2$
\end{theorem}
\end{frame}

%------------------------------------------------

\begin{frame}[fragile] % Need to use the fragile option when verbatim is used in the slide
\frametitle{Verbatim}
\begin{example}[Theorem Slide Code]
\begin{verbatim}
\begin{frame}
\frametitle{Theorem}
\begin{theorem}[Mass--energy equivalence]
$E = mc^2$
\end{theorem}
\end{frame}\end{verbatim}
\end{example}
\end{frame}

%------------------------------------------------

\begin{frame}
\frametitle{Figure}
Uncomment the code on this slide to include your own image from the same directory as the template .TeX file.
%\begin{figure}
%\includegraphics[width=0.8\linewidth]{test}
%\end{figure}
\end{frame}

%------------------------------------------------

\begin{frame}[fragile] % Need to use the fragile option when verbatim is used in the slide
\frametitle{Citation}
An example of the \verb|\cite| command to cite within the presentation:\\~

This statement requires citation \cite{p1}.
\end{frame}

%------------------------------------------------
\section{Problema del clique máximo}
%------------------------------------------------

\subsection{Introducción}

\begin{frame}
\frametitle{Descripción del problema}
Llamamos clique a un grafo cuyos vértices están todos conectados entre sí. \\~\\

Dado un grafo $G = (V, E)$, el problema del clique máximo consiste en buscar el clique de mayor
tamaño dentro de $G$. \\~\\

\end{frame}

\begin{frame}
  \frametitle{Ejemplo}
  \tikzstyle{vertex}=[circle,fill=black!25,minimum size=20pt,inner sep=0pt]
  \tikzstyle{selected vertex} = [vertex, fill=red!24]
  \tikzstyle{edge} = [draw,thick,-]
  \tikzstyle{weight} = [font=\small]
  \tikzstyle{selected edge} = [draw,line width=5pt,-,red!50]
  \tikzstyle{ignored edge} = [draw,line width=5pt,-,black!20]

  \begin{figure}
  \begin{tikzpicture}[scale=1.25, auto,swap]
      % Draw a 7,11 network
      % First we draw the vertices
      \foreach \pos/\name in {{(0,2)/a}, {(2,1)/b}, {(4,1)/c},
                              {(0,0)/d}, {(3,0)/e}, {(0,-2)/f},
                              {(2,-1)/g}, {(4,-1)/h}}
          \node[vertex] (\name) at \pos {$\name$};
      % Connect vertices with edges and draw weights
      \foreach \source/ \dest /\weight in {b/a/, b/g/, c/b/, d/a/,
                                           d/b/, e/b/, e/c/, e/d/,
                                           f/d/, g/d/, g/e/, h/e/,
                                           h/g/}
          \path[edge] (\source) -- (\dest);
      % Start animating the vertex and edge selection.
      \foreach \vertex / \fr in {b/2,d/2,e/2,g/2}
          \path<\fr-> node[selected vertex] at (\vertex) {$\vertex$};
      % For convenience we use a background layer to highlight edges
      % This way we don't have to worry about the highlighting covering
      % weight labels.
      \begin{pgfonlayer}{background}
          \pause
          \foreach \source / \dest / \fr in {b/d/1,b/e/1,b/g/1,d/e/1,d/g/1,e/g/1}
              \path<\fr->[selected edge] (\source.center) -- (\dest.center);
      \end{pgfonlayer}
  \end{tikzpicture}
  \end{figure}

\end{frame}

\begin{frame}

  Es un problema NP-difícil, y salvo que NP = ZPP no existen algoritmos capaces de aproximar el
  problema con factor $1-\epsilon$ para ningún $\epsilon$ entre $0$ y $1$. \\~\\

  Será necesario el uso de algoritmos heurísticos para resolverlo en un tiempo razonable.
\end{frame}

%------------------------------------

\begin{frame}
  \frametitle{Objetivos}
  Los objetivos planteados antes del inicio del trabajo fueron los siguientes:
  \begin{itemize}
    \item Recopilación de bibliografía sobre el trabajo y las heurísticas aplicables.
    \item Estudio y selección de las técnicas a utilizar.
    \item Diseño e implementación de los algoritmos.
    \item Experimentación y estudio comparativo.
  \end{itemize}
  El primer objetivo se ha cumplido parcialmente. El resto de objetivos se han cumplido de forma satisfactoria.

\end{frame}

%---------------------------------------

\subsection{Algoritmos}

\begin{frame}
  \frametitle{Algoritmos}
  Implementados 14 algoritmos divididos en 7 categorías distintas: \textit{greedy}, búsqueda local,
  enfriamiento simulado, ILS, GRASP, algoritmos de colonia de hormigas y algoritmos genéticos.
  Cada categoría incluye dos algoritmos.

\end{frame}

\begin{frame}
  \frametitle{Algoritmos \textit{greedy}}
  Los algoritmos \textit{greedy} resuelven problemas tomando las decisiones óptimas en cada instante. \\~\\

  Dos algoritmos implementados:
  \begin{itemize}
    \item \textit{Greedy} básico: primer acercamiento al problema con un algoritmo sencillo.
    \item \textit{Greedy} adaptativo: algoritmo relajado que permite deshacer cambios. \cite{grosso:2004}
  \end{itemize}

\end{frame}
%-----------------------------------------

\begin{frame}
  \frametitle{Búsqueda local}
   La búsqueda local usa el entorno de una solución para buscar soluciones que mejoren a la actual.
   Nos valdremos de los operadores \textit{add, swap} y \textit{drop} para crear el entorno. \\~\\

  Dos algoritmos implementados:
  \begin{itemize}
    \item 1LS: Utiliza los tres operadores ordenados pror prioridad, favoreciendo add sobre swap y swap sobre drop.
    \item DLS: Utiliza add y swap, junto a una lista tabú para evitar que los nodos que salgan vuelvan a entrar.
  \end{itemize}

\end{frame}

%-----------------------------------------

\begin{frame}
  \frametitle{Enfriamiento simulado}
   El enfriamiento simulado es un algoritmo de búsqueda en entornos que contiene un criterio probabilístico
   de aceptación de soluciones basado en la termodinámica, el criterio de Metrópolis. Esto permite aceptar soluciones
   peores para salir de óptimos locales. \\~\\

  Dos algoritmos implementados:
  \begin{itemize}
    \item ES básico: Primer acercamiento, usando \textit{add, swap} y \textit{drop} y tomando un movimiento aleatorio.
    \item ES adaptado: Trabaja con grafos cualesquiera y los reduce hasta obtener un clique. Considera \textit{add, swap} y \textit{drop},
    con orden aleatorio, y una nueva función objetivo.
  \end{itemize}

\end{frame}

%-----------------------------------------

\begin{frame}
  \frametitle{ILS}
   La búsqueda local iterada es un algoritmo de búsqueda local multiarranque, que utiliza la perturbación de soluciones para
   reinizializar la búsqueda y obtener diversidad. \\~\\

   El método de reinicialización es añadir un nodo aleatorio a una solución final, eliminando todos los nodos que no estén
   conectados a este. Como búsqueda local se ha utilizado 1LS y DLS, dando lugar a dos algoritmos.

\end{frame}

%-----------------------------------------

\begin{frame}
  \frametitle{GRASP}
  GRASP es un algoritmo multiarranque consistente en repetir una fase de construcción de la solución inicial seguida de una búsqueda local.
  La solución inicial se construye mendiante un algoritmo \textit{greedy} aleatorizado. \\~\\

  Para la construcción de la solución inicial tomaremos el 50\% mejor de los nodos candidatos y elegiremos aleatoriamente.
  De nuevo se utiliza 1LS y DLS para la fase de búsqueda local, lo que nos da dos algoritmos.

\end{frame}


%-----------------------------------------

\begin{frame}
  \frametitle{Algoritmos de colonia de hormigas}
  Los ACO son una técnica basada en adaptación social que emulan el comportamiento de las hormigas. Utilizan un recurso basado en
  la feromona para encontrar caminos en grafos.  \\~\\

  Dos algoritmos implementados:
  \begin{itemize}
    \item ACO básico: Construye un clique añadiendo nodos, seleccionándolos con una probabilidad dependiente del nivel de feromona.
    \item ACO con enfriamiento simulado: Añade a la probabilidad información del problema utilizando enfriamiento simulado, para
          que su importancia descienda con el tiempo.
 \end{itemize}

\end{frame}


%-----------------------------------------

\begin{frame}
  \frametitle{Algoritmos genéticos}
  Los algoritmos genéticos son una metaheurística de poblaciones basada en los porcesos evolutivos presentes en la naturaleza.
  Contarán con una población formada por soluciones, que se combinarán dos a dos para producir descendientes. \\~\\

  Se ha implementado un algoritmo elitista, en el que dos hijos compiten con sus padres por entrar a la población.

\end{frame}

%-----------------------------------------

\begin{frame}
  \frametitle{Algoritmos meméticos}
  Son el fruto de combinar los algoritmos genéticos con una búsqueda local.  \\~\\

  Partiendo del algoritmo genético anterior, se ha aplicado DLS a todos los elementos de la población después de
  los cruces, para tratar de mejorarla.

\end{frame}

\begin{frame}
  \frametitle{Implementación}

  Todos los algoritmos se han implementado en \textbf{\emph{Ruby}}

\end{frame}


%-----------------------------------------

\subsection{Resultados}

\begin{frame}
\frametitle{Resultados individuales}

  \tiny
  \begin{table}
  \begin{tabular}{l l l}
    \textbf{Algoritmo} & \textbf{Óptimos} & \textbf{Tamaño medio}  \\ \hline
    Greedy básico          & 1  & 73.2\% \\ \hline
    Greedy adaptativo      & 3  & 84.9\% \\ \hline
    1LS                    & 4  & 87.5\% \\ \hline
    DLS                    & 9  & 91.0\% \\ \hline
    SA básico              & 16  & 93.5\% \\ \hline
    SA adaptado            & 0  & 75.7\% \\ \hline
    ILS + 1LS              & 11  & 90.3\% \\ \hline
    ILS + DLS              & 15  & 93.3\%  \\ \hline
    GRASP + 1LS            & 16  & 93.5\% \\ \hline
    GRASP + DLS            & 18  & 95.3\% \\ \hline
    ACO básico             & 12  & 89.7\% \\ \hline
    ACO + SA               & 7  & 88.5\% \\ \hline
    Genético               & 13  & 90.5\% \\ \hline
    Memético               & 25  & 97.3\% \\ \hline
    \end{tabular}
  \end{table}
\end{frame}

\begin{frame}
\frametitle{Ránking}

  \tiny
  \begin{table}
  \begin{tabular}{l l l l}
    \textbf{Algoritmo} & \textbf{$1^{er}$ puesto} & \textbf{$2$º puesto} & \textbf{$3^{er}$ puesto} \\ \hline
    Greedy básico          &  0  &  2  &  0  \\ \hline
    Greedy adaptativo      &  0  &  2  &  0  \\ \hline
    1LS                    &  0  &  0  &  6  \\ \hline
    DLS                    &  11  &  0  &  2  \\ \hline
    SA básico              &  0  &  3  &  6  \\ \hline
    SA adaptado            &  0  &  0  &  0  \\ \hline
    ILS + 1LS              &  2  &  0  &  0  \\ \hline
    ILS + DLS              &  8  &  12  &  4  \\ \hline
    GRASP + 1LS            &  0  &  1  &  3  \\ \hline
    GRASP + DLS            &  4  &  8  &  8  \\ \hline
    ACO básico             &  0  &  1  &  1  \\ \hline
    ACO + SA               &  0  &  0  &  2  \\ \hline
    Genético               &  3  &  2  &  2  \\ \hline
    Memético               &  7  &  4  &  1  \\ \hline
    \end{tabular}
  \end{table}
\end{frame}

%------------------------------------------------

\begin{frame}
  \frametitle{Trabajo futuro}

  \begin{itemize}
    \item Enfatizar en un tipo de algoritmos, realizando cambios para mejorar su comportamiento.
    \item Ampliar el trabajo con nuevas técnicas.
  \end{itemize}

\end{frame}

%------------------------------------------------

\begin{frame}
\Huge{\centerline{Fin.}}
\end{frame}


%------------------------------------------------

%% Bibliografía
\begin{frame}
\frametitle{Referencias}
\footnotesize{
  \begin{thebibliography}{99} % Beamer does not support BibTeX so references must be inserted manually as below

    \bibitem[Grosso et al, 2004]{grosso:2004}
      \newblock Combining Swaps and Node Weights in an Adaptative Greedy Approach for the Maximum Clique Problem.
      \newblock \emph{Journal of Heuristics} 10, 135--152.


  \end{thebibliography}
}
\end{frame}



%----------------------------------------------------------------------------------------

\end{document}
--
