%************************************************
\chapter{Introducción}\label{ch:resultados}
%************************************************

\section{Definiciones iniciales}

Vamos a utilizar el cálculo de la homología de las esferas para obtener resultados
de carácter topológico.

\begin{proposition}
  a) Esferas de distinta dimensión no tienen el mismo tipo de homotopía. En particular, no son homeomorfas.

  b) Espacios euclídeos de distinta dimensión no son homeomorfos.
\end{proposition}

\begin{proof}
  a) Sean $S^n, S^m$ con $n \neq m$. Si $S^n \cong S^m$, entonces $H_n(S^n) \cong H_n(S^m)$, pero hemos visto que eso no sucede.

  b) Si $n \neq m$, y $h \colon \R^n \to \R^m$ es un homeomorfismo, entonces su restricción a $R^n - \{0\}$ también lo es.
  Como $R^n - \{0\} \cong S^n$, y $R^m - \{f(0)\} \cong S^m$, llegamos de nuevo a una contradicción.
\end{proof}

\begin{remark}
  $\R^n$ y $\R^m$ tienen el mismo tipo de homotopía. Si $n \leq m$, podemos ver $\R^m$ como \\
  $\R^m = \{(x, y) \mid x \in \R^n, y \in \R^{m-n}\}$, y se define
  \[ H((x, y), t) = (1-t)(x, y) + t(x, 0) \]
  homotopía entre ambos espacios, siendo $\R^n$ un retracto de $\R^m$.
\end{remark}
