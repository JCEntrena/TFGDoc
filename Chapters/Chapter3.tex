%************************************************
\chapter{Algunos resultados clásicos}\label{ch:resultados}
%************************************************

\section{Teorema del punto fijo de Brower}

Vamos a utilizar el cálculo de la homología de las esferas para obtener resultados
de carácter topológico.

\begin{proposition}
  a) Esferas de distinta dimensión no tienen el mismo tipo de homotopía. En particular, no son homeomorfas.

  b) Espacios euclídeos de distinta dimensión no son homeomorfos.
\end{proposition}

\begin{proof}
  a) Sean $S^n, S^m$ con $n \neq m$. Si $S^n \cong S^m$, entonces $H_n(S^n) \cong H_n(S^m)$, pero hemos visto que eso no sucede.

  b) Si $n \neq m$, y $h \colon \R^n \to \R^m$ es un homeomorfismo, entonces su restricción a $R^n - \{0\}$ también lo es.
  Como $R^n - \{0\} \cong S^n$, y $R^m - \{f(0)\} \cong S^m$, llegamos de nuevo a una contradicción.
\end{proof}

\begin{remark}
  $\R^n$ y $\R^m$ tienen el mismo tipo de homotopía. Si $n \leq m$, podemos ver $\R^m$ como \\
  $\R^m = \{(x, y) \mid x \in \R^n, y \in \R^{m-n}\}$, y se define
  \[ H((x, y), t) = (1-t)(x, y) + t(x, 0) \]
  homotopía entre ambos espacios, siendo $\R^n$ un retracto de $\R^m$.
\end{remark}

\begin{lemma}
  No hay ninguna retracción de $D^n$ a $S^{n-1}$.
\end{lemma}

\begin{proof}
  Sean $D^n = \{x \in \R^n \mid |x| \leq 1\}, S^{n-1} = \{x \in \R^n \mid |x| = 1\} \subseteq D^n$. Si existiera dicha retracción,
  $r \colon D^n \to S^{n-1}$, tenemos el siguiente diagrama conmutativo:
  \[ \begin{tikzcd}
    S^{n-1} \arrow[hookrightarrow]{dr}{i} \arrow{rr}{Id} & & S^{n-1} \\
    & D^n \arrow{ur}{r}
  \end{tikzcd} \]
  de donde se obtiene un diagrama, también conmutativo, en la homología:
  \[ \begin{tikzcd}
    Z \cong H_{n-1}(S^{n-1}) \arrow[hookrightarrow]{dr}{i_*} \arrow{rr}{Id_*} & & H_{n-1}(S^{n-1}) \\
    & H_{n-1}(D^n) \arrow{ur}{r_*}
  \end{tikzcd} \]
  lo cual es imposible, pues $H_{n-1}(D^n) \cong 0$, y tendríamos una factorización de la identidad que incluye a la aplicación nula,
  que en este caso sería $i_*$.
\end{proof}

\begin{corollary}
  Sea $f \colon D^n \to \R^n$ continua, tal que la restricción de $f$ a $S^{n-1}$ sea la identidad.
  Entonces existe $x \in D^n$ tal que $f(x) = 0$.
\end{corollary}

\begin{proof}
  Supongamos que no existe dicho $x$. En tal caso, es posible definir $h(x) = \frac{f(x)}{|f(x)|}$
  continua, que contradice el resultado anterior.
\end{proof}

Vamos a demostrar un teorema que nos dará como consecuencia directa un resultado clásico, el teorema
del punto fijo de Brower.

\begin{theorem}
  Sea $f \colon D^n \to \R^n$ continua. Entonces o bien existe $y \in D^n$ tal que $f(y) = 0$, o existe
  $z \in S^{n-1}$ con $f(z) = \lambda z, \quad \lambda < 0$.
\end{theorem}

\begin{proof}
  Sea $h \colon D^n \to \R^n$, definida por:
  \[ h(x) = \begin{cases} f(2x) &\text{ si } |x| \leq \frac{1}{2} \\
                          (2-2|x|)f(\frac{x}{|x|}) + (2|x| - 1)x &\text{ si } |x| \geq \frac{1}{2}  \end{cases} \]
  que es continua, ya que en $\frac{1}{2}$ ambas definiciones coinciden.

  Si $|x| = 1, h(x) = x$, y aplicando el corolario anterior, existe $y \in D^n$ tal que $h(y) = 0$. Distinguimos dos casos:

  Si $|y| \leq \frac{1}{2}, 0 = f(2y)$, y estamos en el primer caso.

  Si $|y| > \frac{1}{2}, f(\frac{x}{|x|}) = \frac{(-2|x| + 1)x}{2-2|x|}$. Si llamamos $z = \frac{x}{|x|}$,
  $f(z) = \frac{(-2|x|+1)|x|}{2-2|x|} z$, por lo que estamos en el segundo caso del teorema.
\end{proof}

\begin{corollary}
  Sea $f \colon D^n \to \R^n$ continua. Entonces existe $y \in D^n$ tal que $f(y) = y$ o existe $z \in S^{n-1}$ con $f(z) = \mu z, \mu > 1$.
\end{corollary}

\begin{proof}
  Aplicamos el teorema anterior a $-f + Id$.
\end{proof}

\begin{corollary}[Teorema del punto fijo de Brower]
  Toda aplicación continua de $D^n$ en sí mismo tiene un punto fijo.
\end{corollary}

\begin{proof}
  Basta ver que no puede darse la segunda opción del corolario anterior, pues la imagen de la función caería fuera de $D^n$.
\end{proof}

\section{Teoremas de invarianza}

Definamos el grupo de homología local, herramienta que nos servirá para demostrar varios resultados de invarianza.

\begin{definition}
  Sean $\X$ un espacio topológico, $x \in \X$. A $H_p(\X, \X - \{x\})$ lo llamaremos el \textbf{grupo de homología local} de $\X$ en $x$.
\end{definition}

\begin{proposition}
  Sea $x \in \X$, y supongamos que $\{x\}$ es un cerrado. Entonces, para todo entorno $V$ de $x$, se verifica:
  \[ H_p(\X, \X - \{x\}) \cong H_p(V, V - \{x\}) \quad \forall p \geq 0. \]
\end{proposition}

\begin{proof}
  Sea $U = \X - V \subseteq \X - \{x\}$. Si tomamos el cierre, se tiene
  \[ \overline{U} = \overline{\X - V} = \X - \interior{V} \subseteq \X - x = \X - \overline{x} = \interior{(\X - x)} .\]
  Aplicando el teorema de escisión, $i \colon (\X - U, (\X - \{x\}) - U) \to (\X, \X - \{x\})$ induce un isomorfismo en la homología.
  Como $\X - U = V, (\X - \{x\}) - U = V - \{x\}$, obtenemos el resultado.
\end{proof}

\begin{proposition}
  Sean $\X, \Y$ espacios topológicos, $x \in \X, y \in \Y$ cerrados. Si existen entornos $x \in V, y \in W$ tales que si $(V, \{x\}) \cong (W, \{y\})$,
  entonces $H_*(\X, \X - \{x\}) \cong H_*(\Y, \Y - \{y\})$.
\end{proposition}

\begin{proof}
  Sea $h \colon V \to W$ el homeomorfismo entre $V$ y $W$, con $h(x) = y$. La restricción $h \colon V - \{x\} \to W - \{y\}$ también es un homeomorfismo,
  por lo que $H_*(V, V - \{x\}) \cong H_*(W, W - \{y\})$: Basta aplicar la proposición anterior para obtener el resultado.
\end{proof}

Vamos a calcular la homología local de $\R^n$.

\begin{proposition}
  $H_p(\R^n, \R^n - \{x\}) \cong \begin{cases} 0 &\text{ si } n \neq p \\ \Z &\text{ si } n = p \end{cases}$
\end{proposition}

\begin{proof}
  Si $p = 0$, ya sabemos que $H_0(\R^n, \R^n - \{x\}) = 0$. Supondremos $p \geq 1$.

  Consideramos la sucesión de homología del par, teniendo en cuenta que $\R^n - \{x\}$ y $S^{n-1}$ tienen el mismo tipo de homotopía. \\
  Si $p = 1$:
  \[ 0 = H_1(\R^n) \xrightarrow{\pi_*} H_1(\R^n, \R^n - \{x\}) \xrightarrow{\Delta} H_0(\R^n - \{x\}) \xrightarrow{i_*} H_0(\R^n) = \Z \to 0 \]
  Si $n \geq 2, H_0(\R^n - \{x\}) \cong \Z$. \\
  Como $\Ker \Delta = \Img \pi_* = 0, \Img \Delta = \Ker i_* = 0, H_1(\R^n, \R^n - \{x\}) = 0$. \\
  Si $n = 1, H_0(\R^n - \{x\}) \cong \Z \oplus \Z$, luego $H_1(\R^n, \R^n - \{x\}) \cong \Z$.

  Si $p \geq 2$:
  \[ 0 = H_p(\R^n) \xrightarrow{\pi_*} H_p(\R^n, \R^n - \{x\}) \xrightarrow{\Delta} H_{p-1}(\R^n - \{x\}) \xrightarrow{i_*} H_{p-1}(\R^n) = 0 \]
  Si $p \neq n, H_{p-1}(\R^n - \{x\}) = 0$, luego $H_p(\R^n, \R^n - \{x\}) = 0$. \\
  Si $p = n, H_{p-1}(\R^n - \{x\}) \cong \Z$, luego  $H_p(\R^n, \R^n - \{x\}) = \Z$.
\end{proof}

\begin{theorem}[Invarianza de la dimensión]
  Sean $O_1 \subseteq \R^n, O_2 \subseteq \R^m$ abiertos homeomorfos. Entonces $n = m$.
\end{theorem}

\begin{proof}
  Sea $\phi \colon O_1 \to O_2$ el homeomorfismo entre los abiertos, y sea $x \in O_1$. Como $O_1, O_2$ son entornos de $x, \phi(x)$,
  por la proposición anterior:
  \[H_*(\R^n, \R^n - \{x\}) \cong H_*(\R^m, \R^m - \{\phi(x)\}) \]
  y en particular
  \[ \Z \cong H_n(\R^n, \R^n - \{x\}) \cong H_n(\R^m, \R^m - \{\phi(x)\}) \]
  de donde se deduce $n = m$.
\end{proof}

\begin{corollary}
  Si $M$ es un espacio topológico $T_2$, conexo, y verificando que para cada $p \in M$ existen $V_p$ entorno de $p$, un entero
  $n(p) \geq 1$ y un homeomorfismo $\phi_p \colon V_p \to O_p \subseteq \R^{n(p)}$ con $O_p$ abierto, entonces
  $n(p) = n$ constante y $M$ es una variedad topológica $n$-dimensional.
\end{corollary}

\begin{proof}
  Sea $p_0 \in M$ fijo, $A = \{p \in M \mid n(p) = n(p_0)\}$ no vacío, pues $p_0 \in A$. Es claro que si $q \in V_p$,
  $n(q) = n(p)$, pues si no fuera así, se contradice el teorema de invarianza de la dimensión. Así, si $p \in A, V_p \subseteq A$
  y por tanto $A$ es un abierto.

  Si $p \in \overline{A}$, existe $q \in A \cap V_p$ con $n(q) = n(p_0)$ por $q \in A$, y $n(q) = n(p)$ por $q \in V_p$.
  Por tanto, $n(p) = n(p_0), p \in A$, y $A$ es cerrado.

  En consecuencia, $A = M$.
\end{proof}

Pasemos ahora a un resultado que nos permitirá demostrar el teorema de invarianza del borde.

\begin{lemma}
  Sean $H^n = \{x \in \R^n \mid x_n \geq 0 \}$ semiespacio de $\R^n$, $\partial H^n = Fr H^n = \{x \in \R^n \mid x_n = 0\}$.
  Consideremos también $\interior{H^n} = \{x \in \R^n \mid x_n > 0\}$. Entonces, se verifica
  \[H_p(H^n, H^n - \{x\}) =
    \begin{cases} \Z &\text{ si } p = n, x \in \interior{H^n} \\
                  0  &\text{ si } p \neq n, x \in \interior{H^n} \\
                  0  &\text{ si } x \in \partial H^n  \end{cases} \]
\end{lemma}

\begin{proof}
  Si $x \in \interior{H^n}$, por una proposición anterior, sabemos que
  $H_p(H^n, H^n - \{x\}) \cong H_p(\interior{H^n}, \interior{H^n} - \{x\}) \cong H_p(\R^n, \R^n - \{x\})$, pues
  $\interior{H^n} \cong \R^n$. Con esto obtenemos los valores para $x \in \interior{H^n}$.

  Si $x \in \partial H^n, H^n - \{x\}$ es estrellado desde un punto. Usando esto y la sucesión de homología del par $(H^n, H^n - \{x\})$,
  se obtiene la última parte.
\end{proof}

\begin{corollary}[Invarianza del borde]
  Sean $O_1 \subseteq H^n, O_2 \subseteq H^m$ abiertos homeomorfos. Entonces
  \begin{itemize}
    \item [a)] Si $\phi \colon O_1 \to O_2$ es el homeomorfismo, $\phi(O_1 \cap \partial H^n) = O_2 \cap \partial H^m$.
    \item [b)] $n = m$.
  \end{itemize}
\end{corollary}

\begin{proof}
  Si $x \in O_1 \cap \partial H^n$, por $x \in O_1$, se tiene $H_*(H^n, H^n - \{x\}) = 0$. \\
  $H_*(H^n, H^n - \{x\}) \cong H_*(O_1, O_1 - \{x\}) \cong H_*(O_2, O_2 - \{\phi(x)\}) \cong H_*(H^m, H^m - \{\phi(x)\})$.
  Así, $x \in O_1 \cap \partial H^n \iff \phi(x) \in O_2 \cap \partial H^m$, de donde se deduce a).

  Para demostrar b), solo hemos de aplicar a) para obtener que $\phi(O_1 \cap \interior{H^n}) = O_2 \cap \interior{H^m}$, y como
  $\interior{H^n}, \interior{H^m}$ son abiertos en $\R^n, \R^m$ respectivamente, el teorema de invarianza de la dimensión
  nos dice que $n = m$.
\end{proof}

\begin{definition}
  Una \textbf{variedad topológica con borde} es un espacio topológico Hausdorff $M$ verificando que para todo $p \in M$ existen
  $(V_p, \phi_p), n(p) \geq 1$ con $V_p$ entorno abierto de $p$, $\phi_p \colon V_p \to O_p \subseteq H^{n(p)}$ homeomorfismo
  sobre un abierto $O_p$ de $H^{n(p)}$.
\end{definition}

\begin{proposition}[Consecuencias]
  Se verifican los siguientes resultados sobre variedades topológicas con borde:
  \begin{itemize}
    \item[a)] En cada componente conexa de $M, n(p)$ es constante, y la llamaremos la dimensión de dicha componente.

    \item[b)] Si $\partial M = \{p \in M \mid H_*(M, M - \{p\}) = 0\}$, entonces
    \[ \partial M = \{p \in M \mid \forall (V_p, \phi_p), \phi_p(p) \in \partial H^n \cap O_p\} \]

    \item[c)] Si tomamos $\interior{M} = M - \partial M$, entonces $\interior{M}$ tiene estructura de variedad topológica
    (sin borde) $n$-dimensional, y $\partial M$ tiene estructura de variedad topológica con borde $(n-1)$-dimensional.
    Dicha estructura es la única que convierte a las inclusiones $\interior{M} \xhookrightarrow{} M, \partial M \xhookrightarrow{} M$
    en embebimientos. Además, $\partial M$ es cerrado de $M$ y, por tanto, $\interior{M}$ es abierto de $M$.
  \end{itemize}
\end{proposition}

\begin{proof}
  a) es trivial por una proposición ya vista.

  b) Si $(V_p, \phi_p)$ es una carta en $p$, entonces
  \[ H_*(M, M - \{p\}) \cong H_*(V_p, V_P - \{p\}) \cong H_*(O_p, O_p - \{\phi_p(p)\}) \cong H_*(H^n, H^n - \{\phi_p(p)\}) \]
  que vale $0$ sí y solo sí $\phi_p(p) \in \partial H^n$.

  c) Si $\phi \colon V \to O \subseteq H^n$ es una carta de M, entonces $\phi(V \cap \partial M) = O \cap \partial H^n$, y se tiene que
  $\phi(V \cap \interior{M}) = O \cap \interior{H^n}$, abierto en $O$, luego $V \cap \interior{M}$ es abierto en $V$, y por tanto en $M$.
  Así, $\interior{M}$ es abierto en $M$.

  Las estructuras se definen por:
  \begin{align*}
    \interior{M} &\colon \{(V \cap \interior{M}, \phi_{|V \cap \interior{M}}) \mid (V, \phi) \text{ es una carta en }M\}. \\
    \partial M &\colon   \{(V \cap \partial M,  \phi_{|V \cap \partial M}), \mid (V, \phi) \text{ es una carta en }M\}.
  \end{align*}

  El resto es trivial.
\end{proof}

Vamos a definir el concepto de variedad diferenciable.

\begin{definition}
  Una \textbf{variedad diferenciable con borde} $n$-dimensional es una variedad topológica con borde $n$-dimensional en la que
  los cambios de coordenadas son diferenciables.
\end{definition}

\begin{remark}
  Podemos extender la proposición anterior al caso de variedades diferenciables. Para ello, únicamente hemos de extender
  c) al caso diferenciable.
\end{remark}

\section{Separación de Jordan-Brower. Invarianza del dominio}

\begin{proposition}
  Sea $A \subseteq S^n$ homeomorfo a $I^k$, con $0 \leq k \leq n$. Entonces:
  \[ H_p(S^n - A) \cong \begin{cases} 0 &\text{ si } p > 0 \\ \Z &\text{ si } p = 0 \end{cases} \]
\end{proposition}

\begin{proof}
  Lo demostraremos por inducción sobre $k$.

  Si $k = 0, I^k = \{p\}$, luego $S^n - A \cong \R^n$, y se tiene el resultado.

  Supongamos el resultado cierto hasta $k-1$, y probémoslo para $k$.

  Sea $\phi \colon I^k \to A$ el homeomorfismo, y definamos
  \[I_+^k = \{x \in I^k \mid x_1 \geq \nicefrac{1}{2}\}$, \quad $I_-^k = \{x \in I^k \mid x_1 \leq \nicefrac{1}{2}\}.\]
  Sean $A^+, A^-$ las imágenes por $\phi$ de $I_+^k, I_-^k$ respectivamente.
  En $S^n - (A^+ \cap A^-)$ tomamos la descomposición $U = S^n - A^+, V = S^n - A^-$. Así,
  $U \cap V = S^n - A$, y por Mayer-Vietoris, se tiene:
  \[ H_p(S^n - (A^+ \cap A^-)) \to H_{p-1}(S^n - A) \xrightarrow{(i_*^+, i_*^-)} H_{p-1}(S^n - A^+) \oplus H_{p-1}(S^n - A^-) \dots \]

  Como $A^+ \cap A^- \cong I^{k-1}$, por hipótesis de inducción \\
  \[H_p(S^n - (A^+ \cap A^-)) \cong \begin{cases} 0 &\text{ si } p > 0 \\
                                                \Z &\text{ si } p = 0 \end{cases}\]
  Así, $(i_*^+, i_*^-)$ es un isomorfismo si $p > 0$, y un monomorfismo si $p = 0$.

  Supongamos ahora que el resultado no es cierto. Entonces, para $p > 0$ existe $[\alpha] \in H_p(S^n - A)$ con $[\alpha] \neq 0$, y
  si $p = 0$, existen $x, y \in S^n - A$ tales que $[x-y] \neq 0$.

  Para tratar ambos casos a la vez llamamos $\beta$ a $(x-y)$ o $\alpha$, indistintamente.

  Por lo que sabemos de $(i_*^+, i_*^-)$ se sigue que existe $A_1 \subseteq A, A_1 \cong I^k$ tal que
  $i_{1*}(\beta) \neq 0$, con $i_1 \colon S^n - A \xhookrightarrow{} S^n - A_1$.

  Nos basta con tomar $A_1 = A^+$ o $A_1 = A^-$, dependiendo de si $i_*^+(\beta) = 0$ o $i_*^-(\beta) = 0$.

  Aplicamos el mismo razonamiento a $A_1$, con lo que obtenemos una sucesión
  \[ A \supset A_1 \supset \dots \supset A_k \supset \dots \text{ con } A_i \cong I^k \hspace{0.2em} \forall i \]
  tal que $\bigcap\limits_{i = 1}^\infty A_i \cong I^{k-1}$, y tal que si $i_m \colon S^n - A \xhookrightarrow{} S^n - A_m$
  es la inclusión, entonces $i_{m*}(\beta) \neq 0$.

  Consideramos el diagrama de inclusiones:
  \[ \begin{tikzcd}
    S^n - A \arrow{dr}{i} \arrow{rr}{i_m} & & S^n - A_m \arrow{dl}{j_m} \\
    & S^n - \bigcup\limits_{i = 1}^\infty A_i
  \end{tikzcd} \]
  Como $H_p(S^n - \bigcap\limits_{i = 1}^\infty A_i) = \begin{cases} 0 &\text{ si } p > 0 \\ \Z &\text{ si } p = 0 \end{cases}$ por inducción, entonces \\
  $0 = i_*(\beta) = 0 = [i_\#(\beta)]$, luego existe $d \in S_{p+1}(\bigcup\limits_{i = 1}^\infty A_i)$ con $\partial d = i_\#(\beta)$.

  Si $d = \sum\limits_{i = 1}^r m_i \phi_i$, entonces $\bigcup\limits_{i = 1}^r \Img \phi_i$ es un compacto en $\bigcap\limits_{i = 1}^\infty A_i
  = \bigcup\limits_{i = 1}^\infty (S^n - A_i)$, luego por el teorema de Heine-Borel:
  \[\bigcup\limits_{i = 1}^r \Img \phi_i \subseteq (S^n - A_i) \cup \dots \cup (S^n - A_{ik}) = S^n - A_m \]
  con $m = \max\{i_1, \dots, i_k\}$.

  Así, existe $e \in S_{p+1}(S^n - A_m)$ con $j_{m\#(e)} = d$. Por tanto, $j_{m\#} \circ i_{m\#} (\beta) = i_\#(\beta) = \partial d
  = \partial j_{m\#}(e) = j_{m\#}(\partial e)$.

  Como $j_{m\#}$ es un monomorfismo, $i_{m\#}(\beta) = \partial e$, entonces $i_{m\#}(\beta) = 0$, lo que contradice lo afirmado.
\end{proof}

\begin{theorem}
  Sea $A \subseteq S^n$, $n \geq 2$, $A$ homeomorfo a $S^k$ con $0 \leq k < n$. Entonces:

  Si $k \leq n-2, \quad H_p(S^n - A) = \begin{cases} 0 &\text{ si } p \neq 0, n-k-1 \\ \Z &\text{ si } p = 0, p = n-k-1 \end{cases}$

  Si $k = n-1, \quad    H_p(S^n - A) = \begin{cases} 0 &\text{ si } p > 0 \\ \Z \oplus \Z &\text{ si } p = 0 \end{cases}$
\end{theorem}

\begin{proof}
  Lo probamos por inducción sobre $k$.

  Si $k = 0$, $S^0$ son dos puntos, luego $A$ es homeomorfo a dos puntos, y $S^n - A \cong \R^n - \{x\} \cong S^{n-1}$.
  Se comprueba que el resultado coincide con el que comocemos para el grupo de homología de las esferas.

  Supuesto cierto hasta $k - 1$, lo probamos para $k$.

  Sea $S^k = E_+^k \cap E_-^k$, con $E_+^k = \{x \in S^k \mid x_{k+1} \geq 0\}, E_-^k = \{x \in S^k \mid x_{k+1} \leq 0\}$.
  Entonces, $E_+^k$ y $E_-^k$ son homeomorfos a $I^k$, y $E_+^k \cap E_-^k \cong S^{k-1}$.

  Del resultado anterior y de la hipótesis de inducción, conocemos $H_*(S^k, E_+^k)$, $H_*(S^k, E_-^k)$ y $H_*(S^n - (E_+^k \cap E_-^k))$.
  Aplicando Mayer-Vietoris a $S^n - (E_+^k \cap E_-^k)$, con $U = S^n - E_+^k, V = S^n - E_-^k$, se tiene:
  \begin{align*}
    &H_p(S^n - E_+^k) \oplus H_p(S^n - E_-^k) \to H_p(S^n - (E_+^k \cap E_-^k)) \to H_{p-1}(S^n - S^k) \\
    &\to H_{p-1}(S^n - E_+^k) \oplus H_{p-1}(S^n - E_-^k)
  \end{align*}
  Si $p \geq 2$, se verifica $H_{p-1}(S^n - S^k) \cong H_p(S^n - (E_+^k \cap E_-^k)) \cong H_p(S^n - S^{k-1})$,
  que por inducción es $\begin{cases} 0 &\text{ si } p \neq 0, n-k \\ \Z &\text{ si } p = 0, n-k \end{cases}$

  Llamando $q = p-1$, si $k = n-1$, como $p \geq 2$, $H_q(S^n - A) = 0 \quad \forall q \geq 1$.\\
  Si $k \leq n-2, n-k \geq 2$, y $H_q(S^n - A) = \begin{cases} \Z &\quad q = n-k-1 \\ 0 &\quad \text{en otro caso.} \end{cases}$

  Para $p = 1 (q = 0)$, tomamos la sucesión exacta
  \begin{align*}
    &0 \to H_1(S^n - (E_+^k \cap E_-^k)) \to H_0(S^n - S^k) \to H_0(S^n - E_+^k) \oplus H_0(S^n - E_-^k) \to \\
    &\to H_0(S^n - (E_+^k \cap E_-^k)) \to 0
  \end{align*}
  Como $H_0(S^n - E_+^k), H_0(S^n - E_-^k)$ y $H_0(S^n - (E_+^k \cap E_-^k)) \cong \Z$,
  se deduce que $H_0(S^n - S^k) \cong \Z \oplus H_1(S^n - (E_+^k \cap E_-^k)) = \begin{cases} \Z \oplus \Z &\quad k = n-1 \\
                                                                                                    \Z &\quad k < n - 1  \end{cases} $
\end{proof}

\begin{theorem}[Separación de Jordan-Brower]
  Sea $f \colon S^{n-1} \to S^n$ un embebimiento. Entonces $S^n - f(S^{n-1})$ tiene dos componentes conexas,
  cuya frontera común es $f(S^{n-1})$.
\end{theorem}

\begin{proof}
  Sea $A = f(S^{n-1})$. $H_0(S^n - A) \cong \Z \oplus \Z$, por lo que $S^n - A$ tiene dos componentes arcoconexas,
  a las que llamamos $\Omega_1$ y $\Omega_2$.

  Como $S^n - A$ es un abierto de $S^n$, es localmente arcoconexo, luego sus componentes arcoconexas y conexas coinciden,
  y por tanto $\Omega_1$ y $\Omega_2$ son las componentes conexas de $S^n - A$.

  Como $S^n - A$ es localmente arcoconexo, sus componentes conexas son abiertos en $S^n -A$, y por tanto en $S^n$.
  De esta forma, $\Omega_i \cup A = S^n - \Omega_j, i \neq j$ es un cerrado en $S^n$, de donde se obtiene
  $\overline{\Omega_i} \subseteq \Omega_i \cup A$, y por tanto
  \[ Fr \Omega_i = \overline{\Omega_i} \cap \overline{S^n - \Omega_i} \subseteq (\Omega_i \cup A) \cap (S^n - \Omega_i) =
  (\Omega_i \cup A) \cap (\Omega_j \cup A) = A. \]
  Así, $Fr \Omega_i \subseteq A$.

  Veamos que $A \subseteq Fr \Omega_i$, lo que completa la demostración.

  Sean $a \in A, V$ entorno abierto de $a$ en $S^n$. Usando que $S^n$ es una variedad $n$-dimensional, se puede descomponer
  $A = A^1 \cup A^2$, con $a \in \interior{A^2} \subseteq V \cap A,$ y se verifica $A^1, A^2 \cong I^{n-1}$.

  Sean $p_i \in \Omega_i, p_i \in S^n - A \subseteq S^n - A^1$. Como $H_0(S^n - A^1) \cong \Z$, $S^n - A^1$ es arcoconexo,
  luego existe un arco $\gamma \colon [0,1] \to S^n - A^1$ que va desde $p_1$ a $p_2$.

  Por tanto, existe un $t \in [0,1]$ con $\gamma(t) \in A^2$, pues si no, la curva estaría en $S^n - (A^1 \cup A^2) = S^n - A$
  y uniría dos componentes conexas distintas.

  Como $\gamma^{-1}(A^2) \neq \emptyset$ y es cerrado, es un compacto de $[0,1]$, del que podemos tomar mínimo y máximo,
  a los que llamamos $t_m$ y $t_M$ respectivamente. Tomando $\epsilon$ suficientemente pequeño, $\gamma(t_m - \epsilon) \in \Omega_1 \cap V$,
  y $\gamma(t_M + \epsilon) \in \Omega_2 \cap V$, luego $a \in \overline{\Omega_1} \cap \overline{\Omega_2}, a \in Fr \Omega_i$.
\end{proof}

Este teorema nos permite obtener directamente un resultado que puede verse como una generalización del teorema
de la curva de Jordan a dimensiones superiores.

\begin{corollary}
  Sea $f \colon S^{n-1} \to \R^n$ un embebimiento. Entonces, $\R^n - f(S^{n-1})$ tiene dos componentes conexas,
  y $f(S^{n-1})$ es la frontera común de ambas.
\end{corollary}

\begin{theorem}[Invarianza del dominio]
  Sean $A_1, A_2 \subseteq S^n$ subconjuntos de $S^n$ homeomorfos por $h \colon A_1 \to A_2$. Entonces,
  $A_1$ es abierto de $S^n \iff A_2$ es abierto de $S_n$.
\end{theorem}

\begin{remark}
  Nótese que este hecho no es trivial. Lo sería en caso de que $A_1, A_2$ fueran cerrados, o si $h$ estuviera definida
  en todo $S^n$ y fuera un homeomorfismo entre ambos conjuntos.
\end{remark}

\begin{proof}
  Supongamos $A_1$ abierto, y veamos que $A_2$ también lo es. Sea $a_2 \in A_2$, y sea $a_1 \in A_1$ tal que $h(a_1) = a_2$.
  Existe un entorno de $a_1$, al que notamos $V_1 \subseteq S^n$, tal que $V_1 \subseteq A_1$ y $V_1 \cong D^n$. Sean
  $V_2 = h(V_1), W_2 = h(\partial V_1) (\partial V_1 \cong S^{n-1})$. Se verifica que $a_2 \in V_2 - W_2 \subseteq A_2$.

  Por el teorema de separación de Jordan-Brower, $S^n - W_2$ tiene dos componentes conexas, y al ser $V_2 \cong D^n$,
  $S^n - V_2$ es conexo. Además, $S^n - W_2 = (V_2 - W_3) \cup (S^n - V_2)$, siendo esta unión disjunta, luego $V_2 - W_2$
  y $S^n - V_2$ son las componentes conexas de $S^n - W_2$, y por tanto abiertos en $S^n - W_2$ (y en $S^n$).

  Como $a_2 \in V_2  W_2 \subseteq A_2$, $A_2$ es abierto.

  Para demostrar la otra implicación basta con intercambiar los papeles de $A_1$ y $A_2$ y considerar el homeomorfismo inverso.
\end{proof}

\section{Grado de una aplicación continua}

Vamos a definir el grado de una aplicación continua de una esfera en sí misma, aprovechando lo que sabemos
de los grupos de homología de las esferas.

Sea $f \colon S^n \to S^n$ una aplicaciones continua. Los únicos homomorfismos que induce f y que son no triviales son
$f_* \colon H_0(S^n) \to H_0(S^n)$ y $f_* \colon H_n(S^n) \to H_n(S^n)$.

Se cumple que el primero de ellos es siempre la identidad. En el segundo caso, como $H_n(S^n) \cong \Z$, si $[\alpha]$ es un
generador de $H_n(S^n)$, necesariamente se tiene que $f_*([\alpha]) = n \times [\alpha] \quad n \in Z$.

Definimos como el \textbf{grado} de $f$, y lo notamos $\Deg f$, a dicho entero $n$.
\[ f_*([\alpha]) = \Deg f \times [\alpha] \]
Veamos algunas de sus propiedades.

\begin{proposition}[Propiedades]
  Estando en el escenario anterior, se verifican:
  \begin{itemize}
    \item[a)] $\Deg Id = 1$.
    \item[b)] $\Deg (g \circ f) = \Deg g \times \Deg f$.
    \item[c)] $f \simeq g \implies \Deg f = \Deg g$. La implicación contraria también es cierta, el llamado Teorema de Hopf.
    \item[d)] Si $f$ es una equivalencia homotópica, $\Deg f = \pm 1$ (en particular, es cierto si $f$ es homeomorfismo).
    \item[e)] Si $\Deg f \neq 0 \implies f$ es sobreyectiva.
  \end{itemize}
\end{proposition}

\begin{proof}
  Las demostraciones son sencillas usando la definición. Probamos únicamente e).

  Supongamos que $f$ no es sobreyectiva, luego existe $x \in S^n$ con $x \notin \Img f$. Así, tenemos el diagrama
  \[ \begin{tikzcd}
    S^n  \arrow{dr}{f'} \arrow{rr}{f} & & S^n  \\
    & S^n - \{x\} \arrow[hookrightarrow]{ur}{i}
  \end{tikzcd} \]
  que induce el siguiente diagrama en la homología:
  \[ \begin{tikzcd}
    H_n(S^n)  \arrow{dr}{f'_*} \arrow{rr}{f_*} & & H_n(S^n)  \\
    & H_n(S^n - \{x\}) \cong 0 \arrow[hookrightarrow]{ur}{i_*}
  \end{tikzcd} \]
  que nos da una descomposición de $f_*$ que tiene al la función nula como factor, lo cual es imposible.
\end{proof}

\begin{lemma}
  Si $\phi \colon S^n \to S^n$ es una isometría, entonces $\Deg \phi = \Det \phi$, viendo $\phi$ como una matriz cuadrada
  de orden $n+1$.
\end{lemma}

\begin{proof}
  Por el teorema de Cartan-Dieudonné, sabemos que toda isometría en $\R^m$ puede expresarse como composición de
  $k$ simetrías, con $k \leq m$. Por tanto, podemos tomar $\phi = \sigma_i \circ \dots \circ \sigma_k, k \leq n+1$.

  Como $\Deg \phi = \prod\limits_{i = 1}^k \sigma_i$, y $\Det \phi = \prod\limits_{i = 1}^k \sigma_i$, basta con probarlo para simetrías.

  Sea $\sigma \colon S^n \to S^n$ una simetría respecto al hiperplano $H$, y sea $A \colon \R^{n+1} \to \R^{n+1}$ la isometría
  que lleva $H$ en el hiperplano $\{x \in \R^{n+1} \mid x_1 = 0\}$. Entonces, si tomamos la aplicación que cambia el signo a la primera
  coordenada, $\tau(x_1, \dots x_{n+1}) = (-x_1, \dots, x_{n+1})$, se verifica que $\sigma = A^{-1} \circ \tau \circ A$, luego
  $\Deg \sigma = \Deg A^{-1} \times \Deg \tau \times \Deg A = \Deg \tau$.

  Por tanto, basta probar que $\Deg \tau = \Det \tau = -1$. Lo hacemos por inducción.

  Si $n = 1$, tomamos $U = S^1 - (0, 1), V = S^1 - (0, -1)$. Entonces, el homomorfismo de conexión de la sucesión
  de Mayer-Vietoris asociada (que notábamos $\Delta$) está dado por $\Delta([\alpha]) = [x-y]$, siendo $[\alpha]$ un generador de $H_1(S^1)$.
  Por la naturalidad de dicha sucesión respecto a $\tau$, se tiene la conmutatividad del siguiente diagrama:
  \[ \begin{tikzcd}
    H_1(S^1) \arrow{r}{\Delta} \arrow{d}{\tau_*} & H_0(U \cap V) \arrow{d}{\tau_*} \\
    H_1(S^1) \arrow{r}{\Delta} & H_0(U \cap V)
  \end{tikzcd} \]

  Así, tenemos las dos siguientes cadenas de igualdades:
  \[\tau_*(\Delta([\alpha])) = \Delta(\tau_*([\alpha])) = \Delta(\Deg \tau \times [\alpha]) = \Deg \tau \times [x-y]\]
  \[\tau_*(\Delta([\alpha])) = \tau_*([x-y]) = [\tau_\#(x) - \tau_\#(y)] = [y - x] \]
  de donde se deduce $\Deg \tau = -1$.

  Supongamos ahora que el resultado es cierto hasta $n-1$, y lo probamos para $n$. Denotaremos como $\tau_{n-1}$ y $\tau_{n}$
  las correspondientes aplicaciones en cada espacio, para evitar confusión.

  Considerando en $S^n$ la misma descomposición anterior, se tiene:
  \[ \begin{tikzcd}
    H_n(S^n) \arrow{r}{\Delta}[swap]{\cong} \arrow{d}{\tau_{n*}} & H_{n-1}(U \cap V) \arrow{d}{\tau_{n*}} & \arrow{l}{\cong}[swap]{i_*} H_{n-1}(S^{n-1}) \arrow{d}{\tau_{n-1*}} \\
    H_n(S^n) \arrow{r}{\Delta}[swap]{\cong} & H_{n-1}(U \cap V) & \arrow{l}{\cong}[swap]{i_*} H_{n-1}(S^{n-1})
  \end{tikzcd} \]
  $i_*$ es un isomorfismo, pues $S^{n-1}$ es un retracto de $U \cap V$.

  Así, $\tau_{n*}([\alpha]) = \Deg \tau_n \times [\alpha]$ y además:

  $\tau_{n*}([\alpha]) = \Delta^{-1} \tau_{n*} \Delta([\alpha]) = (\Delta^{-1} i_* \tau_{n-1*} i_*^{-1} \Delta)([\alpha])
  = (\Delta^{-1} i_* i_*^{-1} \Delta)(-[\alpha]) = -[\alpha]$.

  Por tanto, $\Deg \tau_n = -1$.
\end{proof}

Como consecuencia directa de este hecho, si consideramos la aplicación antípoda $-Id \colon S^n \to S^n$, entonces
$\Deg -Id = \Det -Id = (-1)^{n+1}$.

Veamos un lema que nos permitirá demostrar dos resultados interesantes de aplicaciones sobre esferas de dimensión par,
con los que terminaremos el capítulo.

\begin{lemma}
  Sean $f, g \colon S^n \to S^n$ aplicaciones tales que $f(x) \neq g(x) \hspace{0.2em} \forall x \in S^n$.
  Entonces $f \simeq -g$.
\end{lemma}

\begin{proof}
  Basta ver que $H(x, t) = \frac{t f(x) - (1-t)g(x)}{|t f(x) - (1-t)g(x)|}$ es la homotopía buscada.
\end{proof}

\begin{corollary}
  Si $f \colon S^{2n} \to S^{2n}$ es continua, entonces existe $x \in S^{2n}$ tal que $f(x) = x$ o $f(x) = -x$.
\end{corollary}

\begin{proof}
  Supongamos que existe $f$ tal que para todo $x \in S^{2n}$, $f(x) \neq x$ y $f(x) \neq -x$. Entonces, por
  el lema anterior, $f \simeq -Id$ y $f \simeq Id$, luego $Id \simeq -Id$ y $\Deg Id = \Deg -Id$, lo que implica
  $1 = -1$ contradicción.
\end{proof}

\begin{corollary}
  No existen campos continuos sin ceros en una esfera de dimensión par.
\end{corollary}

\begin{proof}
  Recordemos que el tangente a $x \in S^{2n}$ es $\langle x \rangle^\perp$. Por tanto, un campo continuo en $S^{2n}$ es una aplicación
  continua $f \colon S^{2n} \to \R^{2n+1}$ tal que $\langle f(x), x\rangle = 0 \hspace{0.2em} \forall x \in S^{2n}$.

  Si dicho campo no tiene ceros, sea $g \colon S^{2n} \to S^{2n}$ dada por $g(x) = \frac{f(x)}{|f(x)|}$. Como $g$ es
  continua, existe $x$ con $g(x) = x$ o $g(x) = -x$. Sin embargo, $\langle g(x), x\rangle = 0$ para todo $x$, lo cual nos lleva a una
  contradicción, pues los elementos de $S^{2n}$ no pueden tener norma nula.
\end{proof}
