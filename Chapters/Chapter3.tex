%************************************************
\chapter{Introducción}\label{ch:resultados}
%************************************************

\section{Definiciones iniciales}

Vamos a utilizar el cálculo de la homología de las esferas para obtener resultados
de carácter topológico.

\begin{proposition}
  a) Esferas de distinta dimensión no tienen el mismo tipo de homotopía. En particular, no son homeomorfas.

  b) Espacios euclídeos de distinta dimensión no son homeomorfos.
\end{proposition}

\begin{proof}
  a) Sean $S^n, S^m$ con $n \neq m$. Si $S^n \cong S^m$, entonces $H_n(S^n) \cong H_n(S^m)$, pero hemos visto que eso no sucede.

  b) Si $n \neq m$, y $h \colon \R^n \to \R^m$ es un homeomorfismo, entonces su restricción a $R^n - \{0\}$ también lo es.
  Como $R^n - \{0\} \cong S^n$, y $R^m - \{f(0)\} \cong S^m$, llegamos de nuevo a una contradicción.
\end{proof}

\begin{remark}
  $\R^n$ y $\R^m$ tienen el mismo tipo de homotopía. Si $n \leq m$, podemos ver $\R^m$ como \\
  $\R^m = \{(x, y) \mid x \in \R^n, y \in \R^{m-n}\}$, y se define
  \[ H((x, y), t) = (1-t)(x, y) + t(x, 0) \]
  homotopía entre ambos espacios, siendo $\R^n$ un retracto de $\R^m$.
\end{remark}

\begin{lemma}
  No hay ninguna retracción de $D^n$ a $S^{n-1}$.
\end{lemma}

\begin{proof}
  Sean $D^n = \{x \in \R^n \mid |x| \leq 1\}, S^{n-1} = \{x \in \R^n \mid |x| = 1\} \subseteq D^n$. Si existiera dicha retracción,
  $r \colon D^n \to S^{n-1}$, tenemos el siguiente diagrama conmutativo:
  \[ \begin{tikzcd}
    S^{n-1} \arrow{dr}{i} \arrow{rr}{Id} & & S^{n-1} \\
    & D^n \arrow{ur}{r}
  \end{tikzcd} \]
  de donde se obtiene un diagrama, también conmutativo, en la homología:
  \[ \begin{tikzcd}
    Z \cong H_{n-1}(S^{n-1}) \arrow{dr}{i_*} \arrow{rr}{Id_*} & & H_{n-1}(S^{n-1}) \\
    & H_{n-1}(D^n) \arrow{ur}{r_*}
  \end{tikzcd} \]
  lo cual es imposible, pues $H_{n-1}(D^n) \cong 0$, y tendríamos una factorización de la identidad que incluye a la aplicación nula,
  que en este caso sería $i_*$.
\end{proof}

\begin{corollary}
  Sea $f \colon D^n \to \R^n$ continua, tal que la restricción de $f$ a $S^{n-1}$ sea la identidad.
  Entonces existe $x \in D^n$ tal que $f(x) = 0$.
\end{corollary}

\begin{proof}
  Supongamos que no existe dicho $x$. En tal caso, es posible definir $h(x) = \frac{f(x)}{|f(x)|}$
  continua, que contradice el resultado anterior.
\end{proof}

Vamos a demostrar un teorema que nos dará como consecuencia directa un resultado clásico, el teorema
del punto fijo de Brower.

\begin{theorem}
  Sea $f \colon D^n \to \R^n$ continua. Entonces o bien existe $y \in D^n$ tal que $f(y) = 0$, o existe
  $z \in S^{n-1}$ con $f(z) = \lambda z, \quad \lambda < 0$.
\end{theorem}

\begin{proof}
  Sea $h \colon D^n \to \R^n$, definida por:
  \[ h(x) = \begin{cases} f(2x) &\text{ si } |x| \leq \frac{1}{2} \\
                          (2-2|x|)f(\frac{x}{|x|}) + (2|x| + 1)x &\text{ si } |x| \geq \frac{1}{2}  \end{cases} \]
  que es continua, ya que en $\frac{1}{2}$ ambas definiciones coinciden.

  Si $|x| = 1, h(x) = x$, y aplicando el corolario anterior, existe $y \in D^n$ tal que $h(y) = 0$. Distinguimos dos casos:

  Si $|y| \leq \frac{1}{2}, 0 = f(2y)$, y estamos en el primer caso.

  Si $|y| > \frac{1}{2}, f(\frac{x}{|x|}) = \frac{(-2|x| + 1)x}{2-2|x|}$. Si llamamos $z = \frac{x}{|x|}$,
  $f(z) = \frac{(-2|x|+1)|x|}{2-2|x|} z$, por lo que estamos en el segundo caso del teorema.
\end{proof}

\begin{corollary}
  Sea $f \colon D^n \to \R^n$ continua. Entonces existe $y \in D^n$ tal que $f(y) = y$ o existe $z \in S^{n-1}$ con $f(z) = \mu z, \mu > 1$.
\end{corollary}

\begin{proof}
  Aplicamos el teorema anterior a $-f + Id$.
\end{proof}

\begin{corollary}[Teorema del punto fijo de Brower]
  Toda aplicación continua de $D^n$ en sí mismo tiene un punto fijo.
\end{corollary}

\begin{proof}
  Basta ver que no puede darse la segunda opción del corolario anterior, pues la imagen de la función caería fuera de $D^n$.
\end{proof}

Pasemos ahora a la definición del grupo de homología local, herramienta que nos servirá para demostrar varios resultados de invarianza.

\begin{definition}
  Sean $\X$ un espacio topológico, $x \in \X$. A $H_p(\X, \X - \{x\})$ lo llamaremos el grupo de homología local de $\X$ en $x$.
\end{definition}

\begin{proposition}
  Sea $x \in \X$, y supongamos que $\{x\}$ es un cerrado. Entonces, para todo entorno $V$ de $x$, se verifica:
  \[ H_p(\X, \X - \{x\}) \cong H_p(V, V - \{x\}) \quad \forall p \geq 0 \]
\end{proposition}

\begin{proof}
  Sea $U = \X - V \subseteq \X - \{x\}$. Si tomamos el cierre, se tiene
  \[ \overline{U} = \overline{\X - V} = \X - \interior{V} \subseteq \X - x = \X - \overline{x} = \interior{(\X - x)} .\]
  Aplicando el teorema de escisión, $i \colon (\X - U, (\X - \{x\}) - U) \to (\X, \X - \{x\})$ induce un isomorfismo en la homología.
  Como $\X - U = V, (\X - \{x\}) - U = V - \{x\}$, obtenemos el resultado.
\end{proof}

\begin{proposition}
  Sean $\X, \Y$ espacios topológicos, $x \in \X, y \in \Y$ cerrados. Si existen entornos $x \in V, y \in W$ tales que $(V, \{x\}) \cong (W, \{y\})$,
  entonces $H_*(\X, \X - \{x\}) \cong H_*(\Y, \Y - \{y\})$.
\end{proposition}

\begin{proof}
  Sea $h \colon V \to W$ el homeomorfismo entre $V$ y $W$, con $h(x) = y$. La restricción $h \colon V - \{x\} \to W - \{y\}$ también es un homeomorfismo,
  por lo que $H_*(V, V - \{x\}) \cong H_*(W, W - \{y\})$: Basta aplicar la proposición anterior para obtener el resultado.
\end{proof}

Vamos a calcular la homología local de $\R^n$.

\begin{proposition}
  $H_p(\R^n, \R^n - \{x\}) \cong \begin{cases} 0 &\text{ si } n \neq p \\ \Z &\text{ si } n = p \end{cases}$
\end{proposition}

\begin{proof}
  Si $p = 0$, ya sabemos que $H_0(\R^n, \R^n - \{x\}) = 0$. Supondremos $p \geq 1$.

  Consideramos la sucesión de homología del par, teniendo en cuenta que $\R^n - \{x\}$ y $S^{n-1}$ tienen el mismo tipo de homotopía. \\
  Si $p = 1$:
  \[ 0 = H_1(\R^n) \xrightarrow{\pi_*} H_1(\R^n, \R^n - \{x\}) \xrightarrow{\Delta} H_0(\R^n - \{x\}) \xrightarrow{i_*} H_0(\R^n) = \Z \to 0 \]
  Si $n \geq 2, H_0(\R^n - \{x\}) \cong \Z$. \\
  Como $\Ker \Delta = \Img \pi_* = 0, \Img \Delta = \Ker i_* = 0, H_1(\R^n, \R^n - \{x\}) = 0$. \\
  Si $n = 1, H_0(\R^n - \{x\}) \cong \Z \oplus \Z$, luego $H_1(\R^n, \R^n - \{x\}) \cong \Z$.

  Si $p \geq 2$:
  \[ 0 = H_p(\R^n) \xrightarrow{\pi_*} H_p(\R^n, \R^n - \{x\}) \xrightarrow{\Delta} H_{p-1}(\R^n - \{x\}) \xrightarrow{i_*} H_{p-1}(\R^n) = 0 \]
  Si $p \neq n, H_{p-1}(\R^n - \{x\} = 0$, luego $H_p(\R^n, \R^n - \{x\}) = 0$. \\
  Si $p = n, H_{p-1}(\R^n - \{x\} \cong \Z$, luego  $H_p(\R^n, \R^n - \{x\}) = \Z$. 
\end{proof}
