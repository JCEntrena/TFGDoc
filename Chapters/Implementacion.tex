%************************************************
\chapter{Implementación}\label{ch:implementacion}
%************************************************

% Hablar de la implementación de los algoritmos, involucrando a Ruby.

En este capítulo vamos a tratar la implementación del programa encargado de
aplicar los algoritmos al conjunto de instancias. Comenzaremos introduciendo
el lenguaje de programación utilizado, siguiendo con una descripción de la
funcionalidad completa del programa, y concluyendo con un desglose de carácter
más técnico de las partes que lo componen.

\section{Lenguaje utilizado}

La implementación del código de este trabajo ha sido realizada en el lenguaje de
programación \textbf{Ruby}, usando su versión 2.3.0. Ruby es un lenguaje orientado
a objetos, interpretado, con una sintaxis muy similar a la de \textbf{Python}.
Está disponible para descarga bajo una licencia de software libre.

He elegido este lenguaje por ser capaz de aportar simplicidad e intuitividad a
la hora de programar los algoritmos. Como ya había trabajado con este lenguaje en
ocasiones anteriores, conocía sus puntos fuertes, y he tratado de explotarlos
lo máximo posible, para que el resultado sea un código legible y adaptable.

%Además, sale de los lenguajes más utilizados en la carrera, lo que me permite incluir una herramienta menos usual.

Al utilizar un lenguaje interpretado como es Ruby, cabe esperar unos
tiempos de ejecución mayores que con otros lenguajes, debido a que no está hecho
para ejecutar código lo más rápido posible. En el caso de querer realizar un programa
capaz de ejecutar los algoritmos en el menor tiempo posible, lenguajes compilados
como \textbf{C++} serían más útiles que Ruby.

\section{Funcionalidad}

El código implementado cumple varias funciones: la primera de ellas es llevar a
memoria todas las instancias disponibles del problema. Para ello, he tenido que
implementar un lector de ficheros que cree una matriz de adyacencia para cada archivo
disponible. Estos archivos siguen una estructura determinada, que permite
automatizar dicho proceso, no teniendo que considerar casos independientes.
Cada archivo dará lugar a un problema, compuesto de una matriz de adyacencias,
el número de nodos y aristas involucrados, y el nombre del grafo en cuestión.

El segundo punto, y más importante, es la implementación de las heurísticas.
Cada una de ellas está contenida en una clase propia, que contendrá uno o varios
métodos para la resolución de los problemas generados a partir de los archivos.
Adicionalmente, podrán contener métodos auxiliares para uso propio o de otras
heurísticas, en caso de que utilicemos técnicas híbridas.

Finalmente, existe un conjunto de funciones auxiliares, que pueden ser
requeridas por varias heurísticas. La idea bajo su creación es encapsular
operaciones importantes, que serán repetidas frecuentemente y en diversos
escenarios, evitando repetición de código y sirviendo de ayuda para facilitar
el trabajo y para la comprensión del código.


\section{Inicialización y ejecución del programa}

Antes de aplicar ningún algoritmo al conjunto de instancias, hay que resolver la
lectura de los archivos y la creación de las estructuras de datos correspondientes
para cada grafo. Estas operaciones se engloban en un método llamado \textit{readerMethod},
que se encuentra en el archivo \textit{reader.rb}.

Cada instancia del problema dará lugar a un objeto de la clase \textbf{Problem}, que
almacena los datos de cada grafo: su nombre, número de vértices, número de aristas,
su matriz de adyacencias y el orden de cada nodo, esto es, su número de adyacencias.

El método de lectura de los archivos recorre el fichero en el que están almacenadas
las instancias del problema, y por cada uno de ellos construye un objeto de la clase
Problem. Dado que los ficheros siguen todos el mismo formato, es posible extraer
los datos de cada grafo sin más que hacer una distinción de casos.

La parte fundamental de la lectura, que es la creación de la matriz de adyacencias,
se realiza partiendo de una matriz de ceros del tamaño adecuado, e iremos tomando
los pares que se proporcionan en el archivo, que representan los nodos que están
conectados, para marcarlos como unos en la matriz de adyacencias. Dado que los
nodos comienzan su numeración desde el $1$ en los ficheros, y desde el $0$ en el
programa (debido a que las estructuras de datos comienzan en el $0$), hemos de
corregirlo a la hora de marcar las aristas en la matriz de adyacencias. Además,
marcaremos todos los nodos como conectados con ellos mismos, pues será necesario
para el correcto funcionamiento de algunas operaciones. Esto hace que todos los
nodos de un clique pertenezcan a su conjunto $C_0$ asociado, por lo que cuando lo
calculemos tendremos que quitar los nodos del propio clique.

El número de adyacencias de cada nodo no lo proporciona el archivo del grafo,
sino que hemos de calcularlo. Para ello, nos valemos de una función auxiliar
programada con tal fin.

El archivo \textit{reader.rb} también contendrá el método principal. En él, se
ejecutarán el método de lectura y los algoritmos implementados que incluyamos,
para lo que se debe añadir el código necesario. Además, se mostrarán por pantalla
los resultados obtenidos, con la idea de poder redireccionarlos a un archivo de
texto en el que almacenar los resultados.


\section{Algoritmos}

Todos los algoritmos implementados en el programa se encuentran recogidos en
un mismo módulo, el módulo \textbf{Clique}, bajo el que también se encuentra el
archivo \textit{reader.rb} detallado con anterioridad. Cada tipo de algoritmo
estará dentro de una clase, y se separarán los distintos algoritmos del mismo
tipo en métodos diferentes.

\subsection{Consideraciones}

Algunos algoritmos usan aleatoriedad en una o varias fases. Para conseguirla, se
ha utilizado la clase \textbf{Random}, que proporciona un generador de números
aleatorios. El método \textit{rand} nos permite obtener un valor aleatorio
dentro de un rango dado, pudiendo ser un número entero o real, lo que nos permite
cubrir todas las necesidades en este aspecto.

% Hablar de que no se usa información del grafo para construir una clase.

\subsection{Algoritmos \textit{greedy}}

Los dos algoritmos \textit{greedy} implementados se encuentran en el archivo
\textit{greedy.rb}, dentro de la clase \textbf{Greedy}. El algoritmo \textit{greedy}
básico está en el método \textit{solve}, mientras que el \textit{greedy} adaptativo
se encuentra en el método \textit{solve2}. Ambos métodos reciben únicamente una
instancia de la clase Problem como parámetro, el problema a resolver.

Además, en este archivo también se encuentran los distintos algoritmos voraces
encargados de completar cliques o de reparar grafos, utilizados en otras heurísticas.
Son los siguientes:

\begin{itemize}
  \item \textit{complete\_clique}, método que amplía un clique hasta que sea posible,
        tomando de entre los candidatos aquel con más adyacencias.
  \item \textit{complete\_clique\_random}, que, al igual que el anterior, amplía
        un clique hasta que sea maximal, solo que en este caso elige de forma aleatoria.
  \item \textit{solve\_random}, método que crea un clique partiendo desde cero, usando
        \textit{complete\_clique\_random}. Lo usaremos para generar soluciones iniciales
        aleatorias.
  \item \textit{repair}, método que elimina nodos de un grafo hasta que sea un clique.
        Elimina el nodo con menor número de adyacencias.
  \item \textit{repair\_random}, con la misma función que el anterior, pero en este
        caso elimina un nodo aleatorio.
\end{itemize}

\subsection{Búsqueda local}

Los dos algoritmos de búsqueda local están en el archivo \textit{localsearch.rb},
bajo la clase \textbf{LocalSearch}. Para la creación de la clase, solo es necesario
crear un generador de números aleatorio, por lo que no hay que pasarle argumentos
al constructor.

La heurística 1LS está implementada en el método \textit{solve\_with\_solution}, y
recibe como parámetro el elemento de la clase Problem a resolver, una solución
de partida, y un límite de cambios.

Por otra parte, la heurística DLS se encuentra en el método \textit{solve\_dynamic}.
Los parámetros serán los mismos que en el caso anterior: la instancia de la clase
Problem a resolver, la solución de partida y el límite de cambios.

Para la ejecución de cualquiera de los dos métodos partiendo desde el clique
vacío, existe el método \textit{solve}, que será el que llamemos cuando queramos
ejecutar uno de los dos algoritmos. Recibe como parámetros el problema que resolver
y el número de cambios, que los pasará al algoritmo que ejecute, además del clique
vacío como solución de partida.

\subsection{Enfriamiento simulado}

En el archivo \textit{simulatedannealing.rb} se encuentran los dos algoritmo de
enfriamiento simulado considerados, en la clase \textbf{SA}. El constructor de
clase no necesita parámetros, pues crea un generador de números aleatorios y
una instancia de la clase Greedy, ya que el algoritmo de Geng et al. necesita
las funciones para reparar y completar grafos.

El algoritmo de enfriamiento simulado básico está implementado en el método
\textit{solve}, que recibe como único parámetro el problema a resolver. Para el
criterio de aceptación del algoritmo Metrópolis, genera un número aleatorio entre
$0$ y $1$, y si es menor que el valor de la probabilidad, acepta la solución.
Para tomar elementos aleatorios del entorno, se utiliza la función \textit{shuffle},
que baraja los elementos del entorno, y toma un número aleatorio como semilla.

El algoritmo de Geng et al. se encuentra en el método \textit{solve2}, que vuelve
a tener como único parámetro el problema que se quiere resolver. Efectúa las mismas
operaciones para el criterio de Metrópolis y para el baraje del entorno.

Ambas funciones declaran los valores de las temperaturas y del parámetro $\beta$
al inicio. Para cambialos, basta con modificar dichos valores en el código.

\subsection{GRASP}

El algoritmo GRASP está implementado en el archivo \textit{grasp.rb}, dentro de la
clase \textbf{GRASP}. Además de un generador de números aleatorios, el constructor
declara un objeto de la clase LocalSearch, que necesitaremos para aplicar el
algoritmo de búsqueda local que consideremos.

El método \textit{generate\_random\_solution} es el encargado de generar las soluciones
iniciales pseudoaleatorias, tomando un problema como argumento. El tamaño de la
lista restringida de candidatos está fijado por un valor en el código, que debemos
cambiar en caso de querer aumentar o reducir el tamaño de dicha lista.

Como los dos algoritmos implementados solo difieren en la búsqueda local utilizada,
es posible implementar ambos utilizando un único método, y cambiar la búsqueda local
para alternar entre uno u otro. El método \textit{solve} es el encargado de esto,
recibiendo un problema y un número de iteraciones como parámetros. El número de
cambios para el algoritmo de búsqueda local está declarado como una variable, que
deberemos modificar si queremos variar su valor.

\subsection{ILS}

El algoritmo ILS se encuentra en el archivo \textit{ils.rb}, en la clase \textbf{ILS}.
Su construcción no necesita parámetros, y se encarga de crear el generador de números
aleatorios y un objeto de la clase LocalSearch para las búsquedas locales.

La clase tiene un único método, \textit{solve}, que es el encargado de aplicar la
búsqueda local y las permiutaciones a las soluciones obtenidas. Recibe el problema
a resolver y el número de iteraciones. El límite para la búsqueda local es una
variable, que modificaremos de ser necesario. Al igual que en el caso de GRASP,
para cambiar la búsqueda local que aplicamos basta con cambiar la llamada al método
de búsqueda local.

\subsection{Algoritmos de colonia de hormigas}

Los dos algoritmos de colonia de hormigas implementados están en el archivo \textit{aco.rb},
dentro de la clase \textbf{ACO}. El constructor no recibe ningún parámetro, pues solo
crea un generador de números aleatorios.

Mi versión del algoritmo está en el método \textit{solve}, que recibe el problema que
se quiere resolver y el número de iteraciones. El número de hormigas y el parámetro
$\beta$, necesario para la evaporación de feromonas, se encuentran en el código,
pudiendo cambiarse su valor de ser necesario.

Para elegir cada nodo al que va una hormiga, una vez calculadas las probabilidades,
se genera un valor aleatorio entre $0$ y $1$, y vamos tomando nodos, acumulando su
valor de probabilidad, hasta que dicho valor supere el valor aleatorio generado.

El algoritmo de Xu et al. se encuentra en el método \textit{solve2}, y, debido a que
su estructura es igual que el algoritmo anterior, salvo el cálculo de las probabilidades,
se le puede aplicar todo lo mencionado anteriormente.

\subsection{Algoritmos genéticos}

Tanto el algoritmo genético como el memético implementados se encuentran dentro de
la clase \textbf{Genetics}, en el archivo \textit{genetics.rb}. El constructor
de la clase no recibe ningún parámetro, y crea un generador de números aleatorios,
un elemento de la clase LocalSearch y uno de la clase Greedy, que se usarán en el
algoritmo memético y para completar o reparar las soluciones, respectivamente.

El algoritmo genético de Zhang et al. está implementado en el método \textit{solve},
que recibe el problema a resolver y el número de iteraciones. Tanto las probabilidades
de cruce y mutación como el tamaño de la población se encuentran en el código, donde
deben ser modificadas de ser necesario. Como en ocasiones anteriores, para decidir
eventos que dependen de una probabilidad, se genera un número entre $0$ y $1$, y si
este es menor que la probabilidad, el suceso se lleva a cabo. Este proceso se lleva
a cabo en los mecanismo de cruce y de mutación, y además se usa también en el propio
operador de cruce, donde los nodos que pertenecen solo a un padre van a cada uno de
los hijos con probabilidad $0.5$.

El algoritmo memético, implementado en \textit{solve\_memetic}, funciona de la misma
forma, aplicando búsqueda local al generar la nueva población, con un límite que
también está en el código. Por lo demás, opera  igual que el genético, por lo que
son aplicables las consideraciones anteriores.

Además, existe un método privado para las mutaciones, llamado \textit{mutation}.
Recibe como parámetro la solución a mutar y la matriz de adyacencias, y devuelve
la solución mutada.


\section{Métodos auxiliares}

Además de los algoritmos, también se han implementado funciones que realizan
cálculos necesarios en varios de los algoritmos, que han sido incluidas en un
archivo adicional creado para almacenarlas. Estas funciones no se encuentran
dentro de una clase, pero sí que están en un módulo distinto, el módulo \textbf{Algorithm},
para diferenciarlas de las implementaciones de las heurísticas. Todas estas funciones
se encuentran en el archivo \textit{algorithm.rb}, y las detallamos a continuación:

\begin{itemize}
  \item \textit{add}: Operador \textit{add}. Recibe un clique y un vértice, y añade
        el vértice al clique. Supone que el vértice no está en el clique y que
        se mantiene la estructura.

  \item \textit{swap}: Operador \textit{swap}. Recibe un clique, un vértice $v$ del conjunto
        $C_1$, y la matriz de adyacencias. Busca el nodo dentro del clique que no está
        conectado a $v$ y lo intercambia con $v$. De nuevo, suponemos que $v \in C_1$,
        por lo que se puede hacer el intercambio.

  \item \textit{swap\_two}: Operador \textit{swap}. En este caso, se pasan como parámetros
        el clique y los dos nodos a intercambiar, $p$ en $C_1$ y $q$ dentro del clique.
        El método elimina $q$ del clique y mete $p$, suponiendo que se verifica lo
        anterior y que $q$ es el único nodo del clique no conectado a $p$.

  \item \textit{drop}: Operador \textit{drop}. Recibe un clique y un nodo, y elimina
        dicho nodo del clique.

  \item \textit{adjacencies}: Método que recibe un nodo y una matriz de adyacencias,
        y devuelve el número de adyacencias de dicho nodo. Se utiliza para calcular
        las adyacencias de todos los nodos al crear un problema.

  \item \textit{connections}: Dados un nodo $n$, un conjunto de nodos $S$ y una matriz de
        adyacencias, este método devuelve el número de nodos de $S$ conectados con $c$.

  \item \textit{is\_connected}: Recibe un nodo $n$, un conjunto de nodos $S$ y una matriz
        de adyacencias, y devuelve si $n$ está conectado con todos los nodos de $S$.
        Este método nos servirá para calcular el conjunto $C_0$.

  \item \textit{is\_connected\_but\_one}: Funciona igual que el anterior, pero comprueba
        que el nodo $n$ esté conectado con todos los nodos de $S$ salvo uno. Nos servirá
        para el cálculo del conjunto $C_1$.

  \item \textit{connected}: Dados un nodo y una matriz de adyacencias, devuelve todos los
        nodos del grafo conectados a él.

  \item \textit{connected\_with\_all}: Dado un conjunto de nodos $S$ y una matriz de adyacencias,
          usando el método \textit{is\_connected}, devuelve el conjunto $M$, formado por los nodos
        conectados con todos los de $S$, y elimina el conjunto $S$ de $M$, pues si $S$ es un clique,
        $S \subseteq M$. Obtenemos por tanto el conjunto $C_0$, que será para lo que utilizemos
        este método.

  \item \textit{missing\_one\_connection}: Dado un conjunto de nodos $S$ y una matriz de adyacencias,
        devuelve el conjunto de nodos conectados con todos los de $S$ salvo uno. Lo usaremos
        para calcular el conjunto $C_1$ asociado a un clique.

  \item \textit{one\_connected\_with\_all}: Recibe una lista y una matriz de adyacencias, y
        devuelve \textit{true} si existe un nodo fuera de la lista conectado a todos los
        nodos de ella.

  \item \textit{value} y \textit{value2}: Funciones objetivo utilizadas en los algoritmos
        de enfriamiento simulado. Ambas reciben un grafo y una matriz de adyacencias, y
        devuelven el valor calculado.

  \item \textit{is\_clique}: Función que recibe un grafo y una matriz de adyacencias, y
        devuelve si dicho grafo es un clique o no.

  \item \textit{print\_solution}: Método para mostrar por pantalla una solución. Primero,
        comprueba si es un clique con el función anterior y lo muestra. Después, imprime
        su longitud. Además, imprime la solución y su longitud por la salida de error,
        lo que permite redirigirlo a un archivo distinto a la hora de ejecutar el programa.

\end{itemize}
