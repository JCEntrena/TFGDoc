%************************************************
\chapter{Resultados y análisis}\label{ch:resultados}
%************************************************

En este capítulo expondremos los resultados obtenidos por cada algoritmo sobre el
conjunto de cliques que se ha considerado, y evaluaremos el funcionamiento individual
de cada uno de ellos. Además, compararemos los algoritmos del mismo tipo para ver cómo
afectan sus diferencias en los resultados.

\section{Consideraciones}

Para la toma de datos, he utilizado un portátil con el sistema operativo Antergos
Linux 64-bit, con un procesador Intel Core i7-3610QM, que cuenta con $4$ núcleos
y $2$ hebras por núcleo, con una frecuencia de reloj de $2.30$ gigahercios. Los
algoritmos han sido ejecutados individualmente, sin uso de paralelismo, por lo que
cada ejecución ha ocupado una sola hebra.

Algunos de los algoritmos implementados en este trabajo cuentan con una o varias
componentes aleatorias: elección de un elemento aleatorio, cálculo de un valor para
aceptar o no una acción que depende de una probabilidad, etcétera. Debido a esto,
para la obtención de resultados, se han realizado varias ejecuciones de estos algoritmos,
de las cuales se ha tomado el mejor valor obtenido. He tomado el mejor, y no la media,
porque representa el tamaño del clique que dicho algoritmo puede alcanzar en un
número razonable de ejecuciones.

En el caso de los algoritmos que no tienen componentes aleatorias, como su ejecución
lleva siempre a la misma solución, el programa se ha ejecutado una única vez en
cada caso. Estos algoritmos son los dos \textit{greedy} y 1LS. Además, los algoritmos
ILS y GRASP que usan 1LS también se han ejecutado una única vez, aunque el método
de obtención de soluciones iniciales tenga componentes aleatorias, debido a que su
tiempo de ejecución es muy alto.

El resto de algoritmos incluyen componentes aleatorias, y se ha considerado un
número variable de ejecuciones para cada uno de ellos, dependiendo del tiempo que
tardasen. Se detallan a continuación:
\begin{itemize}
  \item DLS: 20 iteraciones.
  \item Enfriamiento simulado: 10 iteraciones para cada algoritmo.
  \item Colonia de hormigas: 10 iteraciones para cada algoritmo.
  \item ILS y GRASP con DLS: 10 iteraciones.
  \item Genético: 10 iteraciones.
  \item Memético: 5 iteraciones.
\end{itemize}

\section{Análisis}

A continuación, vamos a ver los resultados obtenidos por cada algoritmo en forma de tabla,
para hacer un análisis de cada uno de ellos. Debido a la cantidad de algoritmos, estos
han sido divididos según su tipo, en grupos de dos: dos \textit{greedy}, dos búsquedas
locales, dos enfriamientos simulados, dos ILS, dos GRASP, dos algoritmos de colonia de
hormigas y los algoritmos genético y memético. Para cada grupo, mostraremos los resultados
y haremos un análisis individual y comparativo de los dos algoritmos que contiene.

Las tablas tienen la siguiente disposición: la primera columna contiene cada uno de los
grafos considerados. La segunda columna indica el tamaño del mejor clique conocido para
cada grafo, notando con un asterisco los valores que están confirmados como óptimos
globales. El resto de columnas contienen el tamaño del mejor clique encontrado para cada
algoritmo, notado como |C|, y el tiempo de ejecución, que se ha tomado el valor medio en
caso de haberse efectuado más de una ejecución. Como medidas para evaluar a los grafos,
he considerado el porcentaje medio de tamaño de los cliques encontrados, calculado como
la media aritmética de los porcentajes en cada grafo, y el número de óptimos encontrados.

\subsection{Algoritmos \textit{greedy}}

\begin{small}
\begin{longtable}{l l l l l l}
  \label{table:greedy}\\
  & & \multicolumn{2}{c}{Greedy básico} & \multicolumn{2}{c}{Greedy adaptativo} \\ \cline{3-6}
  Grafo              & Clique & |C| & Tiempo (s) & |C| & Tiempo (s) \\ \hline
  \endhead
  \endfoot
  brock200\_2        & 12 & 7 & < 0.01 & 9 & 0.18 \\ \hline
  brock200\_4        & 17 & 13 & 0.02 & 15 & 0.27 \\ \hline
  brock400\_2        & 29 & 20 & 0.08 & 22 & 1.61 \\ \hline
  brock400\_4        & 33 & 18 & 0.07 & 21 & 1.58 \\ \hline
  brock800\_2        & 24 & 14 & 0.18 & 19 & 4.98 \\ \hline
  brock800\_4        & 26 & 14 & 0.18 & 19 & 4.41 \\ \hline
  C125.9             & 34* & 29 & 0.02 & 32 & 0.44 \\ \hline
  C250.9             & 44* & 35 & 0.08 & 40 & 1.49 \\ \hline
  C500.9             & 57 & 43 & 0.31 & 51 & 24.10 \\ \hline
  C1000.9            & 68 & 51 & 1.16 & 56 & 126.73 \\ \hline
  C2000.5            & 16* & 10 & 0.72 & 13 & 24.73 \\ \hline
  C2000.9            & 80 & 56 & 4.39 & 66 & 1058.85 \\ \hline
  C4000.5            & 18* & 12 & 2.86 & 15 & 3.64 \\ \hline
  DSJC500\_5         & 13 & 8 & 0.05 & 11 & 0.07 \\ \hline
  DSJC1000\_5        & 15 & 10 & 0.19 & 12 & 6.18 \\ \hline
  gen200\_p0.9\_44   & 44 & 32 & 0.06 & 36 & 1.81 \\ \hline
  gen200\_p0.9\_55   & 55 & 36 & 0.06 & 39 & 1.56 \\ \hline
  gen400\_p0.9\_55   & 55 & 44 & 0.22 & 43 & 12.67 \\ \hline
  gen400\_p0.9\_65   & 65 & 40 & 0.21 & 44 & 12.82 \\ \hline
  gen400\_p0.9\_75   & 75 & 45 & 0.23 & 46 & 13.36 \\ \hline
  hamming10-4        & 40 & 32 & 0.70 & 36 & 120.37 \\ \hline
  hamming8-4         & 16 & 16 & 0.02 & 16 & 0.57 \\ \hline
  keller4            & 11 & 8 & < 0.01 & 11 & 0.25 \\ \hline
  keller5            & 27 & 17 & 0.29 & 24 & 45.48 \\ \hline
  keller6            & 59 & 37 & 9.39 & 48 & 5575.49 \\ \hline
  MANN\_a27          & 126 & 125 & 1.03 & 125 & 17.20 \\ \hline
  p\_hat300-1        & 8 & 7 & 0.01 & 8 & 0.02 \\ \hline
  p\_hat300-2        & 25 & 23 & 0.05 & 23 & 0.83 \\ \hline
  p\_hat300-3        & 36 & 30 & 0.08 & 34 & 2.68 \\ \hline
  p\_hat700-1        & 11 & 7 & 0.007 & 8 & 0.10 \\ \hline
  p\_hat700-2        & 44 & 38 & 0.32 & 43 & 7.51 \\ \hline
  p\_hat700-3        & 62* & 55 & 0.59 & 58 & 10.20 \\ \hline
  p\_hat1500-1       & 12 & 8 & 0.32 & 11 & 0.44 \\ \hline
  p\_hat1500-2       & 65* & 54 & 1.65 & 62 & 42.14 \\ \hline
  p\_hat1500-3       & 94* & 75 & 2.88 & 87 & 94.61 \\ \hline
  \caption{Resultados en algoritmos greedy.}
\end{longtable}
\end{small}

Los resultados obtenidos por los dos algoritmos \textit{greedy} han sido de un óptimo y una media
de $73.2\%$ de tamaño para el \textit{greedy} básico, y tres óptimos y un $84.9\%$ de media
para el \textit{greedy} adaptativo.

Los resultados del algoritmo \textit{greedy} básico muestran que este falla a la hora de
alcanzar el óptimo en el conjunto de instancias, pues lo logra en solo una de las $35$.
El tamaño medio manifiesta que no se obtienen soluciones de calidad, existiendo
un amplio margen de mejora en la mayoría de instancias del problema. En cuanto a
los tiempos, es un algoritmo rápido, que por lo general requiere menos de $1$
segundo para ejecutarse, y que no supera los $10$ segundos en ningún caso.

El \textit{greedy} adaptativo logra mejorar o igualar los resultados obtenidos por el algoritmo
anterior en todos los grafos, lo que confirma que ampliar el entorno con movimientos
\textit{swap} mejora sensiblemente los resultados, aunque aumenta de forma considerable
los tiempos de ejecución, llegando a superar la hora en \textit{keller6}. Aun así,
en la mayoría de los grafos, la ejecución es rápida, no llegando a superar los $60$
segundos. Si nos fijamos en sus resultados por separado, el porcentaje medio de tamaño
de los cliques hallados y la obtención de únicamente $3$ óptimos, hacen que este
algoritmo tampoco consiga buenos resultados si se aplica sin combinar con ningún otro.

Estos resultados muestran que los algoritmos voraces implementados no
son capaces de alcanzar buenas soluciones por sí solos. Aun así, los cliques
obtenidos pueden servir como soluciones de partida para otros algoritmos que sean
capaces de mejorar las carencias que tienen los \textit{greedy}. En este caso,
si trabajamos con cliques de gran tamaño nos interesa un algoritmo más veloz como
es el \textit{greedy} básico. Si por el contrario los grafos son pequeños, a cambio de algo
más de tiempo de ejecución, el algoritmo \textit{greedy} adaptativo nos da soluciones de
partida de más calidad.

\subsection{Búsqueda local}

\begin{small}
  \begin{longtable}{l l l l l l}
  \label{table:bl}\\
    & & \multicolumn{2}{c}{1LS} & \multicolumn{2}{c}{DLS} \\ \cline{3-6}
    Grafo              & Clique & |C| & Tiempo (s) & |C| & Tiempo (s) \\ \hline
    \endhead
    \endfoot
    brock200\_2        & 12 & 9 & 0.36 & 11 & < 0.01   \\ \hline
    brock200\_4        & 17 & 15 & 0.37 & 15 & 0.01   \\ \hline
    brock400\_2        & 29 & 22 & 2.05 & 23 & 0.06   \\ \hline
    brock400\_4        & 33 & 23 & 1.85 & 23 & 0.07   \\ \hline
    brock800\_2        & 24 & 19 & 6.82 & 19 & 0.10   \\ \hline
    brock800\_4        & 26 & 20 & 6.13 & 18 & 0.09   \\ \hline
    C125.9             & 34* & 33 & 0.36 & 34 & 0.04   \\ \hline
    C250.9             & 44* & 42 & 1.91 & 41 & 0.14   \\ \hline
    C500.9             & 57 & 54 & 7.69 & 54 & 0.39   \\ \hline
    C1000.9            & 68 & 63 & 32.02 & 62 & 1.62   \\ \hline
    C2000.5            & 16* & 13 & 46.60 & 14 & 0.19  \\ \hline
    C2000.9            & 80 & 72 & 179.76 & 68 & 3.63  \\ \hline
    C4000.5            & 18* & 15 & 141.92 & 15 & 0.48   \\ \hline
    DSJC500\_5         & 13 & 11 & 1.89 & 12 & 0.03   \\ \hline
    DSJC1000\_5        & 15 & 13 & 8.28 & 13 & 0.05   \\ \hline
    gen200\_p0.9\_44   & 44 & 39 & 1.01 & 43 & 0.08   \\ \hline
    gen200\_p0.9\_55   & 55 & 39 & 1.01 & 55 & 0.12   \\ \hline
    gen400\_p0.9\_55   & 55 & 50 & 5.03 & 49 & 0.35   \\ \hline
    gen400\_p0.9\_65   & 65 & 45 & 9.12 & 49 & 0.37   \\ \hline
    gen400\_p0.9\_75   & 75 & 49 & 5.83 & 75 & 0.37   \\ \hline
    hamming10-4        & 40 & 36 & 19.48 & 40 & 0.68   \\ \hline
    hamming8-4         & 16 & 16 & 0.50 & 16 & < 0.01   \\ \hline
    keller4            & 11 & 11 & 0.22 & 11 & 0.01  \\ \hline
    keller5            & 27 & 23 & 17.27 & 27 & 0.22   \\ \hline
    keller6            & 59 & 51 & 556.38 & 50 & 4.97   \\ \hline
    MANN\_a27          & 126 & 125 & 62.29 & 123 & 3.84   \\ \hline
    p\_hat300-1        & 8 & 7 & 0.55 & 7 & < 0.01   \\ \hline
    p\_hat300-2        & 25 & 25 & 1.35 & 25 & 0.05   \\ \hline
    p\_hat300-3        & 36 & 34 & 1.67 & 36 & 0.13  \\ \hline
    p\_hat700-1        & 11 & 9 & 2.76 & 9 & 0.02   \\ \hline
    p\_hat700-2        & 44 & 44 & 11.17 & 43 & 0.39   \\ \hline
    p\_hat700-3        & 62* & 60 & 28.40 & 61 & 0.76  \\ \hline
    p\_hat1500-1       & 12 & 11 & 17.38 & 11 & 0.08   \\ \hline
    p\_hat1500-2       & 65* & 64 & 132.85 & 63 & 1.85   \\ \hline
    p\_hat1500-3       & 94* & 91 & 128.08 & 91 & 4.61   \\ \hline
  \caption{Resultados en búsqueda local.}
\end{longtable}
\end{small}

Los dos algoritmos de búsqueda local han dado unos resultados de $4$ óptimos y
un $87.5\%$ de media para 1LS, y $9$ óptimos y un $91.0\%$ de media para DLS.

Vemos como 1LS consigue mejorar los resultados obtenidos por ambos algoritmos
\textit{greedy} tanto en número de óptimos como de tamaño medio de los grafos,
lo que reafirma que la exploración del entorno es importante a la hora de buscar
soluciones. No obstante, en vista de los resultados, estos siguen siendo mejorables
en cuanto a la calidad de las soluciones encontradas. Los tiempos de ejecución no
son excesivos, si bien en los grafos más grandes pueden extenderse un poco,
rozando los $10$ minutos en \textit{keller6}. Aun así, en la mayoría de los grafosgreedy
no se superan los $30$ segundos.

DLS logra mejorar los resultados de 1LS, pues es capaz de alcanzar más óptimos
y de obtener un mayor tamaño medio, que en este caso supera el $90\%$, por lo
que tenemos un algoritmo que empieza a obtener soluciones satisfactorias.
Los tiempos de ejecución son significativamente menores, lo cual era de esperar,
ya que DLS, al contrario que ILS, no hace cálculos para la elección de candidatos,
sino que dicha elección es aleatoria, lo que reduce la complejidad del algoritmo.
Todas las ejecuciones se completan en menos de $5$ segundos, la mayoría de ellas
requiriendo menos de $1$ segundo.

Debido a la mejora en tiempos de ejecución y en resultados, podemos afirmar que
DLS funciona mejor que 1LS al aplicarlo directamente sobre los grafos, y cabe esperar
que también ofrezca mejores resultados cuando lo utilicemos en ILS y GRASP. Aun así,
exiten grafos, como \textit{C2000.9} o \textit{MANN\_a27} en los que 1LS proporciona
resultados ligeramente mejores que DLS. Esto puede indicar que en grafos con una
determinada estructura, 1LS pueda ser de más utilidad que DLS.

Recordando el funcionamiento de los algoritmos, 1LS ordenaba los tres movimientos
para obtener el entorno y trataba de aprovechar información del grafo para elegir
entre los nodos candidatos, mientras que DLS no consideraba movimientos \textit{drop},
y las elecciones eran aleatorias. Los resultados parecen apuntar a que el entorno
considerado en DLS da mejores resultados que el tomado en 1LS. Además, el uso de
información inherente al grafo no parece aportar beneficio, si bien sería necesaria
más experimentación para poder afirmarlo.

\subsection{Enfriamiento simulado}

\begin{small}
\begin{longtable}{l l l l l l}
  \label{table:sa}\\
    & & \multicolumn{2}{c}{SA básico} & \multicolumn{2}{c}{SA adaptado} \\ \cline{3-6}
    Grafo              & Clique & |C| & Tiempo (s) & |C| & Tiempo (s) \\ \hline
    \endhead
    \endfoot
    brock200\_2        & 12 & 10 & 1.25 & 9 & 10.00 \\ \hline
    brock200\_4        & 17 & 16 & 2.15 & 13 & 9.67 \\ \hline
    brock400\_2        & 29 & 24 & 7.73 & 18 & 24.15 \\ \hline
    brock400\_4        & 33 & 24 & 5.55 & 19 & 23.30 \\ \hline
    brock800\_2        & 24 & 19 & 11.19 & 16 & 54.90 \\ \hline
    brock800\_4        & 26 & 20 & 11.13 & 16 & 43.35 \\ \hline
    C125.9             & 34* & 34 & 2.06 & 32 & 6.31 \\ \hline
    C250.9             & 44* & 44 & 8.78 & 36 & 14.42 \\ \hline
    C500.9             & 57 & 53 & 22.86 & 45 & 27.79 \\ \hline
    C1000.9            & 68 & 63 & 44.07 & 52 & 57.76 \\ \hline
    C2000.5            & 16* & 15 & 29.26 & 12 & 136.12 \\ \hline
    C2000.9            & 80 & 71 & 83.85 & 57 & 133.82 \\ \hline
    C4000.5            & 18* & 15 & 44.58 & 13 & 220.79 \\ \hline
    DSJC500\_5         & 13 & 12 & 4.10 & 10 & 28.73 \\ \hline
    DSJC1000\_5        & 15 & 14 & 9.28 & 11 & 60.67 \\ \hline
    gen200\_p0.9\_44   & 44 & 44 & 5.29 & 33 & 11.50 \\ \hline
    gen200\_p0.9\_55   & 55 & 55 & 6.18 & 37 & 10.82 \\ \hline
    gen400\_p0.9\_55   & 55 & 51 & 9.95 & 43 & 19.63 \\ \hline
    gen400\_p0.9\_65   & 65 & 65 & 21.35 & 44 & 24.56 \\ \hline
    gen400\_p0.9\_75   & 75 & 75 & 13.42 & 44 & 23.98 \\ \hline
    hamming10-4        & 40 & 40 & 52.06 & 30 & 61.17 \\ \hline
    hamming8-4         & 16 & 16 & 5.23 & 12 & 13.84 \\ \hline
    keller4            & 11 & 11 & 0.78 & 9 & 8.85 \\ \hline
    keller5            & 27 & 25 & 6.38 & 21 & 48.53 \\ \hline
    keller6            & 59 & 49 & 51.53 & 41 & 244.11 \\ \hline
    MANN\_a27          & 126 & 121 & 12.68 & 122 & 17.65 \\ \hline
    p\_hat300-1        & 8 & 8 & 1.22 & 7 & 18.32 \\ \hline
    p\_hat300-2        & 25 & 25 & 4.16 & 21 & 15.20 \\ \hline
    p\_hat300-3        & 36 & 36 & 7.03 & 32 & 16.92 \\ \hline
    p\_hat700-1        & 11 & 10 & 3.65 & 8 & 39.18 \\ \hline
    p\_hat700-2        & 44 & 44 & 22.02 & 37 & 41.78 \\ \hline
    p\_hat700-3        & 62* & 62 & 28.50 & 52 & 41.75 \\ \hline
    p\_hat1500-1       & 12 & 11 & 9.77 & 8 & 89.71 \\ \hline
    p\_hat1500-2       & 65* & 65 & 62.62 & 51 & 94.75 \\ \hline
    p\_hat1500-3       & 94* & 94 & 77.15 & 75 & 91.43 \\ \hline
  \caption{Resultados en enfriamiento simulado.}
\end{longtable}
\end{small}

El algoritmo de enfriamiento simulado básico ha obtenido unos resultados de $16$
óptimos y un $93.5\%$ de media en el tamaño de los cliques, con respecto al máximo
conocido, mientras que la adaptación del algoritmo de Geng et al. ha obtenido
$0$ óptimos y un $75.7\%$ de tamaño medio.

Los resultados en el algoritmo básico mejoran los obtenidos por ambos algoritmos
de búsqueda local, conviertiéndolo en una buena herramienta para resolver el problema,
pues supera ampliamente el $90\%$ de tamaño medio, y alcanza el óptimo casi en la
mitad de instancias. Recordemos que toma un entorno distinto al considerado en la
búsqueda local, sin priorizar ningún movimiento sobre otro. Esto, y el uso de
técnicas de enfriamiento simulado podrían ser los causantes de que el algoritmo
mejore las búsquedas locales. Además, presenta unos tiempos de ejecución bajos,
sin llegar a los $90$ segundos en ningún caso. Esto se debe a que el número de
iteraciones no depende del grafo, sino que viene dado por la temperatura y es
fijo en todos ellos.

Por otra parte, la adaptación del algoritmo de Geng et al. presenta unos resultados
muy pobres, sin ser capaz de encontrar en óptimo en ningún caso. Podemos afirmar que
falla a la hora de resolver el problema de forma satisfactoria. Dado que la exploración
del entorno es aleatoria, como en el caso anterior, cabe plantearse si considerar grafos
cualesquiera en lugar de cliques como espacio de soluciones es la causa de este mal
rendimiento. La función objetivo, con componentes aparte del tamaño de los grafos,
también podría ser causante de que el algoritmo no se dirija a las zonas del espacio
de soluciones de mayor calidad. Además, la función de reducción de grafos a cliques
es otro factor a tener en cuenta, pues no tenemos garantizado que la reducción sea
óptima, y añade otra etapa de posible pérdida de calidad en las soluciones.

Con estos resultados qued comprobado que el algoritmo básico funciona mucho
mejor que la adaptación del algoritmo de Geng et al.

\subsection{ILS}

\begin{small}
\begin{longtable}{l l l l l l}
  \label{table:ils}\\
    & & \multicolumn{2}{c}{ILS - 1LS} & \multicolumn{2}{c}{ILS - DLS} \\ \cline{3-6}
    Grafo              & Clique & |C| & Tiempo (s) & |C| & Tiempo (s) \\ \hline
    \endhead
    \endfoot
    brock200\_2        & 12 & 10 & 4.36 & 10 & 0.12\\ \hline
    brock200\_4        & 17 & 15 & 6.69 & 16 & 0.21\\ \hline
    brock400\_2        & 29 & 24 & 34.90 & 24 & 0.89\\ \hline
    brock400\_4        & 33 & 24 & 35.40 & 24 & 0.90\\ \hline
    brock800\_2        & 24 & 18 & 116.12 & 19 & 1.72\\ \hline
    brock800\_4        & 26 & 18 & 116.78 & 20 & 1.53\\ \hline
    C125.9             & 34* & 34 & 6.91 & 34 & 0.55\\ \hline
    C250.9             & 44* & 44 & 25.94 & 44 & 1.36\\ \hline
    C500.9             & 57 & 57 & 124.56 & 55 & 3.72\\ \hline
    C1000.9            & 68 & 61 & 514.84 & 63 & 10.28\\ \hline
    C2000.5            & 16* & 12 & 547.22 & 14 & 2.59\\ \hline
    C2000.9            & 80 & 72 & 2512.71 & 72 & 29.10\\ \hline
    C4000.5            & 18* & 16 & 2489.99 & 15 & 6.50\\ \hline
    DSJC500\_5         & 13 & 12 & 31.75 & 12 & 0.43\\ \hline
    DSJC1000\_5        & 15 & 14 & 145.22 & 14 & 1.14\\ \hline
    gen200\_p0.9\_44   & 44 & 39 & 14.99 & 40 & 0.98\\ \hline
    gen200\_p0.9\_55   & 55 & 55 & 17.33 & 55 & 0.99\\ \hline
    gen400\_p0.9\_55   & 55 & 51 & 82.47 & 51 & 2.43 \\ \hline
    gen400\_p0.9\_65   & 65 & 49 & 71.32 & 65 & 2.48 \\ \hline
    gen400\_p0.9\_75   & 75 & 75 & 134.14 & 75 & 2.98 \\ \hline
    hamming10-4        & 40 & 38 & 315.82 & 40 & 3.94\\ \hline
    hamming8-4         & 16 & 16 & 8.27 & 16 & 0.13\\ \hline
    keller4            & 11 & 10 & 4.40 & 11 & 0.17 \\ \hline
    keller5            & 27 & 23 & 180.97 & 25 & 2.91 \\ \hline
    keller6            & 59 & 48 & 10034.52 & 54 & 58.89\\ \hline
    MANN\_a27          & 126 & 126 & 1106.07 & 125 & 55.67 \\ \hline
    p\_hat300-1        & 8 & 7 & 8.36 & 8 & 0.12 \\ \hline
    p\_hat300-2        & 25 & 25 & 21.33 & 25 & 0.83\\ \hline
    p\_hat300-3        & 36 & 36 & 26.64 & 36 & 1.24\\ \hline
    p\_hat700-1        & 11 & 9 & 53.69 & 10 & 0.47\\ \hline
    p\_hat700-2        & 44 & 44 & 179.95 & 44 & 4.36\\ \hline
    p\_hat700-3        & 62* & 61 & 282.32 & 62 & 6.67 \\ \hline
    p\_hat1500-1       & 12 & 10 & 255.15 & 10 & 1.24\\ \hline
    p\_hat1500-2       & 65* & 65 & 1417.50 & 65 & 22.65\\ \hline
    p\_hat1500-3       & 94* & 93 & 2167.37 & 94 & 35.91\\ \hline
  \caption{Resutados en ILS.}
\end{longtable}
\end{small}

El algoritmo ILS que utiliza 1LS, ha obtenido unos resultados de $11$ óptimos y
un $90.3\%$ de tamaño medio, mientras que el que utiliza DLS ha obtenido $15$
óptimos y un $93.3\%$ de tamaño medio.

Los resultados en ambos algoritmos mejoran sensiblemente los alcanzados en las
respectivas búsquedas locales, lo que nos dice que el uso de técnicas de reinicialización
es efectivo a la hora de salir de óptimos globales, obteniendo soluciones de partida
que nos permiten explorar de forma más efectiva el espacio de soluciones y alcanzar
mejores cliques.

ILS + 1LS es capaz de obtener buenas soluciones, estando ligeramente por encima del
$90\%$ de tamaño medio en cliques. Sus tiempos de ejecución empiezan a ser notablemente
altos, con valores que llegan hasta casi $3$ horas en \textit{keller6}, debido a
la repetición de 1LS en $20$ ocasiones. Para tratar de que este algoritmo pueda
competir con el resto, habría que experimentar si se pueden mantener los resultados
reduciendo el número de reinicializaciones, para así disminuir los tiempos de ejecución.

ILS + DLS es mejor que la versión con 1LS tanto en resultados como en tiempos de
ejecución, resultado que era de esperar, vistos los obtenidos en la búsqueda local.
En este caso la ejecución no toma más de $60$ segundos en ninguno de los grafos,
mejorando sensiblemente los tiempos del algoritmo anterior.

Concluímos que ambos algoritmos son capaces de obtener soluciones de calidad,
siendo mejor la versión con DLS por sus mejores resultados y menor tiempo de
ejecución, resultado que es consistente con los obtenidos para DLS e ILS.

\subsection{GRASP}

\begin{small}
\begin{longtable}{l l l l l l}
  \label{table:grasp}\\
    & & \multicolumn{2}{c}{GRASP - 1LS} & \multicolumn{2}{c}{GRASP - DLS} \\ \cline{3-6}
    Grafo              & Clique & |C| & Tiempo (s) & |C| & Tiempo (s) \\ \hline
    \endhead
    \endfoot
    brock200\_2        & 12 & 11 & 4.38 & 11 & 0.28\\ \hline
    brock200\_4        & 17 & 16 & 6.79 & 16 & 0.47\\ \hline
    brock400\_2        & 29 & 24 & 33.76 & 23 & 2.21\\ \hline
    brock400\_4        & 33 & 25 & 32.25 & 33 & 2.23\\ \hline
    brock800\_2        & 24 & 19 & 116.77 & 19 & 4.78\\ \hline
    brock800\_4        & 26 & 19 & 101.59 & 20 & 4.60\\ \hline
    C125.9             & 34* & 34 & 5.90 & 34 & 0.99\\ \hline
    C250.9             & 44* & 44 & 25.92 & 43 & 3.38\\ \hline
    C500.9             & 57 & 53 & 114.25 & 53 & 12.17\\ \hline
    C1000.9            & 68 & 63 & 575.84 & 63 & 36.31\\ \hline
    C2000.5            & 16* & 15 & 598.26 & 15 & 16.16\\ \hline
    C2000.9            & 80 & 71 & 2811.73 & 69 & 134.39\\ \hline
    C4000.5            & 18* & 16 & 2561.37 & 16 & 58.44\\ \hline
    DSJC500\_5         & 13 & 13 & 31.00 & 13 & 1.26\\ \hline
    DSJC1000\_5        & 15 & 14 & 150.85 & 14 & 4.44\\ \hline
    gen200\_p0.9\_44   & 44 & 44 & 17.04 & 44 & 2.24\\ \hline
    gen200\_p0.9\_55   & 55 & 55 & 16.21 & 55 & 2.57\\ \hline
    gen400\_p0.9\_55   & 55 & 51 & 78.42 & 49 & 8.42\\ \hline
    gen400\_p0.9\_65   & 65 & 52 & 76.15 & 65 & 8.72\\ \hline
    gen400\_p0.9\_75   & 75 & 75 & 94.38 & 75 & 9.98\\ \hline
    hamming10-4        & 40 & 40 & 340.81 & 40 & 19.86\\ \hline
    hamming8-4         & 16 & 16 & 8.46 & 16 & 0.47\\ \hline
    keller4            & 11 & 11 & 4.47 & 11 & 0.33\\ \hline
    keller5            & 27 & 26 & 200.16 & 27 & 7.50\\ \hline
    keller6            & 59 & 54 & 8958.52 & 53 & 181.52\\ \hline
    MANN\_a27          & 126 & 125 & 1453.48 & 125 & 79.18\\ \hline
    p\_hat300-1        & 8 & 8 & 9.00 & 8 & 0.32\\ \hline
    p\_hat300-2        & 25 & 25 & 21.48 & 25 & 1.46\\ \hline
    p\_hat300-3        & 36 & 36 & 31.68 & 36 & 2.70\\ \hline
    p\_hat700-1        & 11 & 9 & 54.35 & 11 & 1.51\\ \hline
    p\_hat700-2        & 44 & 44 & 215.50 & 44 & 9.21\\ \hline
    p\_hat700-3        & 62* & 62 & 303.31 & 62 & 18.55\\ \hline
    p\_hat1500-1       & 12 & 10 & 266.69 & 11 & 5.94 \\ \hline
    p\_hat1500-2       & 65* & 65 & 1600.58 & 65 & 48.06\\ \hline
    p\_hat1500-3       & 94* & 94 & 2297.67 & 93 & 103.41\\ \hline
  \caption{Resultados en GRASP.}
\end{longtable}
\end{small}

Los resultados obtenidos por ambos algoritmos GRASP han sido de $16$ óptimos y un
$93.5\%$ de tamaño medio para la versión con 1LS, y $18$ óptimos y un $95.3\%$ para
la versión con DLS.

Estos algoritmos son capaces de alcanzar el óptimo en aproximadamente un $50\%$ de
los casos de estudio, reflejando un buen comportamiento general. Además, la versión
que usa DLS como búsqueda local supera el $95\%$ de tamaño medio en los grafos, por
lo que estamos ante un algoritmo capaz de dar soluciones cercanas al óptimo.
El algoritmo con 1LS también ofrece resultados de calidad, incluso mejorando los
resultados obtenidos por el otro GRASP en grafos como \textit{C2000.9} o \textit{p\_hat1500-3}.

Nuevamente, el uso de DLS como búsqueda local proporciona mejores resultados que
si utilizamos la primera búsqueda local, lo que nos reafirma en su mejor comportamiento,
tanto por separado como si se utiliza en técnicas multiarranque. Los tiempos de
ejecución también favorecen a la versión con DLS, sobre todo en grafos con un mayor
número de nodos, donde la mayor complejidad computacional de 1LS afecta notablemente
a dichos tiempos. Ejemplos de esto son grafos como \textit{keller6, C4000.5} y \textit{C2000.9},
en los que la versión con DLS tiene tiempos del orden de los minutos, y la versión
con 1LS tiene tiempos del orden de horas.

Los dos algoritmos proporcionan mejores resultados que sus correspondientes versiones
de búsqueda local iterada, lo que en un principio nos dice que el uso de soluciones
pseudoaleatorias como solución de partida proporciona mejores resultados que el proceso
de mutación utilizado en ILS. No obstante, dado que estos procesos pueden ser modificados,
queda aun margen de mejora en ambos algoritmos, por lo que sería pronto para decir que
GRASP funciona mejor que ILS en el problema del clique máximo. Sería necesaria más
experimentación para poder sacar conclusiones a la hora de comparar ambas técnicas.

\subsection{Algoritmos de colonia de hormigas}

\begin{small}
\begin{longtable}{l l l l l l}
  \label{table:aco}\\
    & & \multicolumn{2}{c}{ACO básico} & \multicolumn{2}{c}{ACO + SA} \\ \cline{3-6}
    Grafo              & Clique & |C| & Tiempo (s) & |C| & Tiempo (s) \\ \hline
    \endhead
    \endfoot
    brock200\_2        & 12 & 12 & 2.24 & 12 & 2.56\\ \hline
    brock200\_4        & 17 & 16 & 4.60 & 16 & 4.96\\ \hline
    brock400\_2        & 29 & 22 & 17.35 & 22 & 18.92\\ \hline
    brock400\_4        & 33 & 23 & 17.45 & 22 & 18.99\\ \hline
    brock800\_2        & 24 & 19 & 25.39 & 19 & 27.56\\ \hline
    brock800\_4        & 26 & 18 & 25.19 & 19 & 27.41\\ \hline
    C125.9             & 34* & 34 & 17.12 & 34 & 14.18\\ \hline
    C250.9             & 44* & 44 & 45.33 & 42 & 38.78\\ \hline
    C500.9             & 57 & 53 & 111.15 & 50 & 104.84\\ \hline
    C1000.9            & 68 & 55 & 264.30 & 54 & 268.65\\ \hline
    C2000.5            & 16* & 14 & 42.72 & 14 & 46.29\\ \hline
    C2000.9            & 80 & 60 & 724.77 & 60 & 748.91\\ \hline
    C4000.5            & 18* & 15 & 98.38 & 16 & 116.37\\ \hline
    DSJC500\_5         & 13 & 13 & 7.32 & 12 & 8.20\\ \hline
    DSJC1000\_5        & 15 & 13 & 17.80 & 13 & 19.62\\ \hline
    gen200\_p0.9\_44   & 44 & 41 & 32.81 & 39 & 27.14\\ \hline
    gen200\_p0.9\_55   & 55 & 55 & 53.89 & 55 & 34.24\\ \hline
    gen400\_p0.9\_55   & 55 & 49 & 89.14 & 46 & 77.33\\ \hline
    gen400\_p0.9\_65   & 65 & 57 & 93.20 & 48 & 79.71\\ \hline
    gen400\_p0.9\_75   & 75 & 75 & 112.08 & 65 & 85.24 \\ \hline
    hamming10-4        & 40 & 33 & 68.76 & 32 & 72.97\\ \hline
    hamming8-4         & 16 & 16 & 3.52 & 16 & 4.03\\ \hline
    keller4            & 11 & 11 & 2.27 & 11 & 2.54\\ \hline
    keller5            & 27 & 24 & 34.34 & 24 & 37.73\\ \hline
    keller6            & 59 & 48 & 618.39 & 45 & 630.96\\ \hline
    MANN\_a27          & 126 & 125 & 665.68 & 124 & 635.35\\ \hline
    p\_hat300-1        & 8 & 8 & 1.88 & 8 & 2.19\\ \hline
    p\_hat300-2        & 25 & 25 & 14.57 & 25 & 13.66\\ \hline
    p\_hat300-3        & 36 & 36 & 32.46 & 35 & 28.17\\ \hline
    p\_hat700-1        & 11 & 9 & 5.52 & 10 & 6.35\\ \hline
    p\_hat700-2        & 44 & 44 & 70.94 & 43 & 64.89\\ \hline
    p\_hat700-3        & 62* & 61 & 182.29 & 61 & 150.63\\ \hline
    p\_hat1500-1       & 12 & 10 & 15.65 & 10 & 28.12 \\ \hline
    p\_hat1500-2       & 65* & 44 & 195.39 & 54 & 203.79\\ \hline
    p\_hat1500-3       & 94* & 87 & 567.45 & 82 & 524.60 \\ \hline
  \caption{Resultados en algoritmos de colonia de hormigas.}
\end{longtable}
\end{small}

Los algoritmos de colonia de hormigas han obtenido unos resultados de $12$ óptimos
y un $89.7\%$ de tamaño medio la versión básica, y $7$ óptimos, con un $88.5\%$
de tamaño medio la versión con enfriamiento simulado.

Ambos algoritmos ofrecen resultados similares en cuanto al tamaño medio de los cliques
encontrados, si bien la versión que añade información del grafo con enfrimiento simulado
es ligeramente peor, encontrando menos óptimos y con un porcentaje menor de tamaño medio.
Estos resultados indican que son necesarios cambios y más experimentación para ver si
añadir dicho  término a la probabilidad es capaz de aportar mejores nodos a la
construcción de soluciones.

Los resultados del algoritmo básico son buenos, rozando el $90\%$ de tamaño medio,
aunque queda por detrás de otras técnicas, como las búsquedas multiarranque o el
enfriamiento simulado básico. Para tratar de mejorar su rendimiento, podríamos
considerar añadir técnicas para construir un algoritmo híbrido, como una busqueda
basada en entornos, aprovechando mejor el comportamiento de las hormigas e intentando
que estas construyan mejores soluciones.

En el algoritmo con enfriamiento simulado, si comparamos con el anterior, se
alcanzan soluciones de menor o igual tamaño en $31$ de los $35$ casos, por lo que
el término añadido al cálculo de la probabilidad no parece haber tenido repercusión
positiva en la obtención de soluciones. Habría que plantearse otra forma de cambiar
la función de probabilidad para que se use la información del grafo de forma positiva.

Ambos algoritmos son similares en cuanto a tiempos, ya que la única diferencia es
el cálculo añadido en la probabilidad en la versión con enfriamiento simulado,
que no parece repercutir de forma significativa. En general, los tiempos no son
muy extensos en todo el conjunto de grafos, aunque alcanzan valores un tanto altos
en las instancias con más nodos, siendo \textit{C2000.9} la que más tiempo requiere,
unos $12$ minutos para cada uno de los dos algoritmos.

\subsection{Algoritmos genéticos}

\begin{small}
\begin{longtable}{l l l l l l}
  \label{table:geneticos}\\
    && \multicolumn{2}{c}{Genético} & \multicolumn{2}{c}{Memético} \\ \cline{3-6}
    Grafo              & Clique & |C| & Tiempo (s) & |C| & Tiempo (s) \\ \hline
    \endhead
    \endfoot
    brock200\_2        & 12 & 12 & 2.10 & 12 & 6.74\\ \hline
    brock200\_4        & 17 & 17 & 4.02 & 17 & 13.11\\ \hline
    brock400\_2        & 29 & 23 & 16.45 & 25 & 50.66\\ \hline
    brock400\_4        & 33 & 25 & 16.44 & 33 & 51.26\\ \hline
    brock800\_2        & 24 & 20 & 24.77 & 20 & 72.04\\ \hline
    brock800\_4        & 26 & 19 & 24.16 & 20 & 77.78\\ \hline
    C125.9             & 34* & 34 & 12.48 & 34 & 42.86\\ \hline
    C250.9             & 44* & 42 & 33.91 & 44 & 143.46\\ \hline
    C500.9             & 57 & 48 & 94.39 & 56 & 284.240\\ \hline
    C1000.9            & 68 & 54 & 261.16 & 65 & 765.52\\ \hline
    C2000.5            & 16* & 15 & 44.22 & 16 & 123.12\\ \hline
    C2000.9            & 80 & 60 & 737.43 & 73 & 2128.81\\ \hline
    C4000.5            & 18* & 12 & 115.13 & 17 & 297.26\\ \hline
    DSJC500\_5         & 13 & 13 & 7.14 & 13 & 20.25\\ \hline
    DSJC1000\_5        & 15 & 13 & 17.62 & 15 & 51.49\\ \hline
    gen200\_p0.9\_44   & 44 & 40 & 24.00 & 44 & 78.15\\ \hline
    gen200\_p0.9\_55   & 55 & 55 & 24.99 & 55 & 54.66\\ \hline
    gen400\_p0.9\_55   & 55 & 46 & 68.00 & 55 & 233.02 \\ \hline
    gen400\_p0.9\_65   & 65 & 47 & 69.90 & 65 & 216.87\\ \hline
    gen400\_p0.9\_75   & 75 & 69 & 77.20 & 75 & 199.47\\ \hline
    hamming10-4        & 40 & 34 & 75.71 & 40 & 205.56\\ \hline
    hamming8-4         & 16 & 16 & 3.58 & 16 & 6.14\\ \hline
    keller4            & 11 & 11 & 1.95 & 11 & 10.73\\ \hline
    keller5            & 27 & 27 & 33.26 & 27 & 192.61\\ \hline
    keller6            & 59 & 47 & 634.94 & 53 & 4265.78\\ \hline
    MANN\_a27          & 126 & 125 & 646.76 & 125 & 1682.67\\ \hline
    p\_hat300-1        & 8 & 8 & 1.92 & 8 & 6.20\\ \hline
    p\_hat300-2        & 25 & 25 & 11.45 & 25 & 50.88\\ \hline
    p\_hat300-3        & 36 & 36 & 25.37 & 36 & 85.84\\ \hline
    p\_hat700-1        & 11 & 11 & 6.05 & 11 & 18.93\\ \hline
    p\_hat700-2        & 44 & 44 & 81.02 & 44 & 270.53\\ \hline
    p\_hat700-3        & 62* & 60 & 149.41 & 62 & 565.92\\ \hline
    p\_hat1500-1       & 12 & 11 & 18.45 & 11 & 56.23\\ \hline
    p\_hat1500-2       & 65* & 61 & 319.06 & 65 & 1472.25\\ \hline
    p\_hat1500-3       & 94* & 86 & 615.29 & 94 & 2964.62\\ \hline
  \caption{Resultados en algoritmos genéticos.}
\end{longtable}
\end{small}

El algoritmo genético ha dado unos resultados de $13$ óptimos encontrados y un
$90.5\%$ de tamaño medio de los cliques, mientras que su versión híbrida con la
búsqueda local DLS ha encontrado $25$ óptimos, con un tamaño medio de un $97.3\%$,
lo que lo convierte en el mejor de los algoritmos implementados en cuanto a resultados.
Queda claro que usar una búsqueda local para mejorar la población del algoritmo
genético ayuda a obtener mejores soluciones.

Por sí solo, el algoritmo genético presenta un buen comportamiento, siendo capaz de
encontrar soluciones aceptables en el conjunto de instancias considerado. Habría
que experimentar con distintos parámetros tanto en probabilidades de cruce y mutación
como en número de iteraciones y tamaño de población para poder realizar un ajuste más
minucioso, que conllevase una mejora de los resultados. También
debería estudiarse la necesidad de reinicializar algunos individuos de la población,
en caso de que estos fueran similares unos a otros, para introducir diversidad.

El algoritmo memético fruto de la combinación del genético y DLS logra mejorar
considerablemente el funcionamiento del genético, proporcionando muy buenos resultados
en el conjunto de instancias. El $97.3\%$ de tamaño medio nos indica que estamos
ante un algoritmo capaz de encontrar soluciones cercanas al óptimo en todas las
instancias, pues si nos fijamos en los grafos por separado, vemos que no hay ninguno
en el que el rendimiento sea especialmente malo.

Dado que la única diferencia que existe entre las dos versiones es el uso de DLS,
se pone de  manifiesto que es capaz de mejorar buenos algoritmos, lo que nos lleva
a considerar su uso para otros algoritmos, como podría ser un ACO, el cual daba
unos resultados similares al genético.

\subsection{Comportamiento general}

El conjunto de algoritmos implementados muestra, en general, un comportamiento dentro de
lo esperado. Hemos visto como los algoritmos \textit{greedy} no son capaces de encontrar
por sí solos buenas soluciones al problema, aunque aportan cliques que pueden ser utilizados
por otros métodos como soluciones iniciales. La exploración del entorno hecha con búsquedas
locales o enfriamiento simulado mejora estos resultados, si bien no todos los algoritmos de
este tipo proporcionan buenas soluciones, como demuestra la adaptación de la idea de Geng
et al. Estas soluciones pueden ser aun mejores si se utilizan técnicas multiarranque,
hecho que respaldan los resultados de los algoritmos ILS y GRASP.

Dado su amplio uso en problemas de grafos, y sus buenos resultados en general, cabía
esperar que los algoritmos de colonia de hormigas funcionasen bien en este problema, y
así ha sido, aunque sus resultados no han destacado especialmente. Por otra parte,
algoritmos menos usuales en problemas de grafos como los genéticos, han mostrado un
comportamiento bueno, en especial el algoritmo memético, claramente el mejor en
cuanto a resultados de todos los considerados. Ajustar en ambos tipos de algoritmos
los parámetros involucrados podría mejorar su comportamiento, sobre todo en el caso
de los ACO, donde los resultados han sido algo peores.

Para realizar una comparativa entre todos los algoritmos, vamos a clasificarlos en
todos los grafos que se han considerado. Tomaremos los tres algoritmos que den mejores
resultados en cuanto a tamaño del clique encontrado, y usaremos el tiempo para decidir
en caso de empate. De esta forma, otorgaremos primer, segundo y trecer puesto para
cada grafo, y, haciendo un recuento para todos los algoritmos, veremos como funcionan
con respecto a los demas.

A continuación se muestra

\begin{small}
\begin{longtable}{l l l l}
  \label{table:posiciones}\\
  \endhead
  \endfoot
    Grafo              & $1^{er}$ puesto & $2$º puesto & $3^{er}$ puesto \\ \hline
    brock200\_2        & Genético & ACO básico & ACO + SA  \\  \hline
    brock200\_4        & Genético & Memético & ILS + DLS  \\  \hline
    brock400\_2        & Memético & ILS + DLS & SA básico  \\  \hline
    brock400\_4        & GRASP + DLS & Memético & Genético  \\  \hline
    brock800\_2        & Genético & Memético & DLS  \\  \hline
    brock800\_4        & ILS + DLS & GRASP + DLS & 1LS  \\  \hline
    C125.9             & DLS & ILS + DLS & GRASP + DLS  \\  \hline
    C250.9             & ILS + DLS & SA básico & GRASP + 1LS  \\  \hline
    C500.9             & ILS + 1LS & Memético & ILS + DLS \\  \hline
    C1000.9            & Memético & ILS + DLS & 1LS  \\  \hline
    C2000.5            & Memético & GRASP + DLS & SA básico  \\  \hline
    C2000.9            & Memético & ILS + DLS & 1LS  \\  \hline
    C4000.5            & Memético & GRASP + DLS & ACO + SA  \\  \hline
    DSJC500\_5         & GRASP + DLS & Genético & ACO básico  \\  \hline
    DSJC1000\_5        & Memético & ILS + DLS & GRASP + DLS  \\  \hline
    gen200\_p0.9\_44   & GRASP + DLS & SA básico & GRASP + 1LS  \\  \hline
    gen200\_p0.9\_55   & DLS & ILS + DLS & GRASP + DLS  \\  \hline
    gen400\_p0.9\_55   & Memético & ILS + DLS & SA básico  \\  \hline
    gen400\_p0.9\_65   & ILS + DLS & GRASP + DLS & SA básico  \\  \hline
    gen400\_p0.9\_75   & DLS & ILS + DLS & GRASP + DLS  \\  \hline
    hamming10-4        & DLS & ILS + DLS & GRASP + DLS  \\  \hline
    hamming8-4         & DLS & Greedy básico & ILS + DLS  \\  \hline
    keller4            & DLS & ILS + DLS & 1LS  \\  \hline
    keller5            & DLS & GRASP + DLS & Genético  \\  \hline
    keller6            & ILS + DLS & GRASP + 1LS & GRASP + DLS  \\  \hline
    MANN\_a27          & ILS + 1LS & Greedy básico & DLS  \\  \hline
    p\_hat300-1        & DLS & Greedy adaptativo & ILS + DLS  \\  \hline
    p\_hat300-2        & DLS & ILS + DLS & 1LS  \\  \hline
    p\_hat300-3        & DLS & ILS + DLS & GRASP + DLS  \\  \hline
    p\_hat700-1        & GRASP + DLS & Genético & Memético  \\  \hline
    p\_hat700-2        & ILS + DLS & GRASP + DLS & 1LS  \\  \hline
    p\_hat700-3        & ILS + DLS & GRASP + DLS & SA básico  \\  \hline
    p\_hat1500-1       & DLS & Greedy adaptativo & GRASP + DLS  \\  \hline
    p\_hat1500-2       & ILS + DLS & GRASP + DLS & SA básico  \\  \hline
    p\_hat1500-3       & ILS + DLS & SA básico & GRASP + 1LS  \\  \hline
  \caption{Clasificación por grafos.}
\end{longtable}
\end{small}

Si hacemos un recuento de las posiciones obtenidas por los grafos, obtenemos los
siguientes resultados:

\begin{small}
\begin{longtable}{l l l l}
  \label{table:recuento}\\
  \endhead
  \endfoot
    Algoritmo              & $1^{er}$ puesto & $2$º puesto & $3^{er}$ puesto \\ \hline
    Greedy básico          &  0  &  2  &  0  \\ \hline
    Greedy adaptativo      &  0  &  2  &  0  \\ \hline
    1LS                    &  0  &  0  &  6  \\ \hline
    DLS                    &  11  &  0  &  2  \\ \hline
    SA básico              &  0  &  3  &  6  \\ \hline
    SA adaptado            &  0  &  0  &  0  \\ \hline
    ILS + 1LS              &  2  &  0  &  0  \\ \hline
    ILS + DLS              &  8  &  12  &  4  \\ \hline
    GRASP + 1LS            &  0  &  1  &  3  \\ \hline
    GRASP + DLS            &  4  &  8  &  8  \\ \hline
    ACO básico             &  0  &  1  &  1  \\ \hline
    ACO + SA               &  0  &  0  &  2  \\ \hline
    Genético               &  3  &  2  &  2  \\ \hline
    Memético               &  7  &  4  &  1  \\ \hline
  \caption{Recuento.}
\end{longtable}
\end{small}

Vemos como los primeros puestos los ocupan los algoritmos DLS, las versiones de ILS y
GRASP que usan DLS, la versión ILS de 1LS y los algoritmos genético y memético, mientras
que el resto de posiciones están más distribuidas entre todos los algoritmos. El único
algoritmo que no logra estar entre los tres mejores en ningún grafo es la adaptación del
enfriamiento simulado de Geng et al., lo que nos reafirma en su mal comportamiento.

En grafos más simples, donde el óptimo era alcanzado por varios algoritmos, los
algoritmos más veloces son los que ocupan los primeros puestos. Así, DLS es capaz
de alcanzar 11 primeros puestos, gracias ser el algoritmo más rapido de entre los
considerados, haciendo que siempre que alcance el óptimo se coloque en cabeza.
Las versiones multiarranque de DLS también logran una gran presencia entre los tres
mejores, dado que en grafos más complejos, son capaces de mejorar el comportamiento
de DLS sin aumentar su tiempo de ejecución tanto como otros algoritmos

Es en los grafos complejos donde algoritmos como el genético o el memético logran
diferenciarse del resto, pues, si bien son más lentos que los mencionados anteriormente,
son capaces de alcanzar mejores soluciones. De esta forma, son capaces de colocarse
entre los tres primeros, y usualmente en primer lugar, sobre todo el algoritmo memético,
cuyos tiempos elevados le han hecho perder en la comparativa con búsquedas locales
en caso de empate.

Por tanto, con esta comparación, podemos decir que las búsquedas locales basadas en
DLS y los algoritmos genéticos son los mejores algoritmos de todos los considerados.
Los primeros destacan por su velocidad, y buena capacidad de resolución de un gran
número de instancias. Los segundos tratan de profundizar más en el espacio de soluciones
a costa de mayores tiempos de ejecución, lo que les permite ganar calidad en los grafos
donde algoritmos más simples no no trabajan tan bien. Otros algoritmos, como el
enfriamiento simulado adaptado, o las búsquedas basadas en 1LS, también dan buenos
resultados en ciertas instancias, dando a entender que son algoritmos competitivos
en ciertos escenarios.


\section{Conclusiones}

Hemos estudiado de forma satisfactoria el funcionamiento de todos los algoritmos implementados,
que pueden verse como un primer acercamiento al problema que nos permite evaluar las
prestaciones de distintas formas de abordarlo. El trabajo ha sido provechoso en cuanto
a resultados obtenidos, y puede utilizarse como base para futuros trabajos en esta materia.

La línea de trabajo a seguir a partir de aquí podría ser una de las siguientes:
\begin{itemize}
  \item Enfatización en un tipo de algoritmos, experimentando con diversos cambios que
        mejoren su funcionamiento. Dependiendo del tipo de algoritmo, podrían considerarse
        estrategias de reinicialización, énfasis en ciertas fases de búsqueda de soluciones,
        o búsqueda de nuevas medidas que clasifiquen mejor a las soluciones obtenidas, entre
        muchas otras. De esta forma, se intentaría obtener un algoritmo más específico para
        resolver el problema, con un mejor funcionamiento.

  \item Ampliación del trabajo realizado. Existen numerosas técnicas que no se han implementado
        en este trabajo, como la búsqueda tabú, la búsqueda en entornos variables, la búsqueda
        dispersa, etc. Analizar su comportamiento nos daría más información sobre el problema,
        aportando nuevas ideas para tratar de mejorar estos u otros algoritmos.
\end{itemize}




% ------------------------------------------------
% Tablas sin tiempos
\iffalse

\begin{table}[H]
\centering
\caption{Resultados}
\begin{tabular}{|l|l|l|l|l|l|l|l|}
\hline
Grafo              & Mejor conocido & Greedy1 & Greedy2 & BL1 & BL2 & SA1 & SA2 \\ \hline
brock200\_2        & 12 & 7 & 9 & 9 & 11 & 10 & 9 \\ \hline
brock200\_4        & 17 & 13 & 15 & 15 & 15 & 16 & 13 \\ \hline
brock400\_2        & 29 & 20 & 22 & 22 & 23 & 24 & 18 \\ \hline
brock400\_4        & 33 & 18 & 21 & 23 & 23 & 24 & 19 \\ \hline
brock800\_2        & 24 & 14 & 19 & 19 & 19 & 19 & 16 \\ \hline
brock800\_4        & 26 & 14 & 19 & 20 & 18 & 20 & 16 \\ \hline
C125.9             & 34* & 29 & 32 & 33 & 34 & 34 & 32 \\ \hline
C250.9             & 44* & 35 & 40 & 42 & 41 & 44 & 36 \\ \hline
C500.9             & 57 & 43 & 51 & 54 & 54 & 53 & 45 \\ \hline
C1000.9            & 68 & 51 & 56 & 63 & 62 & 63 & 52 \\ \hline
C2000.5            & 16* & 10 & 13 & 13 & 14 & 15 & 12 \\ \hline
C2000.9            & 80 & 56 & 66 & 72 & 68 & 71 & 57 \\ \hline
C4000.5            & 18* & 12 & 15 & 15 & 15 & 15 & 13 \\ \hline
DSJC500\_5         & 13 & 8 & 11 & 11 & 12 & 12 & 10 \\ \hline
DSJC1000\_5        & 15 & 10 & 12 & 13 & 13 & 14 & 11 \\ \hline
gen200\_p0.9\_44   & 44 & 32 & 36 & 39 & 43 & 44 & 33 \\ \hline
gen200\_p0.9\_55   & 55 & 36 & 39 & 39 & 55 & 55 & 37 \\ \hline
gen400\_p0.9\_55   & 55 & 44 & 43 & 50 & 49 & 51 & 43 \\ \hline
gen400\_p0.9\_65   & 65 & 40 & 44 & 45 & 49 & 65 & 44 \\ \hline
gen400\_p0.9\_75   & 75 & 45 & 46 & 49 & 75 & 75 & 44 \\ \hline
hamming10-4        & 40 & 32 & 36 & 36 & 40 & 40 & 30 \\ \hline
hamming8-4         & 16 & 16 & 16 & 16 & 16 & 16 & 12 \\ \hline
keller4            & 11 & 8 & 11 & 11 & 11 & 11 & 9 \\ \hline
keller5            & 27 & 17 & 24 & 23 & 27 & 25 & 21 \\ \hline
keller6            & 59 & 37 & 48 & 51 & 50 & 49 & 41 \\ \hline
MANN\_a27          & 126 & 125 & 125 & 125 & 123 & 121 & 122 \\ \hline
p\_hat300-1        & 8 & 7 & 8 & 7 & 7 & 8 & 7 \\ \hline
p\_hat300-2        & 25 & 23 & 23 & 25 & 25 & 25 & 21 \\ \hline
p\_hat300-3        & 36 & 30 & 34 & 34 & 36 & 36 & 32 \\ \hline
p\_hat700-1        & 11 & 7 & 8 & 9 & 9 & 10 & 8 \\ \hline
p\_hat700-2        & 44 & 38 & 43 & 44 & 43 & 44 & 37 \\ \hline
p\_hat700-3        & 62* & 55 & 58 & 60 & 61 & 62 & 52 \\ \hline
p\_hat1500-1       & 12 & 8 & 11 & 11 & 11 & 11 & 8 \\ \hline
p\_hat1500-2       & 65* & 54 & 62 & 64 & 63 & 65 & 51 \\ \hline
p\_hat1500-3       & 94* & 75 & 87 & 91 & 91 & 94 & 75 \\ \hline
Óptimos            & 35  & 1  & 3  & 4  & 9  & 16 & 0  \\ \hline
Aciertos           & 100\% & 75.5\% & 85.0\% & 88.5\% & 92.2\% & 94.9\% & 76.7\% \\ \hline
Media aciertos     & 100\% & 73.2\% & 84.9\% & 87.5\% & 91.0\% & 93.5\% & 75.7\% \\ \hline
\end{tabular}
\end{table}

\begin{table}[H]
\centering
\caption{Resultados}
\begin{tabular}{|l|l|l|l|l|l|}
\hline
Grafo              & Mejor conocido & ILS1 & ILS2 & GRASP1 & GRASP2 \\ \hline
brock200\_2        & 12 & 10 & 10 & 11 & 11 \\ \hline
brock200\_4        & 17 & 15 & 16 & 16 & 16 \\ \hline
brock400\_2        & 29 & 24 & 24 & 24 & 23 \\ \hline
brock400\_4        & 33 & 24 & 24 & 25 & 33 \\ \hline
brock800\_2        & 24 & 18 & 19 & 19 & 19 \\ \hline
brock800\_4        & 26 & 18 & 20 & 19 & 20 \\ \hline
C125.9             & 34* & 34 & 34 & 34 & 34 \\ \hline
C250.9             & 44* & 44 & 44 & 44 & 43 \\ \hline
C500.9             & 57 & 57 & 55 & 53 & 53 \\ \hline
C1000.9            & 68 & 61 & 63 & 63 & 63 \\ \hline
C2000.5            & 16* & 12 & 14 & 15 & 15 \\ \hline
C2000.9            & 80 & 72 & 72 & 71 & 69 \\ \hline
C4000.5            & 18* & 16 & 15 & 16 & 16 \\ \hline
DSJC500\_5         & 13 & 12 & 12 & 13 & 13 \\ \hline
DSJC1000\_5        & 15 & 14 & 14 & 14 & 14 \\ \hline
gen200\_p0.9\_44   & 44 & 39 & 40 & 44 & 44 \\ \hline
gen200\_p0.9\_55   & 55 & 55 & 55 & 55 & 55 \\ \hline
gen400\_p0.9\_55   & 55 & 51 & 51 & 51 & 49 \\ \hline
gen400\_p0.9\_65   & 65 & 49 & 65 & 52 & 65 \\ \hline
gen400\_p0.9\_75   & 75 & 75 & 75 & 75 & 75 \\ \hline
hamming10-4        & 40 & 38 & 40 & 40 & 40 \\ \hline
hamming8-4         & 16 & 16 & 16 & 16 & 16 \\ \hline
keller4            & 11 & 10 & 11 & 11 & 11 \\ \hline
keller5            & 27 & 23 & 25 & 26 & 27 \\ \hline
keller6            & 59 & 48 & 54 & 54 & 53 \\ \hline
MANN\_a27          & 126 & 126 & 125 & 125 & 125 \\ \hline
p\_hat300-1        & 8 & 7 & 8 & 8 & 8 \\ \hline
p\_hat300-2        & 25 & 25 & 25 & 25 & 25 \\ \hline
p\_hat300-3        & 36 & 36 & 36 & 36 & 36 \\ \hline
p\_hat700-1        & 11 & 9 & 10 & 9 & 11 \\ \hline
p\_hat700-2        & 44 & 44 & 44 & 44 & 44 \\ \hline
p\_hat700-3        & 62* & 61 & 62 & 62 & 62 \\ \hline
p\_hat1500-1       & 12 & 10 & 10 & 10 & 11 \\ \hline
p\_hat1500-2       & 65* & 65 & 65 & 65 & 65 \\ \hline
p\_hat1500-3       & 94* & 93 & 94 & 94 & 93 \\ \hline
Óptimos            & 35  & 11 & 15 & 16 & 18 \\ \hline
Aciertos           & 100\% & 92.6\% & 95.1\% & 94.6\% & 95.8\% \\ \hline
Media aciertos     & 100\% & 90.3\% & 93.3\% & 93.5\% & 95.3\% \\ \hline
\end{tabular}
\end{table}

\begin{table}[H]
\centering
\caption{Resultados}
\begin{tabular}{|l|l|l|l|l|l|}
\hline
Grafo              & Mejor conocido & ACO1 & ACO2 & Genético & Memético \\ \hline
brock200\_2        & 12 & 12 & 12 & 12 & 12 \\ \hline
brock200\_4        & 17 & 16 & 16 & 17 & 17 \\ \hline
brock400\_2        & 29 & 22 & 22 & 23 & 25 \\ \hline
brock400\_4        & 33 & 23 & 22 & 25 & 33 \\ \hline
brock800\_2        & 24 & 19 & 19 & 20 & 20 \\ \hline
brock800\_4        & 26 & 18 & 19 & 19 & 20 \\ \hline
C125.9             & 34* & 34 & 34 & 34 & 34 \\ \hline
C250.9             & 44* & 44 & 42 & 42 & 44 \\ \hline
C500.9             & 57 & 53 & 50 & 48 & 56 \\ \hline
C1000.9            & 68 & 55 & 54 & 54 & 65 \\ \hline
C2000.5            & 16* & 14 & 14 & 15 & 16 \\ \hline
C2000.9            & 80 & 60 & 60 & 60 & 73 \\ \hline
C4000.5            & 18* & 15 & 16 & 12 & 17 \\ \hline
DSJC500\_5         & 13 & 13 & 12 & 13 & 13 \\ \hline
DSJC1000\_5        & 15 & 13 & 13 & 13 & 15 \\ \hline
gen200\_p0.9\_44   & 44 & 41 & 39 & 40 & 44 \\ \hline
gen200\_p0.9\_55   & 55 & 55 & 55 & 55 & 55 \\ \hline
gen400\_p0.9\_55   & 55 & 49 & 46 & 46 & 55 \\ \hline
gen400\_p0.9\_65   & 65 & 57 & 48 & 47 & 65 \\ \hline
gen400\_p0.9\_75   & 75 & 75 & 65 & 69 & 75 \\ \hline
hamming10-4        & 40 & 33 & 32 & 34 & 40 \\ \hline
hamming8-4         & 16 & 16 & 16 & 16 & 16 \\ \hline
keller4            & 11 & 11 & 11 & 11 & 11 \\ \hline
keller5            & 27 & 24 & 24 & 27 & 27 \\ \hline
keller6            & 59 & 48 & 45 & 47 & 53 \\ \hline
MANN\_a27          & 126 & 125 & 124 & 125 & 125 \\ \hline
p\_hat300-1        & 8 & 8 & 8 & 8 & 8 \\ \hline
p\_hat300-2        & 25 & 25 & 25 & 25 & 25 \\ \hline
p\_hat300-3        & 36 & 36 & 35 & 36 & 36 \\ \hline
p\_hat700-1        & 11 & 9 & 10 & 11 & 11 \\ \hline
p\_hat700-2        & 44 & 44 & 43 & 44 & 44 \\ \hline
p\_hat700-3        & 62* & 61 & 61 & 60 & 62 \\ \hline
p\_hat1500-1       & 12 & 10 & 10 & 11 & 11 \\ \hline
p\_hat1500-2       & 65* & 44 & 54 & 61 & 65 \\ \hline
p\_hat1500-3       & 94* & 87 & 82 & 86 & 94 \\ \hline
Óptimos            & 35  & 12 & 7 & 13 & 25 \\ \hline
Aciertos           & 100\% & 89.6 \% & 87.4\% & 89.4\% & 97.6 \\ \hline
Media aciertos     & 100\% & 89.7 \% & 88.5\% & 90.5\% & 97.3 \\ \hline
\end{tabular}
\end{table}

\fi
