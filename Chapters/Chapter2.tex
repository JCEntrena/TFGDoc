%************************************************
\chapter{Cálculo de la homología singular}\label{ch:introduccion}
%************************************************

En esta parte, vamos a enunciar y demostrar el teorema de subdivisión baricéntrica,
el cual nos permite calcular directa o indirectamente (a través de sus consecuencias)
el grupo de homología de ciertos espacios topológicos, como haremos con las esferas
n-dimensionales.

\section{Teorema de subdivisión baricéntrica}

\begin{definition}
  Sea $\X$ un espacio topológico. Diremos que $\U \subseteq \mathcal P(\X)$ es un recubrimiento de $\X$ si $\X = \bigcup\limits_{U \in \U} U$.
\end{definition}

Tomemos ahora $\X$ un espacio topológico y $\U$ un recubrimiento suyo. Para $ p \geq 0$, podemos definir
\[ F_p^\U(\X) = \{\phi \in F_p(\X) \mid \exists U \in \U \text{ tal que } \Img \phi \subseteq U\} \]
que es, por definición, subconjunto de $F_p(\X)$.

Si $S_p^\U(\X)$ es el grupo abeliano libre generado por $F_p^\U(\X)$, por $F_p^\U(\X) \subseteq F_p(\X)$, se tiene
que $S_p^\U(\X)$ es un subgrupo de $S_p(\X)$. Si consideramos la inclusión $i \colon S_p^\U(\X) \to S_p(\X)$, si
$\phi \in F_p^\U(\X)$, entonces $\partial_i(\phi) \in F_{p-1}^\U(\X)$ y así, $\partial(\phi) \in S_{p-1}^\U(\X)$. \\
En consecuencia, la restricción del borde a $S_p^\U(\X)$ define un homomorfismo borde, y así, construímos un complejo
de cadenas $S_*^\U(\X) = \{S_p^\U(\X), \partial\}_{p \geq 0}$.

Este complejo es un subcomplejo de $S_*(\X)$. Si tomamos la inclusión $i \colon S_*^\U(\X) \to S_*(\X)$, induce un
homomorfismo en la homología, $i_* \colon H_p(S_*^\U(\X)) \to H_p(\X) \hspace{0.2em} \forall p \geq 0$, donde
$H_p(S_*^\U(\X)) = \frac{\Ker \partial}{\Img \partial}$

Sea ahora $\Y$ otro espacio topológico, y $\V$ un recubrimiento suyo. Diremos que una función continua $f \colon \X \to \Y$ es compatible
con $\U, \V$ si $\forall U \in \U \exists V \in \V$ tal que $f(U) \subseteq V$. En tal caso, puede definirse un homomorfismo
$f_\# \colon S_p^\U(\X) \to S_p^\V(\Y)$ dado por la restricción de $f_\# \colon \SX{p} \to \SY{p} a S_p^\U(\X)$.

Esto hace que el siguiente diagrama sea conmutativo:
\[ \begin{tikzcd}
  S_p^\U(\X) \arrow{r}{f_\#} \arrow{d}{i} & S_p^\V(\Y) \arrow{d}{i} \\
  \SX{p} \arrow{r}{f_\#} & \SY{p}
\end{tikzcd} \]

\begin{remark}
  El diagrama conmutativo está bien definido, pues si $\phi \in F_p^\U(\X)$, entonces existe $U \in \U$ con $\Img \phi \subseteq U$, de donde
  $f_\#(\phi) = f \circ \phi$ verifica $\Img(f \circ \phi) \subseteq f(U) \subseteq V \in \V$.
\end{remark}

Como $f_\# \partial = \partial f_\#$, $f$ induce $f_* \colon H_p(S_p^\U(\X)) \to H_p(S_p^\V(\Y))$.

\begin{theorem}[Teorema de subdivisión baricéntrica]
  Sean $\X$ un espacio topológico y $\U$ un recubrimiento de $\X$ tal que $\interior{\U} = \{\interior{U} \mid U \in \U\}$
  también recubra a $\X$. Entonces la aplicación
  \[ i_* \colon H_p(S_*^\U(\X)) \to H_p(\X) \]
  es un isomorfismo, $\forall p \geq 0$.

  Además, si $(\Y, \V)$ es otro par de las mismas características, y $f \colon \X \to \Y$ es una función continua
  compatible con $\U, \V$, entonces el siguiente diagrama es conmutativo:
  \[ \begin{tikzcd}
    H_p(S_*^\U(\X)) \arrow{r}{i_*} \arrow{d}{f_*} & H_p(\X) \arrow{d}{f_*} \\
    H_p(S_*^\V(\Y)) \arrow{r}{i_*} & H_p(\Y)
  \end{tikzcd} \]

\end{theorem}

Antes de pasar a la demostración del teorema, necesitaremos introducir nuevos conceptos. La idea es definir la subdivisión
de símplices singulares mediante la subdivisión baricéntrica de símplices, que nos llevan a poder dar una especie de
inversa homotópica de la inclusión.

\section{Subdivisión de símplices}

\begin{definition}
  Sea $s_p = (a_0, \dots, a_p)$ un p-símplice. Definimos el baricentro de $s_p$ como
  \[b(s_p) = \frac{a_0 + \dots + a_p}{p+1}\]
\end{definition}

Los puntos $\{b(s_p), a_0, \dots, a_{i-1}, a_{i+1}, \dots, a_p\}$ genera un p-símplice $\forall i = 0, \dots, p$.
Definimos la subdivisión baricéntrica del símplice $s_p$ por inducción sobre $p$, y la notamos $Sd(s_p)$.
\begin{itemize}
  \item $Sd(s_0) = s_0$
  \item $Sd(s_p) = (b(s_p), t_0, \dots, t_{p-1}) \mid (t_0, \dots, t_{p-1}) \in Sd(\U_{p-1}), \U_{p-1} \in \interior{S_p}$
        siendo $\interior{S_p}$ el conjunto de caras de orden $p-1$ de $S_p$.
\end{itemize}

\begin{remark}
  Es importante destacar que el orden en el que se ponen los puntos es importante, pues si el baricentro fuera en otro lugar,
  la orientación del símplice cambiaría, y afectaría al cálculo del borde.
\end{remark}

Si $K$ es un conjunto de símplices, la subdivisión de $K$ vienen dada por $Sd(K) = \bigcup\limits_{s_p \in K} Sd(s_p)$.

Se define la n-ésima subdivisión como \[Sd^n(s_p) = Sd(Sd^{n-1}(s_p)), \text{ con } Sd^1 = Sd \]

\begin{proposition}
  Sea $t_p \in Sd(s_p)$ un p-símplice de la subdivisión. Entonces $diam(t_p) \leq \frac{p}{p+1} diam(s_p)$.
\end{proposition}

\begin{proof}
  Haremos la demostración por inducción sobre $p$.

  Para $p = 0$, se tiene $0 = 0$.

  Supuesto cierto hasta $p-1$, lo probamos para $p$.

  Sea $s_p = (a_0, \dots, a_p)$. Se verifica que $diam(s_p) = \max\limits_{i,j} |a_i - a_j|$.
  Como $t_p \in Sd(s_p), t_p = (b(s_p), t_0, \dots, t_{p-1}) \in Sd(\U_{p-1})$.
  Entonces, $diam(t_p) = max\{|b(s_p) - t_i|,\hspace{0.2em} |t_i - t_j|, i,j = 0, \dots, p-1\}$.

  Por h. de inducción sabemos que $|t_i - t_j| \leq diam(t_0, \dots, t_{p-1}) \leq \frac{p-1}{p} diam(\U_{p-1})$.
  Como $U_{p-1} \subseteq s_p$, entonces $diam(U_{p-1}) \leq diam(s_p)$, y como $\frac{p-1}{p} \leq \frac{p}{p+1}$
  se tiene $|t_i - t_j| \leq \frac{p}{p+1} diam(s_p)$

  Además, $|b(s_p) - t_i| \leq \max\limits_i |b(s_p) - a_i| = \max\limits_i |\frac{a_0 + \dots + a_p}{p-1} - a_i| =
  \max\limits_i | \frac{\sum\limits_{i \neq j} |a_j - a_i|}{p+1} \leq \frac{p}{p+1} diam(s_p)$
\end{proof}
