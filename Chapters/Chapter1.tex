%************************************************
\chapter{Introducción}\label{ch:introduccion}
%************************************************

\section{Definiciones iniciales}

En este capítulo introduciremos la \textbf{homología singular} de un espacio topológico. Comenzaremos con
un conjunto de definiciones iniciales, introduciendo la noción de \textbf{símplice singular}. Con ellos,
construiremos un conjunto al que dotaremos de estructura algebraica, y sobre el que definiremos una aplicación
que, gracias a sus propiedades, nos permitirá definir la homología singular del espacio. Seguidamente,
demostraremos algunos resultados sencillos sobre ciertos espacios topológicos, continuando con la invarianza
homotópica de la homología. Concluiremos el capítulo definiendo la homología sobre pares topológicos, y
viendo como se trasladan los resultados demostrados con anterioridad.

\begin{definition}
Un \textbf{p-símplice} $s_p$ es la envolvente convexa de $p+1$ puntos afínmente independientes en $\mathbb{R}^p$,
a los que llamaremos vértices, $\{x_0, \dots, x_n\}$, esto es
\[ s_p = \{\lambda_0 x_0 + \dots + \lambda_p x_p \mid \lambda_0 + \dots + \lambda_p = 1, \lambda_i \geq 0 \hspace{0.2em} \forall i \}. \]
\end{definition}
% \underbrace{\smash{1}}_{i}
Cuando los vértices son $x_i = (0, \dots, 1, \dots, 0)$, con el $1$ en la posición $i$-ésima, le llamaremos el p-símplice \underline{standard},
y lo notaremos por $\sigma_p$.

Podemos establecer un homeomorfismo $h \colon \sigma_p \to s_p$ por
\[ h(t_0, \dots, t_p) = \sum\limits_{i = 0}^p t_i x_i. \]

\begin{definition}
Sea $\X$ un espacio topológico. Un \textbf{p-símplice singular} en $\mathbb{X}$ es una aplicación continua $\phi \colon \sigma_p \to \X, p \geq 0$.
\end{definition}

Representaremos el conjunto de p-símplices singulares de $\mathbb{X}$ como
\[F_p(\mathbb{X}) = \{\phi \colon \sigma_p \to \mathbb{X} \mid \phi \text{ es p-símplice singular}\}.\]
Es claro que $\mathbb{F}_0(\mathbb{X}) = \mathbb{X}$. Además, $F_1(\mathbb{X})$ son los arcos en $\X$, identificables por
$\mathbb \mathcal{C}([0,1], \X)$.

Si tenemos una aplicación continua entre espacios topológicos $f \colon \X \to \Y$, podemos definir
\[f_{\#} \colon \FX{p} \to \FY{p}  \quad \text{ por }  \quad f_{\#} = f \circ \phi \colon \sigma_p \to Y, \]
que es una aplicación entre los p-símplices singulares de ambos espacios.

Se verifica trivialmente que $I_\# = Id$, puesto que $I_\#(\phi) = I \circ \phi = \phi$.
Además, se tiene que $(g \circ f)_\# = g_\# \circ f_\#$.

\begin{definition}
  Sea $p \geq 1$, $0 \leq i < p$. Definimos $F_p^i$ como la aplicación de un $(p-1)$-símplice a un p-símplice de la siguiente forma:
  \begin{align*}
    F_p^i \colon \sigma_{p-1} &\to \sigma_p\\
    (t_0, \dots, t_{p-1}) &\mapsto (t_0, \dots, t_{i-1}, 0, t_i, \dots, t_{p-1})
  \end{align*}

  Podemos definir una nueva aplicación a partir de esta, a la que llamaremos \textbf{i-esima cara}:
  \begin{align*}
    \partial_i \colon \FX{p} &\to \FX{p-1}\\
    \partial_i(\phi) &= \phi \circ F_p^i
  \end{align*}
\end{definition}

Vamos a probar un resultado sobre la aplicación que acabamos de definir.

\begin{lemma}
  Se verifica que $F_p^i \circ F_{p-1}^j = F_p^j \circ F_{p-1}^{i-1} \quad \forall p \geq 2$, $\forall i, j \colon 0 \leq j < i \leq p$.
\end{lemma}

\begin{proof}
Lo comprobamos realizando los cálculos pertinentes:
  \begin{align*}
    F_p^i \circ F_{p-1}^j (t_0, \dots, t_{p-2}) &= F_p^i (t_0, \dots, t_{j-1}, 0, t_j, \dots, t_{p-2}) \\
    &= (t_0, \dots, t_{j-1}, 0, t_j, \dots, t_{i-2}, 0, t_{i-1}, \dots, t_{p-2}) \\
    &= F_p^j (t_0, \dots, t_{i-2}, 0, t_{i-1}, \dots, t_{p-2}) \\
    &= F_p^j \circ F_{p-1}^{i-1} (t_0, \dots, t_{p-2}).
  \end{align*}
\end{proof}

Se deduce directamente este corolario:

\begin{corollary}
  Para $p \geq 2, 0 \leq j < i \leq p$, se verifica $\partial_j \circ \partial_i = \partial_{i-1} \circ \partial_j$.
\end{corollary}

Debido a que $\FX{p}$ no está dotado de estructura algebraica, es posible realizar la siguiente construcción:

\begin{definition}
  Si $\X$ es un espacio topológico, y $p \geq 0$, se define el \textbf{grupo de p-cadenas singulares} de $\X$
  como el grupo abeliano libre generado por $\FX{p}$
  \[ S_p(\X) = \{\sum_{\phi \in \FX{p}} n_\phi \cdot \phi  \mid n_\phi \in \Z \text{, todos cero salvo un número finito} \}.\]

  Podemos definir un operador \textbf{borde} $\partial$ con las aplicaciones cara $\partial_i$ de la siguiente forma:
  \begin{align*}
    \partial \colon S_p(\X) &\to S_{p-1}(\X)\\
    \partial(\phi) &= \sum_{i = 0}^p (-1)^i \partial_i(\phi)  \hspace{1cm} (p \geq 1)
  \end{align*}
  considerando $\partial_i$ como una extensión de la aplicación cara a un homomorfismo:
  \[\partial_i(\sum_{\phi \in F_p} n_\phi \cdot \phi) = \sum_{\phi \in F_p} n_\phi \cdot \partial_i(\phi) \]
\end{definition}

Por conveniencia, consideraremos $\partial(S_0(\X)) = S_{-1}(\X) = \{0\}$.

Veamos un resultado importante sobre la aplicación borde que acabamos de definir.

\begin{proposition}
  $\partial \circ \partial \colon \SX{n} \to \SX{n-2}$ es el homomorfismo cero.
\end{proposition}

Geométricamente, este resultado nos dice que el borde de una $p$-cadena es una $(p-1)$-cadena sin
borde. Esta propiedad nos permitirá definir los grupos de homología singular.

\begin{proof}
  Consideramos $S_p(\X) \xrightarrow{\partial} S_{p-1}(\X) \xrightarrow{\partial} S_{p-2}(\X)$. \\
  Es claro que si $p = 0, 1$, la cuestión es trivial, pues $S_{-1} = \{0\}$. Tomaremos $p > 1$.

  Sea $\phi \in \FX{p}$. Realizamos las operaciones necesarias:
  \begin{align*}
    \partial^2(\phi) &= \partial(\sum_{i = 0}^p (-1)^i \partial_i(\phi)) = \sum_{i = 0}^p (-1)^i \partial(\partial_i(\phi)) \\
                     &= \sum_{i = 0}^p (-1)^i \sum_{j = 0}^{p-1} (-1)^j \partial_j \partial_i(\phi) \\
                     &= \sum_{0 \leq j < i \leq p} (-1)^{i + j} \partial_j \partial_i(\phi)
                        + \sum_{0 \leq i \leq j \leq p-1} (-1)^{i + j} \partial_j \partial_i(\phi) \\
                     &= \sum_{0 \leq j < i \leq p} (-1)^{i + j} \partial_{i-1} \partial_j(\phi)
                        + \sum_{0 \leq i \leq j \leq p-1} (-1)^{i + j} \partial_j \partial_i(\phi) \\
                     &= \sum_{0 \leq j \leq i-1 \leq p} (-1)^{i + j} \partial_{i-1} \partial_j(\phi)
                        + \sum_{0 \leq i \leq j \leq p-1} (-1)^{i + j} \partial_j \partial_i(\phi) = 0,\\
                     &\text{pues son iguales salvo el signo.}
  \end{align*}
\end{proof}

Hasta ahora, hemos tomado un espacio topológico $\X$ y le hemos asignado un conjunto de p-símplices singulares $\FX{p}$, del que
hemos tomado un $\Z$-módulo, obteniendo el grupo abeliano libre de p-cadenas singulares, $S_p(\X)$. Sobre él, hemos definido un
operador borde $\partial$, que verifica $\partial^2 = 0$. A la familia $\{S_p(\X), \partial\}_{p \geq 0}$ se le llama el complejo de cadenas
singular asociado a $\X$.

Representaremos también por $f_\# \colon S_p(\X) \to S_p(\Y)$ a la composición entre
$f \colon \X \to \Y$, continua, y $\phi \in S_p(\X)$, de forma que:
\[ f_\#(\sum_\phi n_\phi \cdot \phi) = \sum_\phi n_\phi \cdot (f \circ \phi). \]

Vamos a ver que esta aplicación conmuta con el borde, es decir, $f_\# \circ \partial = \partial \circ f_\#$. Esto hace conmutativo el siguiente diagrama:
\[
  \begin{tikzcd}
    \dots S_{p+1}(\X) \arrow{r}{\partial} \arrow{d}{f_\#} & S_p(\X) \arrow{r}{\partial} \arrow{d}{f_\#} & S_{p-1}(\X) \arrow{d}{f_\#} \dots \\
    \dots S_{p+1}(\Y) \arrow{r}{\partial}                 & S_p(\Y) \arrow{r}{\partial}                 & S_{p-1}(\Y) \dots
  \end{tikzcd}
\]

Su demostración es casi inmediata, pues se tiene que
\[ f_\# \circ \partial_i(\phi) = f_\#(\phi \circ F_p^i) = f \circ \phi \circ F_p^i = \partial_i(f \circ \phi) = \partial_i \circ f_\#(\phi). \]

El resultado obtenido implica, por la definición de $\partial$, que $\partial f_\# = f_\# \partial$. De hecho, es más fuerte, pues vemos
como las aplicaciones cara conmutan una a una con el homomorfismo $f_\#$.

A esta aplicación entre complejos de cadenas se le llama en álgebra una aplicación de cadenas. Nos referiremos a ella como la aplicación de cadenas
inducida por la aplicación continua f.

\begin{definition}
  Diremos que $c \in S_p(\X)$ es un \textbf{p-ciclo} si $\partial(c) = 0$. \\
  Diremos que $c \in S_p(\X)$ es un \textbf{p-borde} si existe $d \in S_{p+1}(\X) \text{ tal que } \partial(d) = c$.
\end{definition}

Representaremos por: \\
$Z_p(\X) = \{c \in S_p(\X) \mid \text{ c es un p-ciclo}\},$ \\
$B_p(\X) = \{c \in S_p(\X) \mid \text{ c es un p-borde}\}.$

Se verifica que: \\
$Z_p(\X) = \Ker (\partial \colon S_p(\X) \to S_{p-1}(\X)),$ \\
$B_p(\X) = \Img (\partial \colon S_{p+1}(\X) \to S_p(\X)).$

Ambos son subgrupos de $S_p(\X)$, llamados grupos de los ciclos y los bordes, respectivamente. Como hemos visto que $\partial^2 = 0$, se deduce
inmediatamente que $B_p(\X) \subseteq Z_p(\X) \hspace{0.2em}\forall p \geq 0$, luego se puede construir el grupo cociente.
\begin{equation*}
  H_p(\X) = \ddfrac{Z_p(\X)}{B_p(\X)}
\end{equation*}
al que llamaremos el \textbf{\underline{\smash{p-ésimo grupo de homología singular}}}, el cual es un grupo abeliano.

\begin{remark}
  Nótese que $Z_0(\X) = S_0(\X)$, pues $\partial(S_0(X)) = 0$.
\end{remark}

A los elementos de $H_p(\X)$ los notaremos $[c]$, con $\partial(c) = 0$, esto es, $c \in Z_p(\X)$, y los llamaremos clases de homología.

La motivación geométrica para la construcción de los grupos de homología singular se basa en querer
estudiar los ciclos en espacios topológicos. Como este conjunto tiene un tamaño demasiado grande para
poder estudiarlo de forma efectiva, restringimos nuestra atención a las clases de equivalencia, haciendo
que dos ciclos sean equivalentes si su diferencia es el borde de una cadena en una dimensión superior.

Tomemos nuevamente una función continua $f \colon \X \to \Y$, y definamos
  \[ f_* \colon H_p(\X) \to H_p(\Y) \text{ por } f_*([c]) = [f_\#(c)]. \]

Comprobemos que está bien definida, esto es, que no depende del representante que se elija.

Sean $[c_1] = [c_2]$, lo que implica $c_1 - c_2 \in B_p(\X)$. Por tanto, existe $d \in S_{p+1}(\X)$ tal que $\partial(d) = c_1 - c_2$.
Así, $f_\#(c_1) - f_\#(c_2) = f_\#(c_1 - c_2) = f_\#(\partial(d)) = \partial(f_\#(d))$. Como $f_\#(d) \in S_{p+1}(\Y)$, se tiene $[f_\#(c_1)] = [f_\#(c_2)].$

Además, $\partial(f_\#(c)) = f_\#(\partial(c)) = 0$, por lo que $f_\#(c)$ es un ciclo en $\Y$. Con esto hemos visto que $f_*$ está bien definida,
y es un homomorfismo, al serlo $f_\#$, al que llamaremos homomorfismo inducido por f en la homología.

De las propiedades de $f_\#$ obtenemos directamente que:
\begin{itemize}
  \item $(g \circ f)_* = g_* \circ f_*$,
  \item $I_* = Id$.
\end{itemize}

Así, hemos construído un funtor covariante de la categoría de espacios topológicos en la categoría de grupos abelianos,
el funtor de la homología singular.
\[  \begin{tikzcd}
  \X \arrow{r} \arrow{d}{f} & [5em] H_p(\X) \arrow{d}{f_*} \\
  \Y \arrow{r}  & H_p(\Y)
\end{tikzcd} \]

Si $\X$ e $\Y$ son homeomorfos, el homomorfismo que induce dicho homeomorfismo en la homología es un isomorfismo de grupos de homología, pues
basta considerar el inverso del homeomorfismo para construir el inverso del homomorfismo inducido.

\section{Homología de un punto}

Vamos a calcular los grupos de homología singular de los puntos, sin más que ver como se comporta la función borde
en ellos y utilizando la definición de los grupos de ciclos y bordes como núcleo e imagen de dicha aplicación.

Sea $\X$ un espacio topológico formado por un punto, $\X = \{x\}$.
$\FX{p} = \{ \phi \colon \sigma_p \to \{x\} \} = \{\phi_p\}$ donde $\phi_p$ es la función constante.

$S_p({\{x\}})$ son grupos abelianos libres con $\phi_p$ como generador. Además, $\partial \colon \SX{p} \to \SX{p-1}$ verifica:
\[\partial(\phi_p) = \sum_{i = 0}^p (-1)^i \partial_i(\phi_p) =  \sum_{i = 0}^p (-1)^i \phi_{p-1} = \begin{cases}
                                                                                                            0 & p \text{ es impar,} \\
                                                                                                            \phi_{p-1} & p \text{ es par.}
                                                                                                    \end{cases}   \]

Así, si p es impar, $S_{p+1}(\{x\}) \to S_p(\{x\}) \xrightarrow{0} S_{p-1}(\{x\})$ y $Z_p(\{x\}) = B_p(\{x\}) = S_p(\{x\})$, luego
$H_p(\{x\}) = 0$.

Si p es par, con $p > 0$, $Z_p(\{x\}) = B_p(\{x\}) = 0$, y $H_p(\{x\}) = 0$.

Por último, si $ p = 0, H_0(\{x\}) = \ddfrac{S_0(\{x\})}{B_0(\{x\})} = S_0(\{x\}) \cong \mathbb Z$.

Así, $H_p(\{x\}) = 0 \hspace{1em} \forall p \geq 1, \hspace{1em} H_0(\{x\}) \cong \mathbb Z$.

\section{El grupo $H_0(\X)$ y la arcoconexión}

En esta sección vamos a estudiar algunas de las propiedades de los grupos de homología singular, asumiendo que el espacio
sobre el que trabajamos es arcoconexo.

\begin{proposition}
  Sea $\X$ un espacio topológico arcoconexo. Entonces $H_0(\X) \cong \mathbb Z$.
\end{proposition}

\begin{proof}
  Consideramos $\SX{1} \xrightarrow{\partial} \SX{0} \to \{0\}$. Tomamos $H_0(\X)$.

  $H_0(\X) = \ddfrac{Z_0(\X)}{B_0(\X)} = \ddfrac{S_0(\X)}{\Img \partial}$

  Sea $\epsilon \colon S_0(\X) \to \Z$ definido por \[\epsilon(\sum\limits_{\phi} n_\phi \phi) = \epsilon(\sum\limits_{x} n_x \cdot x) = \sum\limits_{x} n_x, \]
  suma finita, por cómo definimos $S_0$.

  Se verifica que $\epsilon$ es un epimorfismo de grupos, pues podemos definir el conveniente elemento de $S_0(\X)$ para obtener
  cualquer entero. Calculamos el núcleo de la aplicación:
  \[ \epsilon \circ \partial(\phi_1) = \epsilon(\sum_{i=0}^1 \partial_i(\phi_i)) = 1 - 1 = 0.\]
  Así, $\Img \partial \subseteq \Ker \epsilon$.

  Sea ahora $\sum n_x \cdot x \in \Ker \epsilon$, esto es, $\sum n_x = 0$. \\
  $\sum n_x \cdot x = n_1 x_1 + \dots + n_k x_k$ con $\sum\limits_{i=1}^k n_i = 0, x_i \in \X$. \\
  Sean $\phi^i \colon \sigma_1 \to \X$ arcos tales que $\phi^i(e_1) = x_i, \phi^i(e_0) = x$,
  donde $e_1, e_0$ representan los vértices de $\sigma_1$ y $x \in \X$ es un punto fijo.
  Es posible hacer esta construcción gracias a la arcoconexión de $\X$. De esta forma:
  \begin{align*}
    &\sum_i n_i \phi^i \in \SX{1}, \text{ y se tiene} \\
    &\partial(\sum_i n_i \phi^i) = \sum_i n_i \partial(\phi^i) = \sum_i n_i(x_i - x) = (\sum_i n_i x_i) - (\sum_i n_i) x = \sum_i n_i x_i.
  \end{align*}
  De aquí se deduce que $\Ker \epsilon \subseteq \Img \partial$, y por tanto tenemos la igualdad.

  Finalmente como $\Ker \epsilon = \Img \partial$, $H_0(\X) = \ddfrac{S_0(\X)}{\Ker \epsilon} \cong \Z$, al ser $\epsilon$ epimorfismo.
\end{proof}

\begin{proposition}
  Sea $\X$ un espacio topológico, $\X = \bigcup\limits_{\alpha \in A} \X_\alpha$ su descomposición en componentes arcoconexas. Entonces se verifican:
  \begin{itemize}
    \item[a)] $\forall \alpha \in A$, la función $i_{\alpha*} \colon H_p(\X_\alpha) \to H_p(\X)$ es un monomorfismo, donde la aplicación
              $i_\alpha \colon \X_\alpha \to \X$ representa la inclusión.
    \item[b)] $H_p(\X) = \bigoplus\limits_{\alpha \in A} i_{\alpha*}(H_p(\X_\alpha)) \hspace{0.5em} \forall p \geq 0$. En particular, por la inyectividad de $i_{\alpha*}$,
              \[ H_p(\X) \cong \bigoplus_{\alpha \in A} H_p(\X_\alpha). \]
  \end{itemize}
\end{proposition}

\begin{proof}
  Sea $\phi \colon \sigma_p \to \X$ un p-símplice singular. Como $\sigma_p$ es convexo y $\phi$ es continua, $\phi(\sigma_p) \subseteq \X_\alpha$, para algún
  $\alpha \in A$, por lo que $\phi = i_{\alpha*}(\phi)$ para dicho $\alpha$. \\
  Si $c = \sum n_\phi \cdot \phi$ es una p-cadena, poniendo cada $\phi$ de la forma anterior y agrupando por componentes conexas:
  \begin{equation}
    c = \sum\limits_{\alpha \in A} i_{\alpha*}(c_\alpha), \quad c_\alpha \in S_p(\X_\alpha).  \tag{*}
  \end{equation}

  Sea $[c]_\alpha \in H_p(\X_\alpha)$, y supongamos que $i_{\alpha*}[c]_\alpha = 0$ en $H_p(\X)$. En ese caso, $[i_{\alpha*}(c)] = 0$ en $H_p(\X)$,
  luego existe $d \in \SX{p+1}$ tal que $i_{\alpha\#}(c) = \partial(d)$, y además:
  \[\partial(d) = \sum\limits_{\beta \in A} i_{\beta\#}(\partial(d_\beta)) = i_{\alpha\#}(c), \]
  por lo que se tiene $c = \partial(d_\alpha), \partial(d_\beta) = 0 \hspace{0.5em} \forall \beta \neq \alpha$. Así, $[c]_\alpha = 0$ y $i_{\alpha*}$ es un monomorfismo.
  Esto prueba a).

  Observamos que la suma en (*) es finita, pues si no, habría un número no finito de generadores en la expresión de c. Así:
  \[  [c] = \sum\limits_{\alpha \in A} [i_{\alpha\#}(c_\alpha)] =  \sum\limits_{\alpha \in A} i_{\alpha*}([c_\alpha]_\alpha).\]

  Como $\partial(c) = 0 \iff \partial(c_\alpha) = 0 \hspace{0.5em} \forall \alpha \in A$, se obtiene directamente b), como consecuencia de a).
\end{proof}

En particular, para $H_0(\X)$, por el resultado anterior, se tiene que $H_0(\X)$ es el grupo abeliano libre con tantos generadores como
componentes arcoconexas tenga $\X$.

\section{Invarianza homotópica de la homología singular}

En esta sección demostraremos la invarianza de los grupos de homología singular en espacios homotópicamente equivalentes, viendo antes como
dos aplicaciones homotópicas inducen el mismo homomorfismo en la homología.

Vamos a comenzar demostrando el llamado Lema de Poincaré.

\begin{lemma}[Lema de Poincaré]
  Sea $\X \subseteq \R^n$ estrellado desde algún punto. Entonces:
  \[ H_p(\X) = \begin{cases}  \Z & \text{ si } p = 0, \\
                              0  & \text{ si } p \neq 0.
                            \end{cases} \]
\end{lemma}

\begin{proof}
  El caso $p = 0$ está claro, pues un conjunto estrellado es, en particular, arcoconexo, por lo que solo hemos de aplicar el resultado ya demostrado.
  Suponemos $p > 0$.

  Para demostrarlo, vamos a construir un homomorfismo $ T: \SX{p} \to \SX{p+1}$ tal que $ T \partial + \partial  T = Id_{\SX{p}}$ para $p > 0$.

  De darse la situación anterior, si $[c] \in H_p(\X), c \in Z_p(\X)$, se tiene $\partial(c) = 0$ y $c \in S_p(\X)$.
  En consecuencia, $c = (\partial  T +  T \partial)(c)  = \partial( T(c))$ con $ T(c) \in \SX{p+1}$.
  Por tanto, $c$ es un borde, luego $[c] = 0$, y eso implica que $H_p(\X) = 0 \quad \forall p > 0$.

  Definamos $T$: \\
  Sean $x_0 \in \X$ un punto de estrella, $\phi \colon \sigma_p \to \X$ un p-símplice singular. Tomemos $T(\phi) \colon \sigma_{p+1} \to \X$ dado por:
  \[ T(\phi)(t_0,\dots,t_{p+1}) = \begin{cases} x_0 & \text{ si } t_0 = 1, \\
                                                (1-t_0)\phi(\frac{t_1}{1-t_0},\dots,\frac{t_{p+1}}{1-t_0}) + t_0 x_0 & \text{ si } t_0 < 1.
                                  \end{cases} \]
  Como $\X$ es estrellado, $\Img(T(\phi)) \subseteq \X$.

  Veamos que $T(\phi)$ es continuo. \\
  La continuidad está clara para $t_0 < 1$, por lo que la vemos para $t_0 = 1$. Se verifica que:
  \begin{align*}
    &|T(\phi)(t_0,\dots,t_{p+1}) - x_0| = |t_0 x_0 + (1-t_0)\phi(\frac{t_1}{1-t_0},\dots,\frac{t_{p+1}}{1-t_0}) - x_0| \\
    &= (1-t_0)|\phi(\frac{t_1}{1-t_0},\dots,\frac{t_{p+1}}{1-t_0}) - x_0| \leq (1-t_0)(c + |x_0|), \\
    & \text{ pues $\phi$ está acotada.}
  \end{align*}

  Así, $\lim\limits_{t_0 \to 1} T(\phi) = x_0$.

  Queda claro que $\partial_0(T(\phi)) = \phi$. Si seguimos llamando $T$ al homomorfismo extensión de $T$ a $\SX{p}$, tenemos construído
  \[  T \colon \SX{p} \to \SX{p+1} \hspace{0.5em} \text{tal que } \partial_0 \circ  T = I. \]
  Si $p = 0 \implies i = 1$ y $\partial_1( T(\phi)) = x_0$.

  Si $1 \leq i \leq p+1$, entonces:
  \begin{align*}
    &\partial_i( T(\phi))=(t_0,\dots,t_p)) =  T(\phi)(t_0, \dots, t_{i-1} \circ t_i, \dots, t_p) \\
    &= (1-t_0)\phi(\frac{t_1}{1-t_0}, \dots, \frac{t_{i-1}}{1-t_0}, \dots, \frac{t_p}{1-t_0}) + t_0 x_0 \\
    &= (1-t_0)\partial_{i-1}(\frac{t_1}{1-t_0}, \dots, \frac{t_p}{1-t_0}) + t_0 x_0 = T(\partial_{i-1} \phi)(t_0, \dots, t_p).
  \end{align*}
  Como consecuencia:
  \begin{align*}
    \partial( T(\phi)) &= \sum\limits_{i = 1}^{p+1} (-1)^i \partial_i  T(\phi) = \phi + \sum\limits_{i = 1}^{p+1} (-1)^i  T \partial_{i-1}(\phi) \\
    &= \phi - \sum\limits_{j = 0}^p (-1)^j  T \partial_j(\phi) = \phi -  T \partial(\phi).
  \end{align*}

  De donde se obtiene directamente que $\partial  T +  T \partial = Id$.
\end{proof}

\begin{theorem}[Invarianza homotópica de la homología singular]
  Sean $\X, \Y$ espacios topológicos y $f, g \colon \X \to \Y$ aplicaciones continuas. Si $f$ es homotópica a $g$,
  entonces $f_* = g_*$, siendo $f_*, g_* \colon H_p(\X) \to H_p(\Y) \hspace{0.5em} \forall p \geq 0$.
\end{theorem}

\begin{proof}
  Supongamos que $\forall p \geq 0$ podemos definir homomorfismos \[T: \SX{p} \to \SY{p+1} \] tales que $T\partial + \partial T = f_\# - g_\#$,
  esto es, $f$ y $g$ son homotópicas como aplicaciones de cadenas. Entonces, $\forall [c] \in H_p(\X)$ se tiene:
  \[ (f_* - g_*)([c]) = [f_\#(c) - g_\#(c)] = [(T\partial + \partial T)(c)] = [\partial(T(c))] = 0,\]
  de donde se deduce $f_* = g_*$.

  Veamos la definición de dicho $T$.

  Sea $H \colon \X \times I \to \Y$ la homotopía entre f y g, esto es,
  $H(x, 0) = f(x), H(x, 1) = g(x)$, donde $I$ representa el intervalo $[0, 1]$.

  Sean $i_0, i_1 \colon \X \to \X \times I$ dados por $i_0(x) = (x, 0), i_1(x) = (x, 1)$. Es claro que $f = H \circ i_0, g = H \circ i_1$, de donde
  se tiene \[ f_\# - g_\# = H_\# \circ i_{0\#} - H_\# \circ i_{1\#} = H_\# \circ (i_{0\#} - i_{1\#}). \]

  Si somos capaces de construir $\tau \colon \SX{p} \to S_{p+1}(\X \times I)$ tal que $\partial \tau + \tau \partial = i_{0\#} - i_{1\#}$, definiendo
  $T = H_\# \circ \tau$, se verifica:
  \[ \partial \tau + \tau \partial = \partial(H_\# \circ \tau) + (H_\# \circ \tau)\partial = H_\#(i_{0\#} - i_{1\#}) = f_\# - g_\#,  \]
  lo que demostraría el teorema.

  Construiremos $\tau$ por inducción. Para ello, vamos a probar el siguiente resultado:
  \begin{lemma}
    Para todo $p \geq 0$ y para todo espacio topológico $\X$, existe un homomorfismo de grupos $T^\X \colon \SX{p} \to S_{p+1}(\X \times I)$ verificando:
    \begin{itemize}
      \item[a)] $\partial T^\X + T^\X \partial = (i_0^\X)_\# - (i_1^\X)_\#,$
      \item[b)] (Condición de naturalidad) Para todo espacio topológico $\Y$ y toda aplicación continua $h \colon \X \to \Y$, el diagrama
      \[
      \begin{tikzcd}
        \SX{p} \arrow{r}{T^\X} \arrow{d}{h_\#} & [2em] S_{p+1}(\X \times I) \arrow{d}{(h \times Id)_\#} \\
        \SY{p} \arrow{r}{T^\Y}                 & S_{p+1}(\X \times I)
      \end{tikzcd}
      \quad \text{es conmutativo.}
      \]
    \end{itemize}
    Nótese que en b) pedimos más de lo requerido, pero nos sirve para definir $T^\X$.
  \end{lemma}

  Demostrémoslo para $p = 0$.

  Tomamos $\X = \sigma_0$ e intentamos definir $T^{\sigma_0} \colon S_0(\sigma_0) \to S_1(\sigma_0 \times I)$. \\
  Si $\tau_0 \colon \sigma_0 \to \sigma_0$ es la identidad, entonces $\tau_0$ genera $S_0(\sigma_0)$. Como $T^{\sigma_0}$
  debe verificar $\partial T^\sigma_0 + T^\sigma_0 \partial = (i_0^{\sigma_0})_\# - (i_1^{\sigma_0})_\#$, se deduce
  $\partial T^{\sigma_0}(\sigma_0) = (i_0^{\sigma_0})_\# (\tau_0) - (i_1^{\sigma_0})_\# (\tau_0)$.

  Definiendo:
  \begin{align*}
    T^{\sigma_0}(\tau_0) \colon \sigma_1 &\to \sigma_0 \times I \\
                       (t, 1-t) &\mapsto (\sigma_0, 1-t)
  \end{align*}
  es claro que verifica lo anterior.

  Sea ahora $\X$ espacio topológico, y sea $\phi \in S_0(\X)$ un generador, $\phi \colon \sigma_0 \to \X$.
  Entonces, como la construcción debe ser natural, el diagrama:
  \[
  \begin{tikzcd}
    S_0(\sigma_0) \arrow{r}{T^{\sigma_0}} \arrow{d}{\phi_\#} & [2em] S_1(\sigma_0 \times I) \arrow{d}{(\phi \times Id)_\#} \\
    \SX{0} \arrow{r}{T^\X}                 & S_1(\X \times I)
  \end{tikzcd}
  \]
  debe ser conmutativo, luego $T^\X \circ \phi_\# = (\phi \times Id)_\# \circ T^{\sigma_0}$.

  Como $\phi_\#(\tau_0) = \phi$, se tiene $T^\X(\phi) = (\phi \times Id)_\# \circ T^{\sigma_0}(\tau_0)$.

  Definimos $T^\X$ de esta manera sobre los generadores, y lo extendemos por linealidad. Veamos que verifica a) y b).

  a)
  \begin{align*}
    \partial T^\X(\phi) &= \partial(\phi \times Id)_\# T^{\sigma_0}(\tau_0) = (\phi \times Id)_\# \partial T{ \sigma_0}(\tau_0) \\
                        &= (\phi \times Id)_\# ((i_0)_\#^{\sigma_0}(\tau_0) - (i_1)_\#^{\sigma_0}(\tau_0)) \\
                        &= ((\phi \times Id) (i_0)_\#^{\sigma_0}) - (\phi \times Id) (i_1)_\#^{\sigma_0}))(\tau_0) \\
                        &= ((i_0^\X \circ \phi)_\# - (i_1^\X \circ \phi)_\#) (\tau_0) = ((i_0^\X)_\# - (i_1^\X)_\#)(\phi),
  \end{align*}
  de donde se obtiene $\partial T^\X + T^\X \partial = (i_0^\X)_\# - (i_1^\X)_\#$.

  b) Si $h \colon \X \to \Y$ es continua, entonces:
  \begin{align*}
    (T^\Y \circ h_\#)(\phi) &= T^\Y(h \circ \phi) = (h\phi \times Id)_\# T^{\sigma_0}(\tau_0) \\
    &= ((h \times Id)_\# \circ (\phi \times Id)_\#) (T^{\sigma_0}(\tau_0)) = ((h \times Id)_\# \circ T^\X)(\phi),
  \end{align*}
  lo que finaliza la demostración para $p = 0$.

  Suponemos el resultado cierto para $1, \dots, p-1$ y lo probamos para $p$. El razonamiento es similar al anterior.

  Tomemos $\X = \sigma_p$, y vamos a tratar de definir $T^{\sigma_p} \colon S_p(\sigma_p) \to S_{p+1}(\sigma_p \times I)$.
  De nuevo, como $T^{\sigma_p}$ debe verificar a), se tiene que $\forall \phi \in F_p(\sigma_p), T^{\sigma_p}(\phi)$ ha de satisfacer
  \[ \partial(T^{\sigma_p}(\phi)) = (i_0^{\sigma_p})_\# (\phi) - (i_1^{\sigma_p})_\# (\phi) - T^{\sigma_p}(\partial(\phi)), \]
  donde $T^{\sigma_p}(\partial(\phi))$ está bien definida por hipótesis de inducción.

  Sea $d = (i_0^{\sigma_p})_\# (\phi) - (i_1^{\sigma_p})_\# (\phi) - T^{\sigma_p}(\partial(\phi)) \in S_p(\sigma_p \times I)$.
  Entonces:
  \begin{align*}
    &\partial(\phi) = (i_0^{\sigma_p})_\# (\partial(\phi)) - (i_1^{\sigma_p})_\# (\partial(\phi)) - \partial T^{\sigma_p}(\partial(\phi)) \\
    &\stackrel{\text{h.i.}}{=} (i_0^{\sigma_p})_\#(\partial(\phi)) - (i_1^{\sigma_p})_\#(\partial(\phi)) - [ - T^{\sigma_p}(\partial^2(\phi)) +
    (i_0^{\sigma_p})_\#(\partial(\phi)) - (i_1^{\sigma_p})_\#(\partial(\phi)) ] = 0
  \end{align*}
  y por tanto d es un ciclo.

  Como $p \geq 1$ y $\sigma_p \times I$ es convexo, por el lema de Poincaré, $d$ es un borde, luego existe $e \in S_{p+1}(\sigma_p \times I)$
  tal que $\partial(e) = d$. Definimos $T^{\sigma_p}(\phi) = e$. Notemos que si bien $e$ no tiene por qué ser único, basta con hacer una elección
  entre los existentes.

  Extendiendo por linealidad, conseguimos construir $T^{\sigma_p}$ verificando a).

  Sean ahora $\X, \phi \colon \sigma_p \to \X$ u p-símplice singular. De nuevo, al tener que verificar b), el diagrama
  \[
  \begin{tikzcd}
    S_p(\sigma_p) \arrow{r}{T^{\sigma_p}} \arrow{d}{\phi_\#} & [2em] S_{p+1}(\sigma_p \times I) \arrow{d}{(\phi \times Id)_\#} \\
    \SX{p} \arrow{r}{T^\X}                 & S_{p+1}(\X \times I)
  \end{tikzcd}
  \]
  ha de ser conmutativo.

  Si $\tau_p \colon \sigma_p \to \sigma_p$ es la identidad (que es p-símplice singular en $\sigma_p$), $\phi_\#(\tau_p) = \phi$,
  de donde definimos $T^\X(\phi) = (\phi \times Id)_\# T^{\sigma_p}(\tau_p)$. \\
  Como los $\phi$ son la base de $\SX{p}, T^\X$ se extiende por linealidad a un homomorfismo. Veamos que verifica a) y b).

  a)
  \begin{align*}
    &(\partial T^\X + T^\X \partial)(\phi) = \partial((\phi \times Id)_\# T^{\sigma_p}) + T^\X \partial(\phi_\# \tau_p) \\
    &= (\phi \times Id)_\# \partial T^\sigma_p(\tau_p) + T^\X \phi_\#(\partial(\tau_p)) \\
    &= (\phi \times Id)_\# (- T^{\sigma_p} \partial(\tau_p) + (i_0^{\sigma_p})_\# (\tau_p) - (i_1^{\sigma_p})_\# (\tau_p)) + (\phi \times I)_\# T^{\sigma_p}\partial(\tau_p) \\
    &= (\phi \times Id)_\# (i_0^{\sigma_p})_\#(\tau_p) - (\phi \times Id)_\# (i_1^{\sigma_p})_\#(\tau_p) \\
    &= (i_0^\X)_\# \phi_\#(\tau_p) - (i_1^\X)_\# \phi_\#(\tau_p) = ((i_0^\X)_\# - (i_1^\X)_\#) (\phi).
  \end{align*}

  b) Sea $h\colon \X \to \Y$ continua y $\phi \colon \sigma_p \to \X$ un generador de $\SX{p}$. Entonces:
  \begin{align*}
    T^\Y(h_\#(\phi)) &= T^\Y(h \circ \phi) = ((h \circ \phi) \times Id)_\# T^{\sigma_p}(\tau_p) \\
    &= (h \times Id)_\# (\phi \times Id)_\# T^{\sigma_p} (\tau_p) = (h \times Id)_\# T^\X(\phi).
  \end{align*}
\end{proof}

\begin{corollary}
  Si $\X$ e $\Y$ son espacios homotópicamente equivalentes, entonces \[H_p(\X) = H_p(\Y) \quad \forall p \geq 0.\]
\end{corollary}

\begin{proof}
  Sea $f$ la equivalencia homotópica de $\X$ en $\Y$. Entonces existe su inversa, $g$, que verifica $g \circ f = Id_\X, f \circ g = Id_\Y$.
  Así, $(f \circ g)_* = Id_{H_p(\Y)},$ y $(g \circ f)_* = Id_{H_p(\X)}$. Como $(g \circ f)_* = g_* \circ f_*, f_*$ tiene inverso, y por tanto
  $f_*$ es un isomorfismo entre $H_p(\X)$ y $H_p(\Y)$.
\end{proof}

\section{Homología singular de un par}

Sea $A \subseteq \X$. Vamos a tratar de cuantificar la relación existente entre $H_*(A)$ y $H_*(\X)$.

Para ello, vamos a definir la homología singular del par $(\X, A)$, que será una extensión de la homología singular de un espacio, en el sentido
$H_*(\X, \emptyset) = H_*(\X)$.

Consideremos $i \colon A \xhookrightarrow{} \X$ la inclusión, $i_\# \colon S_p(A) \to \SX{p}$ el homomorfismo inducido.
Entonces $i\#(\sum\limits_{j = 1}^n n_j \phi_j) = \sum\limits_{j = 1}^n n_j i_\#(\phi_j) = \sum\limits_{j = 1}^n n_j (i \circ \phi_j) = \sum\limits_{j = 1}^n n_j \phi_j$,
luego $i_\#$ es un monomorfismo.

Definimos $S_p(\X, A) = \frac{\SX{p}}{i_\#(S_p(A))} = \frac{\SX{p}}{S_p(A)} = \{\overline{c} \mid c \in \SX{p} \}$ con $\overline{c} = c + S_p(A)$.
Este grupo se corresponde con el grupo abeliano libre generado por $\{\overline{\phi} \mid \phi \in \FX{p}\}$, ya que
$ \overline{c} = \overline{\sum_i n_i \phi_i} = \sum_i n_i \overline{\phi_i}, \quad c \in \SX{p}$.

Llamamos a $S_p(\X, A)$ el grupo de p-cadenas singulares del par $(\X, A)$. \\
Podemos definir un operador borde, que seguiremos notando $\partial$:
\begin{align*}
  \partial \colon S_p(\X, A) &\to S_{p-1}(\X, A) \\
  \partial \overline{c} &= \overline{\partial c}.
\end{align*}
Este borde está bien definido, pues si $\overline{c} = \overline{d}$; entonces $d = c + a, a \in \S_p(A)$ y
$\partial d = \partial c + \partial a \implies \overline{\partial d} = \overline{\partial c} + \overline{\partial a} = \overline{\partial c}$ ya que
$\partial a \in S_{p-1}(A) \implies \overline{\partial a} = 0$.

Es claro, por la definición que hemos hecho, que el operador borde es un homomorfismo de clases y que $\partial^2 = 0$.

Con esto, hemos construído un complejo de cadenas $\{S_p(\X, A), \partial\}_{p \geq 0}$ al que llamaremos el complejo de cadenas singulares del par $(\X, A)$.

Para cada $p \geq 0$ definimos: \\
$Z_p(\X, A) = \{\overline{c} \in S_p(\X, A) \mid \partial\overline{c} = 0\}$ p-ciclos, \\
$B_p(\X, A) = \{\overline{c} \in S_p(\X, A) \mid \overline{c} = \partial\overline{d} \}$ p-bordes.

Así, dado que de nuevo $B_p(\X, A) \subseteq Z_p(\X, A)$, podemos considerar el cociente
\[H_p(\X, A) = \frac{Z_p(\X, A)}{B_p(\X, A)} \quad \forall p \geq 0. \]
que llamamos el \textbf{\underline{\smash{p-ésimo grupo de homología singular}}} del par $(\X, A)$.

Dados dos pares $(\X, A)$ y $(\Y, B)$, una aplicación continua $f \colon (\X, A) \to (\Y, B)$ es una aplicación continua
$f \colon \X \to \Y$ tal que $f(A) \subseteq B$. Llamaremos a $f$ una aplicación de pares entre $(\X, A)$ e $(\Y, B)$.

Si $f$ es tal aplicación, definimos:
\begin{align*}
  f_\# \colon S_p(\X, A) &\to S_p(\Y, B) \hspace{4em}\\
  f_\#(\overline{c}) &= \overline{f_\#(c)},
\end{align*}
que está bien definida.

Con esto, hemos hecho conmutativo el diagrama:
\[
\begin{tikzcd}
    S_p(\X) \arrow{r}{f_\#} \arrow{d}{\pi} & S_p(\Y) \arrow{d}{\pi} \\
    S_p(\X, A) \arrow{r}{f_\#} & S_p(\Y, B)
  \end{tikzcd}
  \hspace{2em} \text{con } \pi(c) = \overline{c}.
\]
Es claro que $\partial f_\#(\overline{c}) = \partial \overline{f_\#(c)} = \overline{\partial f_\#(c)} = f_\# \partial(c) = f_\# \partial(\overline{c})$,
de donde se deduce $\partial f_\# = f_\# \partial$.

A la función $f_\# \colon S_*(\X, A) \to S_*(\Y, B)$ la llamamos la aplicación de cadenas inducida por la aplicación de pares $f$.
Nuevamente, se verifica que $(g \circ f)_\# = g_\# \circ f_\#$ y $Id_\# = Id$.

Definimos $f_* \colon H_p(\X, A) \to H_p(\Y, B)$ por $f_*([\overline{c}]) = [f_\#(\overline{c})]$, que está bien definida.
Claramente, $f_*$ es un homomorfismo de grupos, que verifica $(g \circ f)_* = g_* \circ f_*$ y $Id_* = Id$.

Así, hemos construído un funtor covariante de la categoría de pares y aplicaciones entre pares en la de grupos graduados y homomorfismos,
el funtor de la homología singular.
\[
\begin{tikzcd}
  (\X, A) \arrow{r} \arrow{d}{f} & [3em] \{H_p(\X, A)\}_{p \geq 0} \arrow{d}{f_*} \\
  (\Y, B) \arrow{r} & \{H_p(\Y, B)\}_{p \geq 0}
\end{tikzcd}
\]
Tenemos un resultado de obtención directa:

\begin{corollary}
  Si $h \colon (\X, A) \to (\Y, B)$ es un homeomorfismo de $\X$ sobre $\Y$ que verifica $h(A) = B$, entonces
  $h_* \colon H_p(\X, A) \to H_p(\Y, B)$ es un isomorfismo de grupos $\forall p \geq 0$.
\end{corollary}

\begin{proof}
  Basta considerar el inverso de $h$, que induce una aplicación inversa en la homología.
\end{proof}

Vamos a trasladar los resultados que vimos de arcoconexión para el caso de pares.

\begin{proposition}\label{prop1-17}
  Si $\X$ es arcoconexo y $A \neq \emptyset$, se tiene $H_0(\X, A) = 0$.
\end{proposition}

\begin{proof}
  Sea $[\overline{c}] \in H_0(\X, A)$. $\partial\overline{c} = 0$, con $c = \sum\limits_{i = 1}^n n_i x_i \quad x_i \in \X$. \\
  $\overline{c} = \sum\limits_{i = 1}^n n_i \overline{x_i}$ y así, tenemos $[\overline{c}] = \sum\limits_{i = 1}^n n_i [\overline{x_i}]$. \\
  Veamos que $\forall x \in \X, [\overline{x}] = 0$.

  Como $A \neq \emptyset$, sea $a \in A$. Por arcoconexión, tomamos $\alpha \colon \sigma_1 \to \X$ verificando $\alpha(0, 1) = x, \alpha(1, 0) = a$.
  Se cumple que $\overline{\alpha} \in S_1(\X, A)$ y $\partial\overline{\alpha} = \overline{\partial\alpha} = \overline{x - a} = \overline{x} - \overline{a}
  = \overline{x}$, pues $a \in S_0(A) \implies \overline{a} = 0$. \\
  De esta forma, $[\overline{x}] = [\partial\overline{a}] = 0$, ya que es un borde.
\end{proof}

\begin{proposition}
  Sea $X = \bigcup\limits_{\alpha \in A} \X_\alpha$ la descomposición de $\X$ en componentes arcoconexas, y sea $A \subseteq \X$. Entonces se verifica:
  \begin{itemize}
    \item[a)] $\forall \alpha, i_{\alpha *} \colon H_p(\X_\alpha, \X_\alpha \cap A) \to H_p(\X, A)$ es un monomorfismo,
    \item[b)] $H_p(\X, A) = \bigoplus\limits_{\alpha \in A}  i_{\alpha *}(H_p(\X_\alpha, \X_\alpha \cap A))$.
  \end{itemize}
\end{proposition}

La demostración es similar al caso visto para espacios simples. No la repetimos, pues no aporta nada nuevo.

Tenemos también un resultado equivalente al ya visto para aplicación homotópicas.

\begin{proposition}
  Sean $f, g \colon (\X, A) \to (\Y, B)$ aplicaciones de pares. Si existe $H \colon \X \times I \to \Y$ continua tal que
  $H(\_, 0) = f, H(\_, 1) = g$ y $H(A, t) \subseteq B \hspace{0.2em} \forall t \in I$, entonces $g_* = f_*$. En tal caso, se dice que
  f y g son homotópicas.
\end{proposition}

La demostración es similar a la ya vista, por lo que no se incluye.

Consideremos ahora la proyección $\pi \colon \SX{p} \to S_p(\X, A) \cong \frac{\SX{p}}{i_*(S_p(A))}$. \\
Tenemos que $\pi(c) = \overline{c}$, y la aplicación de cadenas inducida por la inclusión,
$i_\# \colon S_p(A) \to \SX{p}$, que es claramente un monomorfismo. \\
$\pi$ es un epimorfismo, y se verifica que $\Img i_\# \cong \Ker \pi = i_\#(S_p(A))$ por la definición de $S_p(\X, A)$.
Esto nos permite definir la sucesión:
\[0 \to S_p(A) \xrightarrow{i_\#} \SX{p} \xrightarrow{\pi} S_p(\X, A) \to 0, \]
que verifica $\partial i_\# = i_\# \partial$ y $\partial \pi = \pi \partial$. \\
A estas sucesiones, en las que cada triple $C \xrightarrow{f} D \xrightarrow{g} E$ verifica $\Img f = \Ker g$, las llamaremos exactas.
Si además tienen solo tres elementos, como en el caso anterior, las llamaremos sucesiones exactas cortas.

Podemos extender el resultado anterior, y obtenemos una sucesión exacta corta de complejos de cadenas:
\[0 \to S_*(A) \xrightarrow{i_\#} S_*(\X) \xrightarrow{\pi} S_*(\X, A) \to 0. \]

Sea ahora $\pi_* \colon H_*(\X) \to H_*(\X, A)$ el homomorfismo inducido por $\pi$. \\
$\pi_*([c]) = [\overline{c}]$ que está bien definido, pues si $[c] = [d] \implies d = c + \partial h \implies \overline{d} = \overline{c} + \partial \overline{h}
\implies [\overline{c}] = [\overline{d}]$.

Veamos que $\Img i_* = \Ker \pi_*$.

Se verifica $\pi_* i_* ([a]) = \pi_*[i_\#(a)] = [\pi i_\#(a)] = 0$, pues $\pi i_\# = 0$. Así, $\Img i_* \subseteq \Ker \pi_*$.

Si $[c] \in \Ker \pi_*, \pi_*([c]) = [\overline{c}] = 0 \implies \overline{c} = \overline{\partial d}, c = \partial d + i_\#(a)$,
luego $\partial c = \partial^2 d + \partial i_\#(a) \implies \partial c = \partial^2 d + i_\# \partial a$, y se deduce que $\partial a = 0$.
$a$ es un ciclo y genera $[a]$. Como $c = \partial d + i_\#(a), [c] = [\partial d] + [i_\#(a)] = i_*[a] \implies \Ker \pi_* \subseteq \Img i_*$,
lo que demuestra la igualdad.

Generalmente $i_*$ no tiene por qué ser un monomorfismo, y $\pi_*$ no tiene por qué ser un epimorfismo, aunque se verifica el siguiente resultado.

\begin{proposition}
  $\pi_* \colon H_0(\X) \to H_0(\X, A)$ es un epimorfismo.
\end{proposition}

\begin{proof}
  Si tomamos $[\overline{c}] \in H_0(\X, A)$ entonces $c \in S_0(\X) = Z_0(\X)$, y por tanto $[\overline{c}] \in H_0(\X)$ y
  $\pi_*([c]) = [\overline{c}]$.
\end{proof}

\begin{remark}
  Para p mayor que 1 no podemos asegurar que sea un ciclo, por lo que el resultado solo vale para el caso visto.
\end{remark}

\begin{proposition}
  Para todo $p \geq 1$ existe un homomorfismo $\Delta \colon H_p(\X, A) \to H_{p-1}(A)$ tal que $\Img \Delta = \Ker i_*$ y $\Ker \Delta = \Img \pi_*$.

  A la sucesión exacta así construída:
  \[\dots H_{p+1}(\X, A) \xrightarrow{\Delta} H_p(A) \xrightarrow{i_*} H_p(\X) \xrightarrow{\pi_*} H_p(\X, A) \xrightarrow{\Delta} H_{p-1}(A) \dots\]
  se le llama la \textbf{\underline{\smash{sucesión exacta}}} del par $(\X, A)$.
\end{proposition}

\begin{proof}
  En primer lugar, vamos a definir $\Delta$.

  Sea $[\overline{c}] \in H_p(\X, A)$, con $\overline{c} \in S_p(\X, A)$ y $\partial \overline{c} = 0$. Como $\partial \overline{c} = \overline{\partial c} = 0,
  \partial c \in i_\#(S_{p-1}(A))$. Por la inyectividad de $i_\#$, existe un único $a \in S_{p-1}(A)$ tal que $\partial c = i_\#(a)$. \\
  Además, $i_\# \partial a = \partial i_\#(a) = \partial^2 c = 0$, y al ser $i_\#$ monomorfismo, $\partial a = 0$ luego $a \in Z_{p-1}(A), [a] \in H_{p-1}(A)$.

  Definimos pues $\Delta([\overline{c}]) = [a]$, que está bien definido, ya que:
  \begin{align*}
    [\overline{c}] = [\overline{d}] &\implies \overline{d} = \overline{c} + \partial \overline{h} \implies d = c + \partial h + i_\#(e) \\
    &\implies \partial d = \partial c + i_\#(\partial e) = i_\#(a) + i_\#(\partial e) = i_\#(a + \partial e) \\
    &\implies \Delta([\overline{d}]) = [a + \partial e] = [a] = \Delta([\overline{c}]).
  \end{align*}

  Veamos que $\Delta$ es homomorfismo, es decir, $\Delta([\overline{c_1}] + [\overline{c_2}]) = \Delta([\overline{c_1 + c_2}])$.

  $\Delta([\overline{c_1}]) = [a_1]$ con $\partial c_1 = i_\#(a_1)$, \\
  $\Delta([\overline{c_2}]) = [a_2]$ con $\partial c_2 = i_\#(a_2)$.

  De esta forma, $\partial(c_1 + c_2) = i_\#(a_1 + a_2)$ y $\Delta([\overline{c_1 + c_2}]) = [a_1 + a_2]$.

  Veamos que $\Img \Delta = \Ker i_*$.

  $i_*(\Delta([\overline{c}])) = i_*[a] = [i_\#(a)] = [\partial c] = 0$, de donde $\Img \Delta \subseteq \Ker i_*$.

  Si $[a] \in H_{p-1}(A)$ verificando $i_*(a) = 0$, entonces $[i_\#(a)] = 0$, y eso nos dice que $i_\#(a) = \partial b, b \in S_p{\X}$.
  Además, $\overline{\partial b} = \overline{i_\#(a)} = 0$, y como $\overline{\partial b} = \partial \overline{b}, b \in Z_p(\X, A)$ con
  $\partial b = i_\#(a)$. Por definición, $\Delta([\overline{b}]) = [a]$. Así se da la igualdad.

  Veamos que $\Ker \Delta = \Img \pi_*$.

  $\Delta\pi_*([c]) = \Delta([\overline{c}]) = [a]$, con $\partial c = i_\#(a)$. \\
  Como $c$ es un ciclo, $i_\#(a) = 0$, y al ser $i_\#$ un monomorfismo, $a = 0$, luego $[a] = 0$. Así, $\Img \pi_* \subseteq \Ker \Delta$.

  Si $[\overline{c}] \in \Ker \Delta, \Delta[\overline{c}] = [\overline{a}] = 0$, con $\partial c = i_\#(a)$. Como $[a] = 0, a = \partial b, b \in \S_p(A)$.
  Así, $\partial(c - i_\#(b)) = 0$, y $c - i_\#(b) \in Z_p(\X)$, de donde obtenemos $[c - i_\#(b)]$. \\
  $\pi_*([c . i_\#(b)]) = \pi_*([c]) + e = [\overline{c}]$. Con esto, $\Img \pi_* = \Ker \Delta$.
\end{proof}

\begin{proposition}[Naturalidad de la sucesión exacta]
  Sea $f \colon (\X, A) \to (\Y, B)$ una aplicación de pares. Entonces el siguiente diagrama es conmutativo:
  \[  \begin{tikzcd}
    \dots H_p(A) \arrow{r}{i_*} \arrow{d}{(f_{|A})_*} & H_p(\X) \arrow{r}{\pi_*} \arrow{d}{f_*} & H_p(\X, A) \arrow{r}{\Delta} \arrow{d}{f_*} & H_{p-1}(A) \dots \arrow{d}{(f_{|A})_*} \\
    \dots H_p(B) \arrow{r}{i_*} & H_p(Y) \arrow{r}{\pi_*} & H_p(\Y, B) \arrow{r}{\Delta} & H_{p-1}(B) \dots
  \end{tikzcd} \]
\end{proposition}

\begin{proof}
  Nos basta con probar que $\Delta \circ f_* = f_{|A} \circ \Delta$, pues ya conocemos el resto.

  Realizamos el cálculo:
  \[ (\Delta \circ f_*)([c]) = \Delta([f_\#(\overline{c}]) = \Delta([\overline{f_\#(c)}]) \stackrel{(*)}{=} [(f_{|A})_\#(a)]
    = (f_{|A})_*[a] = (f_{|A})_* \Delta[\overline{c}]. \]

  Donde en $(*)$ estamos usando que $\partial(f_\#(c)) = f_\# \partial c = f_\# i_\#(a) = i_\#(f_{|A})_\#(a)$, con $\partial c = i_\#(a).$
\end{proof}
