%************************************************
\label{ch:introduccionHOM}
%************************************************

\section{Introducción al trabajo}

El propósito de este trabajo es introducir la teoría de homología singular sobre
espacios topológicos. Esto lo haremos de forma constructiva, introduciendo los conceptos
y estructuras necesarios para poder definir la homología singular.
Tras definirla, estudiaremos sus propiedades más inmediatas y algunos casos sencillos,
y extenderemos el concepto para pares topológicos, trasladando los resultados obtenidos
con anterioridad.

Una vez asentadas las bases, trataremos el teorema de subdivisión baricéntrica, el
resultado más importante del trabajo, que nos permite obtener herramientas para el
cálculo de la homología de espacios topológicos. Será necesario introducir nociones
geométricas y demostrar ciertas propiedades de estas nuevas construcciones para la
demostración del teorema.

Demostrado el teorema de subdivisión baricéntrica, trataremos sus consecuencias,
siendo las más importantes la sucesión de Mayer-Vietoris y el teorema de escisión.
Con ellas, seremos capaces de calcular la homología de las esferas, y lo haremos
de dos formas distintas; una mediante escisión, y otra usando la sucesión de Mayer-Vietoris.

Después de calcular la homología de las esferas demostraremos un número de resultados clásicos,
para lo que definiremos la homología local de un espacio topológico. Obtendremos, entre otros,
el teorema del punto fijo de Brower, el teorema de separación de Jordan-Brower y
ciertos teoremas de invarianza. Finalmente, apoyándonos en la definición del grado de
una aplicación continua sobre una esfera, seremos capaces de demostrar resultados
sobre campos definidos en esferas.

\section{Objetivos}
Los objetivos de este trabajo son los siguientes:

\begin{itemize}
  \item Construcción de la homología singular y estudio de sus propiedades.

  \item Cálculo de la homología singular de las esferas y obtención de aplicaciones.

\end{itemize}

El primer objetivo se ha cumplido satisfactoriamente, y está desarrollado tanto en el
\autoref{ch:introduccion} como en el \autoref{ch:calculoI}.

El segundo objetivo también se ha cumplido, con el cálculo de la homología de las esferas
al final del \autoref{ch:calculoI}, y la obtención de aplicaciones desarrollada en el
\autoref{ch:resultadosHom}.

\section{Organización de la memoria}

El \autoref{ch:introduccion} contiene todo lo referente a la definición de la homología
singular de un espacio topológico: definiciones, propiedades y construcciones algebraicas
y geométricas. También se demostrará la invarianza homotópica de la homología, y se
definirá la homología de los pares topológicos.

En el \autoref{ch:calculoI} se demostrará el teorema de subdivisión baricéntrica y
se obtendrán herramientas para el cálculo de la homología singular: el teorema
de escisión y la sucesión de Mayer-Vietoris. Estos serán utilizados para deducir
la homología de las esferas.

Finalmente, en el \autoref{ch:resultadosHom} obtendremos resultados clásicos, como el
teorema del punto fijo de Brower, el teorema de separación de Jordan-Brower y algunos
teoremas de invarianza. Para ello, nos haremos valer de los conceptos de homología
local y de grado de una aplicación continua definida en una esfera.

\section{Referencias}

Para la realización de este trabajo se ha utilizado principalmente el libro de
teoría de homología de James W. Vick \citep{vick:1973}, complementado por los títulos
de William S. Massey \citep{massey:1972} y Sze-Tsen Hu \citep{hu:1966}, que también
tratan la teoría de homología.
