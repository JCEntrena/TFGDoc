%************************************************
\chapter{Introducción}\label{ch:introduccionMCP}
%************************************************

\section{Introducción al problema}

Dado un grafo $G = (V, E)$, donde $V$ representa el conjunto de vértices
y $E$ el conjunto de aristas, llamaremos un clique a un subgrafo de $G$
en el que todos sus vértices están conectados entre sí. Llamaremos a los
cliques \textbf{maximales} si no son subgrafo de un clique de mayor tamaño,
y \textbf{máximos} si no existe ningún clique de mayor tamaño en el grafo.
Es claro que todos los cliques máximos son maximales, pero no al contrario.

El problema del clique máximo, o \textit{maximum clique problem} en inglés (MCP),
consiste en encontrar el clique de mayor número de vértices dentro de un
grafo $G$ cualquiera. Este problema es equivalente a otros dos también muy conocidos,
que son el problema del conjunto independiente, en el que se intenta buscar el mayor
conjunto de nodos no conectados entre sí, y el problema del recubrimiento por vértices,
en el que buscamos el mínimo conjunto de nodos que cubran todas la aristas.

% Intentar añadir un gráfico aquí, para explicar un poco.

El problema del clique máximo lleva siendo estudiado desde segunda mitad del siglo XX, y fue sujeto
de estudio en el segundo desafío de implementación de DIMACS, que tuvo lugar entre 1992
y 1993, junto con el problema del coloreo de grafos y de satisfabilidad booleana (SAT).
A día de hoy, se encuentran disponibles de forma libre las instancias utilizadas y los
datos de los mejores cliques conocidos hasta la fecha.

En 1979, Garey y Johnson \citep{garey:1979} demostraron que el problema es NP-difícil.
En 1999, Hastad \citep{hastad:1999} probó que, salvo que NP = ZPP, no existe ningún
algoritmo capaz de aproximar MCP con factor $n^{1-\epsilon} \quad \forall \epsilon$,
donde la clase ZPP (zero-error probabilistic polynomial time) es la clase de complejidad
de los problemas reconocibles por algoritmos aleatorizados con complejidad polinómica.

Debido a esto, no es posible obtener algoritmos exactos ni que den una solución
aproximada al problema, de ahí la necesidad del uso de algoritmos heurísticos para
resolverlo.

\section{Introducción a las metaheurísticas}

En Inteligencia Artificial, se emplea el término \textbf{heurística} para referirnos
a una técnica, método o procedimiento de realizar una tarea, que no es producto de
un análisis formal riguroso, sino del conocimiento experto sobre la tarea \citep{melian:2003}.

El uso de heurísticas es una parte fundamental en la resolución de problemas de
optimización, pues usualmente estos serán lo bastante complejos como para no poder
resolverlos de forma exacta en un tiempo razonable. Así, somos capaces de crear
procedimientos heurísticos que obtienen soluciones cercanas al óptimo en un tiempo
razonable \citep{herrera:2014}.

Las heurísticas diseñadas para resolver un problema pueden ser más generales o más
específicas que otras. Los métodos más específicos deben ser diseñados de forma
distinta para cada problema, utilizando toda la información y el análisis teórico
disponible sobre él. Bien diseñados, suelen tener un rendimiento más alto que las
heurísticas generales. Por el contrario, las heurísticas más generales son más sencillas,
adaptables y robustas, pues no dependen de forma tan fuerte del problema que se resuelve.

Las metaheurísticas pueden ser vistas como estrategias de diseño generales para
procedimientos heurísticos de alto rendimiento. Estas estrategias quedan por encima
de las heurísticas convencionales, representando un \textit{nivel superior} de abstracción.
El término metaheurística apareció por primera vez en un artículo de Fred Glover
\citep{glover:1986}. Desde entonces, han surgido numerosas propuestas, mediante la
aparición de pautas para diseñar procedimientos para resolver distintos problemas,
los cuales, al ser extendidos a otros ámbitos, han adquirido la denominación de metaheurísticas.

Podemos clasificar las metaheurísticas en los siguientes tipos \citep{melian:2003}:
\begin{itemize}
  \item Las metaheurísticas de relajación utilizan un modelo relajado del problema
        original, que lo hace más fácil de resolver, y cuya solución facilita la
        solución del problema original.

  \item Las metaheurísticas de búsqueda siguen movimientos o transformaciones guiadas
        para explorar el espacio de soluciones, aprovechando las estructuras de
        entornos asociadas.

  \item Las metaheurísticas constructivas tratan de obtener una solución a través
        de un proceso de análisis y selección de sus componentes.

  \item Las metaheurísticas evolutivas trabajan con un conjunto de soluciones que
        va cambiando a lo largo del tiempo mediante el intercambio de información.
\end{itemize}

Las metaheurísticas pueden ser de uno o varios tipos, lo que daría lugar a metaheurísticas
híbridas. También es posible encontrar metaheurísticas basadas en algún recurso que
no se engloben en estos cuatro tipos, como las basadas en redes neuronales. Esta
clasificación queda abierta a la interpretación, pudiendo ser modificada, y no
debe tomarse nunca como definitiva o infalible.

\section{El repositorio DIMACS}
Las siglas DIMACS hacen referencia al centro de matemática discreta y computación
teórica (del inglés, Center for Discrete Mathematics and Theoretical Computer Science).
Es una colaboración entre dos universidades estadounidenses, las de Rutgers y Princeton,
además de las secciónes de investigación de las empresas AT\&T, Bell Labs, Applied
Communication Science y NEC. Fue fundada en el año 1989, y desde 1990 convoca desafíos
de implementación para determinados problemas de interés en el ámbito científico, con
el objetivo de conseguir algoritmos de calidad.

El problema del clique máximo fue, junto con el coloreo de grafos y el problema de satisfabilidad
de cláusulas (SAT), protagonista del segundo desafío DIMACS, que tuvo lugar entre 1990 y 1991.
Para el problema, se proporcionó un conjunto de instancias de grafos, que son de acceso público,
son las que utiliza la comunidad científica para evaluar nuevos algoritmos, y las que he utilizado
para los considerados en este trabajo. Pueden encontrarse en la URL
\url{http://iridia.ulb.ac.be/~fmascia/maximum_clique/DIMACS-benchmark}.

En dicho repositorio podemos encontrar un total de $37$ grafos, de los cuales he tomado $35$.
Ha quedado fuera los grafos \textit{MANN\_a81} y \textit{MANN\_a45}, debido a que por su tamaño
los tiempos de ejecución eran muy elevados. Para ejecutar sobre ellos los algoritmos con los
mismos parámetros que sobre el resto de grafos, se hubiera requerido de más de una semana.

Los grafos están subdivididos en distintos grupos, que hacen referencia al método de
generación de los mismos. Debido a esto, es posible que algunos algoritmos muestren
un buen comportamiento en ciertos tipos de grafos, y den peores resultados en grafos
que correspondan a otra familia. Los métodos de generación de cada familia de grafos
se pueden consultar con detalle en la web del repositorio.

En ciertas instancias se ha confirmado el valor del óptimo global, mientras que en
otras solo se conoce una cota inferior a la mejor solución posible. Estos valores
serán tomados como referencia a la hora de evaluar cada algoritmo implementado y
determinar la calidad de las soluciones.

\section{Objetivos}
Los objetivos de este trabajo son los siguientes:

\begin{itemize}
  \item Recopilación de bibliografía sobre el problema del clique máximo y sobre
        las posibles metaheurísticas a aplicar al problema. Con esto, queremos
        aprovechar el conocimiento existente sobre el problema para tener una
        base sobre la que trabajar, junto a un conjunto de técnicas que se pueden
        utilizar para resolver el problema. Así, no solo se conocerá mejor el problema,
        sino que será posible abordarlo de forma más eficiente.

  \item Estudio y selección de las técnicas a utilizar. Buscaremos, de entre todas
        las técnicas aplicables, aquellas que sean de interés a la hora de realizar
        un posterior análisis.

  \item Diseño e implementación de los algoritmos seleccionados. Tomaremos las
        características de los algoritmos que nos interesen, y las modificaremos
        para adaptarlas a la representación de problema, aprovechando la funcionalidad
        del lenguaje de programación escogido.

  \item Experimentación y estudio comparativo de los algoritmos implementados.
        Una vez obtengamos los datos experimentales, compararemos los algoritmos
        para comprobar empíricamente su funcionamiento, basándonos en la calidad
        de las soluciones obtenidas por cada uno de ellos. Veremos cómo actúan
        sobre el conjunto de instancias del problema y discutiremos su viabilidad
        a la hora de resolverlo.
\end{itemize}

El primer objetivo se ha cumplido en cierta medida, pues, si bien se ha recopilado
bibliografía referente al problema del clique máximo, esta se centraba en su gran
mayoría en los métodos de resolución del problema, y no tanto en el estudio teórico
del mismo.

El segundo objetivo se ha cumplido en un grado alto, como se demuestra en la recopilación
de algoritmos implementados que veremos en el \autoref{ch:algoritmos}.

El tercer objetivo se ha completado satisfactoriamente, como veremos en el \autoref{ch:implementacion}.

El cuarto objetivo también se ha completado de forma satisfactoria, y queda
reflejado en el \autoref{ch:resultados}.

\section{Metodología y trabajo realizado}
% Sin dar detalles de la implementación.

(Aun por completar)

%Este trabajo se ha desarrollado siguiendo una metodología ágil, tratando de crear
%un proyecto desde cero e ir agregando distintas componentes de forma incremental.

%He implementado un programa que incluye los distintos algoritmos considerados,
%para comprobar sus eficacia a la hora de resolver el problema.

%Las ideas detrás de cada algoritmo provienen de una combinación de mis conocimientos
%sobre la materia y el estudio que se lleva realizando sobre el problema durante décadas,
%reflejado en numerosos artículos, que han servido como base en la mayoría de implementaciones.
%Se referenciará en cada caso el material utilizado y cómo se ha adaptado a este trabajo.

\section{Organización}

El resto de la memoria está organizada de la siguiente forma:

Primero, introduciremos teóricamente los algoritmos considerados en este trabajo.
Para ello, comenzaremos con un pequeño apartado donde se tratan elementos generales a
todos ellos, seguido de un desglose de los algoritmos divididos según la metaheurística
a la que pertenezcan, consistente en una introducción a la metaheurística correspondiente,
seguida de cómo esta se adapta al problema del clique máximo, y concluyendo con los
detalles de cada algoritmo.

Posteriormente, trataremos la parte de implementación, en la que estudiaremos todo lo
referente a la parte de programación de este trabajo: la organización de archivos,
la lectura de datos y la implementación de las técnicas descritas con anterioridad,
teniendo en cuenta el lenguaje de programación utilizado.

Una vez introducidas y explicadas las técnicas que se han utilizado y su implementación,
pasaremos a la sección de resultados, donde se expondrán los cálculos que cada algoritmo
ha realizado sobre el conjunto de instancias. Seguidamente, se realizará un análisis
de dichas técnicas, en los que evaluaremos si han tenido el comportamiento esperado y
las causas de los resultados obtenidos. Finalmente, compararemos entre sí todas las
técnicas implementadas, tratando de clasificarlas según su funcionamiento.
