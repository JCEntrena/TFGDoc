%************************************************
\chapter{Introducción}\label{ch:introduccionMCP}
%************************************************

\section{Introducción al problema}

Dado un grafo $G = (V, E)$, donde $V$ representa el conjunto de vértices
y $E$ el conjunto de aristas, llamaremos un clique a un subgrafo de $G$
en el que todos sus vértices están conectados entre sí. Llamaremos a los
cliques \textbf{maximales} si no son subgrafo de un clique de mayor tamaño,
y \textbf{máximos} si no existe ningún clique de mayor tamaño en el grafo. \\
Es claro que todos los cliques máximos son maximales, pero no al contrario.

El problema del clique máximo, o \textit{maximum clique problem} en inglés (MCP),
consiste en encontrar el clique de mayor número de vértices dentro de un
grafo $G$ cualquiera. Este problema es equivalente a otros dos también muy conocidos,
que son el problema del conjunto independiente, en el que se intenta buscar el mayor
conjunto de nodos no conectados entre sí, y el problema del recubrimiento por vértices,
en el que buscamos el mínimo conjunto de nodos que cubran todas la aristas.

Este problema lleva siendo estudiado desde segunda mitad del siglo XX, y fue sujeto
de estudio en el segundo desafío de implementación de DIMACS, que tuvo lugar entre 1992
y 1993, junto con el problema del coloreo de grafos y de satisfabilidad booleana (SAT).
A día de hoy, se encuentran disponibles de forma libre las instancias utilizadas y los
datos de los mejores cliques conocidos hasta la fecha.

En 1977, Garey y Johnson demostraron que el problema es NP-difícil. En 1999, Hastad
probó que, salvo que NP = ZPP, no existe ningún algoritmo capaz de aproximar MCP con
factor $n^{1-\epsilon} \quad \forall \epsilon$, donde la clase ZPP (zero-error
probabilistic polynomial time) es la clase de complejidad de los problemas
reconocibles por algoritmos aleatorizados con complejidad polinómica.

Debido a esto, no es posible obtener algoritmos exactos ni que den una solución
aproximada al problema, de ahí la necesidad del uso de algoritmos heurísticos para
resolverlo.


\section{Trabajo realizado}

En este trabajo he implementado distintos algoritmos, basados en diferentes
metaheurísticas, para comprobar sus eficacia a la hora de resolver el problema.
En varios de ellos se incluyen distintas variantes de la misma heurística, para
analizar cómo afectan cambios en la toma de ciertas decisiones, o la estructura
de elementos del problema.

El objetivo final del trabajo es comparar todos los algoritmos implementados
para comprobar empíricamente su eficacia, basándonos fundamentalmente en la
calidad de las soluciones obtenidas por cada uno de ellos. Veremos como actúan
sobre instancias similares del problema, y discutiremos su viabilidad a la
hora de resolverlo, pues cabe esperar que ciertos algoritmos se comporten
mejor que otros.

Además de la calidad de las soluciones, tendremos en cuenta el tiempo de ejecución
de los algoritmos, si bien no será un aspecto relevante a la hora de evaluarlos.

Las ideas detrás de cada algoritmo provienen de una combinación de mis conocimientos
sobre la materia y el estudio que se lleva realizando sobre el problema durante décadas,
reflejado en numerosos artículos, que han servido como base en la mayoría de implementaciones.
Se referenciará en cada caso el material utilizado y cómo se ha adaptado a este trabajo.

\subsection{Implementación}

La implementación del código de este trabajo ha sido realizada en el lenguaje de
programación \textbf{Ruby}, usando su versión 2.3.0. \textbf{Ruby} es un lenguaje
orientado a objetos, interpretado, con una sintaxis muy similar a la de \textbf{Python}.
Está disponible para descarga bajo una licencia de software libre.

He elegido este lenguaje por haber trabajado con él en varias ocasiones, y aporta
simplicidad e intuitividad a la hora de programar los algoritmos, además de salir
un poco de los lenguajes más utilizados en la carrera.

Al utilizar un lenguaje interpretado como es \textbf{Ruby}, cabe esperar unos
tiempos de ejecución mayores que con otros lenguajes, debido a que no está hecho
para ejecutar código lo más rápido posible. En el caso de querer tener en cuenta la
velocidad, lenguajes compilados como \textbf{C++} serían más útiles que \textbf{Ruby}.

El código implementado cumple varias funciones: la primera de ellas es llevar a
memoria todas las instancias disponibles del problema. Para ello, he tenido que
implementar un lector de ficheros que cree una matriz de adyacencia para cada archivo
disponible. Estos archivos siguen una estructura determinada, que permite
automatizar dicho proceso, no teniendo que considerar casos independientes.
Cada archivo dará lugar a un problema, compuesto de una matriz de adyacencias,
el número de nodos y aristas involucrados, y el nombre del grafo en cuestión.

El segundo punto, y más importante, es la implementación de las heurísticas.
Cada una de ellas está contenida en una clase propia, que contendrá uno o varios
métodos para la resolución de los problemas generados a partir de los archivos.
Adicionalmente, podrán contener métodos auxiliares para uso propio o de otras
heurísticas, en caso de que utilicemos técnicas híbridas. Cada clase estará
en un archivo independiente, y todas ellas se recogerán en un mismo módulo.

Finalmente, existe un archivo adicional con métodos auxiliares, que pueden ser
requeridos por varias heurísticas. La idea bajo su creación es encapsular
operaciones importantes, que serán repetidas frecuentemente y en diversos
escenarios, para facilitar el trabajo y la comprensión del código. Dichos
métodos no se encuentran dentro de una clase, pero sí que están en un
módulo aparte, para diferenciarlos de las implementaciones de las heurísticas.


\section{El repositorio DIMACS}
Las siglas DIMACS hacen referencia al centro de matemática discreta y computación teórica
(del inglés, Center for Discrete Mathematics and Theoretical Computer Science). Es una colaboración
entre dos universidades estadounidenses, las de Rutgers y Princeton, además de las secciónes de investigación
de las empresas AT\&T, Bell Labs, Applied Communication Science y NEC. Fue fundada en el año 1989,
y desde 1990 convoca desafíos de implementación para determinados problemas de interés en
el ámbito científico, con el objetivo de conseguir algoritmos de calidad.

El problema del clique máximo fue, junto con el coloreo de grafos y el problema de satisfabilidad
de cláusulas (SAT), protagonista del segundo desafío DIMACS, que tuvo lugar entre 1990 y 1991.
Para el problema, se proporcionó un conjunto de instancias de grafos, que son de acceso público,
y son las que utiliza la comunidad científica para evaluar nuevos algoritmos, que he utilizado
para los míos. Pueden encontrarse en la URL \url{http://iridia.ulb.ac.be/~fmascia/maximum_clique/DIMACS-benchmark}.

En dicho repositorio podemos encontrar un total de 37 grafos. Están subdivididos en distintos grupos,
que hacen referencia al método de generación de los mismos. Debido a esto, es posible que algunos
algoritmos muestren un buen comportamiento en ciertos tipos de grafos, y den peores resultados en
grafos que correspondan a otra familia. Los métodos de generación de cada familia de grafos se pueden
consultar en profundidad en la web del repositorio.

En ciertas instancias se ha confirmado el valor del óptimo global, mientras que en otras solo se conoce
una cota inferior a la mejor solución posible. Estos valores serán tomados como referencia a la hora
de evaluar cada algoritmo implementado y determinar la calidad de las soluciones.
