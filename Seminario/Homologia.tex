%%%
% Plantilla de Presentación
% Modificación de una plantilla de Latex de LaTeXTemplates para adaptarla
% al castellano y a las necesidades de escribir informática y matemáticas.
%
% Tomada de https://github.com/M42/plantillas/blob/master/presentacion/presentacion.tex
% Créditos a MRoman42 por la adaptación.
% Más adaptaciones a mi gusto.
%
% License:
% CC BY-NC-SA 3.0 (http://creativecommons.org/licenses/by-nc-sa/3.0/)
%%%

%%%%%%%%%%%%%%%%%%%%%%%%%%%%%%%%%%%%%%%%%
% Beamer Presentation
% LaTeX Template
% Version 1.0 (10/11/12)
%
% This template has been downloaded from:
% http://www.LaTeXTemplates.com
%%%%%%%%%%%%%%%%%%%%%%%%%%%%%%%%%%%%%%%%%

%----------------------------------------------------------------------------------------
%	PAQUETES Y CONFIGURACIÓN DEL DOCUMENTO
%----------------------------------------------------------------------------------------

\documentclass{beamer}

%% Configuración de la presentación
\mode<presentation> {
  %%% Selección de estilo
  % The Beamer class comes with a number of default slide themes
  % which change the colors and layouts of slides. Below this is a list
  % of all the themes, uncomment each in turn to see what they look like.

  %\usetheme{default}
  %\usetheme{AnnArbor}
  %\usetheme{Antibes}
  %\usetheme{Bergen}
  %\usetheme{Berkeley}
  %\usetheme{Berlin}
  %\usetheme{Boadilla}
  %\usetheme{CambridgeUS}
  %\usetheme{Copenhagen}
  %\usetheme{Darmstadt}
  %\usetheme{Dresden}
  %\usetheme{Frankfurt}
  %\usetheme{Goettingen}
  %\usetheme{Hannover}
  %\usetheme{Ilmenau}
  %\usetheme{JuanLesPins}
  %\usetheme{Luebeck}
  %\usetheme{Madrid}
  %\usetheme{Malmoe}
  %\usetheme{Marburg}
  %\usetheme{Montpellier}
  %\usetheme{PaloAlto}
  %\usetheme{Pittsburgh}
  %\usetheme{Rochester}
  %\usetheme{Singapore}
  %\usetheme{Szeged}
  \usetheme{Warsaw}

  %% Selección de color
  % As well as themes, the Beamer class has a number of color themes
  % for any slide theme. Uncomment each of these in turn to see how it
  % changes the colors of your current slide theme.

  %\usecolortheme{albatross}
  %\usecolortheme{beaver}
  %\usecolortheme{beetle}
  %\usecolortheme{crane}
  %\usecolortheme{dolphin}
  %\usecolortheme{dove}
  %\usecolortheme{fly}
  %\usecolortheme{lily}
  %\usecolortheme{orchid}
  %\usecolortheme{rose}
  %\usecolortheme{seagull}
  %\usecolortheme{seahorse}
  \usecolortheme{whale}
  %\usecolortheme{wolverine}

  %% Configuración del pie de línea
  %\setbeamertemplate{footline} % To remove the footer line in all slides uncomment this line
  %\setbeamertemplate{footline}[page number] % To replace the footer line in all slides with a simple slide count uncomment this line
  %\setbeamertemplate{navigation symbols}{} % To remove the navigation symbols from the bottom of all slides uncomment this line
}

\setbeamertemplate{headline}{%
\leavevmode%
  \hbox{%
    \begin{beamercolorbox}[wd=\paperwidth,ht=2.5ex,dp=2ex]{palette quaternary}%
    \insertsectionnavigationhorizontal{\paperwidth}{}{\hskip0pt plus1filll}
    \end{beamercolorbox}%
  }
}

%% Fuentes de tamaño arbitrario
\usepackage{lmodern}

%% Gráficos
\usepackage{graphicx} % Allows including images
\usepackage{booktabs} % Allows the use of \toprule, \midrule and \bottomrule in tables

%%% Castellano.
% noquoting: Permite uso de comillas no españolas.
% lcroman: Permite la enumeración con numerales romanos en minúscula.
% fontenc: Usa la fuente completa para que pueda copiarse correctamente del pdf.
\usepackage[english,spanish,es-noquoting,es-lcroman]{babel}
\usepackage[utf8]{inputenc}
\usepackage[T1]{fontenc}
\selectlanguage{spanish}

%-------------------------------------------
% Paquetes añadidos por JC.
%-------------------------------------------
% Tikz
\usepackage{tikz}
\usepackage{tikz-cd}
\usepackage{tikz-3dplot}

\usepackage{verbatim}
\usetikzlibrary{arrows,shapes}
\usepackage{amsthm}
\usepackage{accents}

% Definitions

\def\X{{\mathbb X}}
\def\A{{\mathbb A}}
\def\Y{{\mathbb Y}}
\def\B{{\mathbb B}}
\def\Z{{\mathbb Z}}
\def\C{{\mathbb C}}
\def\U{{\mathbb U}}
\def\R{{\mathbb R}}

\DeclareMathOperator{\Ker}{Ker}
\DeclareMathOperator{\Img}{Img}

\newcommand{\interior}[1]{\accentset{\smash{\raisebox{-0.12ex}{$\scriptstyle\circ$}}}{#1}\rule{0pt}{2.3ex}}

\theoremstyle{theorem}
\newtheorem{mytheorem}{Teorema}
\newtheorem{mycorollary}{Corolario}
\newtheorem{myproposition}{Proposición}
\newtheorem{mylemma}{Lema}
\newtheorem{mydefinition}{Definición}

% Counter

\newcounter{saveenumi}
\newcommand{\seti}{\setcounter{saveenumi}{\value{enumi}}}
\newcommand{\conti}{\setcounter{enumi}{\value{saveenumi}}}

% Sections
\AtBeginSection{\frame{\sectionpage}}
\newtranslation[to=spanish]{Section}{Sección}

\defbeamertemplate{section page}{mine}[1][]{%
  \begin{centering}
    {\usebeamerfont{section name}\usebeamercolor[fg]{section name}#1}
    \vskip1em\par
    \begin{beamercolorbox}[sep=12pt,center]{part title}
      \usebeamerfont{section title}\insertsection\par
    \end{beamercolorbox}
  \end{centering}
}

%----------------------------------------------------------------------------------------
%	TÍTULO
%----------------------------------------------------------------------------------------

\title[Homología singular]{Introducción a la homología singular} % The short title appears at the bottom of every slide, the full title is only on the title page

\author{José Carlos Entrena Jiménez} % Your name
\institute[UGR] % Your institution as it will appear on the bottom of every slide, may be shorthand to save space
{
  Universidad de Granada \\ % Your institution for the title page
  \medskip
  \textit{jentrena@correo.ugr.es} \\ % Your email address
  \medskip
  \textit{github.com/JCEntrena}
}
\date{\today} % Date, can be changed to a custom date



\begin{document}
% Spanish
\selectlanguage{spanish}
% Para tikz
\pgfdeclarelayer{background}\theoremstyle{definition}
\pgfsetlayers{background,main}
% Secciones
\setbeamertemplate{section page}[mine]

%% Diapositiva de título.
\frame{\titlepage}

%----------------------------------------------------------------------------------------
%	PRESENTACIÓN
%----------------------------------------------------------------------------------------

%------------------------------------------------

%% Bibliografía
\begin{frame}
\frametitle{Referencias}
\footnotesize{
  \begin{thebibliography}{99} % Beamer does not support BibTeX so references must be inserted manually as below

    \bibitem[James W. Vick, 1973]{vick:1973}
      \newblock Homology Theory. An Introduction to Algebraic Topology.
      \newblock Academic Press, 1973.

    \bibitem[W. S. Massey, 1972]{massey:1972}
      \newblock Introducción a la topología algebraica.
      \newblock Reverté, 1972.

  \end{thebibliography}
}
\end{frame}


%% Diapositiva de contenidos.
% Throughout your presentation, if you choose to use \section{} and \subsection{} commands,
% these will automatically be printed on this slide as an overview of your presentation
\begin{frame}
  \frametitle{Contenidos} % Table of contents slide, comment this block out to remove it
  \tableofcontents
\end{frame}


%------------------------------------------------

\section{Introducción}


\begin{frame}
  \frametitle{¿Qué vamos a hacer?}
  \begin{itemize}
    \item Construir la homología singular.
    \item Obtener resultados sencillos sobre la homología.
    \item Obtener herramientas para calcularla.
    \item Aplicarlas para calcular la homología de las esferas.
  \end{itemize}

\end{frame}

%------------------------------------------------

\begin{frame}
  \frametitle{¿Cómo lo vamos a hacer?}
  \begin{itemize}
    \item Construiremos la homología singular usando nociones geométricas y algebraicas.
    \item Demostraremos resultados mediante nuevas definiciones y el uso de propiedades.
  \end{itemize}

\end{frame}


%------------------------------------------------
\section{Definiciones}

\begin{frame}
  \frametitle{Definiciones}
  \begin{columns}[c]

    \column{.45\textwidth}
    Un $p$-símplice es la envolvente convexa de $p+1$ puntos afínmente independientes.
    Lo llamaremos estándar si sus puntos son de la forma $(0, \dots, 1, \dots, 0)$, y lo notaremos $\sigma_p$.

    \column{.5\textwidth}
    \begin{figure}
      \tdplotsetmaincoords{70}{135}
      \begin{tikzpicture}[tdplot_main_coords]
        \def\laxis{3}
        \def\ltriangle{1}
        \def\ltick{.2}
        %%% axes
        \draw [->] (0,0,0) -- (\laxis,0,0) node [below] {$x$};
        \draw [->] (0,0,0) -- (0,\laxis,0) node [right] {$y$};
        \draw [->] (0,0,0) -- (0,0,\laxis) node [left] {$z$};
        %%% axes ticks
        \pgfmathtruncatemacro{\nticks}{floor(\laxis)-1}
        \begin{scope}[
          help lines,
          every node/.style={inner sep=1pt,text=black}
          ]
          \foreach \coord in {1,...,\nticks} {
            \draw (\coord,\ltick,0) -- ++(0,-\ltick,0) -- ++(0,0,\ltick)
            node [pos=1,left] {\coord};
            \draw (\ltick,\coord,0) -- ++(-\ltick,0,0) -- ++(0,0,\ltick)
            node [pos=1,right] {\coord};
            \draw (\ltick,0,\coord) -- ++(-\ltick,0,0) -- ++(0,\ltick,0)
            node [at start,above right] {\coord};
          }
        \end{scope}
        %%% figure
        \filldraw [opacity=.33,red] (\ltriangle,0,0) -- (0,\ltriangle,0)
        -- (0,0,\ltriangle) -- cycle;
      \end{tikzpicture}
      \caption{$2$-símplice estándar}
    \end{figure}

  \end{columns}

\end{frame}

%------------------------

\begin{frame}
  \frametitle{Definiciones}
  Un $p$-símplice singular en un espacio topológico $\X$ será una aplicación continua de un $p$-símplice a $\X$.
  Notaremos por $F_p(\X)$ al conjunto de todos los $p$-símplices singulares de $\X$. \\~\\

  Dada una aplicación continua $f \colon \X \to \Y$, se puede definir $f_\# \colon F_p(\X) \to F_p(\Y)$ como
  $f_\#(\phi) = f \circ \phi$. \\~\\

  Definimos la aplicación $F^i_p \colon \sigma_{p-1} \to \sigma_p$ por $F^i_p(t_0, \dots, t_{p-1}) =
  (t_0, \dots, t_{i-1}, 0, t_i, \dots, t_{p-1})$. \\~\\

  Nos permite definir la aplicación que llamaremos i-ésima cara $\partial_i \colon F_p(\X) \to F_{p-1}(\X)$ como
  $\partial_i(\phi) = \phi \circ F^i_p$. \\~\\

\end{frame}

%---------------------------


\begin{frame}
  \frametitle{Definiciones}
  Tomando un $\mathbb{Z}$-módulo sobre $F_p(\X)$, creamos el grupo de $p$-cadenas singulares,
  $S_p(\X) = \{\sum\limits_{\phi \in F_p(\X)} n_{\phi} \phi \mid n_{\phi} \in \mathbb Z,  \sum n_\phi < \infty\}$. \\~\\

  Sobre el grupo de $p$-cadenas singulares podemos definir un homomorfismo de grupos $\partial$, al que llamaremos borde, dado por
  \[ \partial \colon S_p(\X) \to S_{p-1}(\X) \quad \partial(\phi) = \sum\limits_{i = 0}^p (-1)^i \partial_i(\phi) \]
  \begin{mylemma}
      $\partial \circ \partial \colon S_p(\X) \to S_{p-2}(\X)$ es el homomorfismo cero.
  \end{mylemma}

\end{frame}

%-------------------------------

\begin{frame}[fragile]
  \frametitle{Definiciones}
  Se puede extender la definición de $f_\#$ de $F_p(\X)$ a $S_p(\X)$ por linealidad. \\~\\

  \begin{mylemma}
    La aplicación $f_\#$ conmuta con el borde, esto es, $f_\# \circ \partial = \partial \circ f_\#$
  \end{mylemma}

  Esto hace conmutativo el diagrama:
  \[
    \begin{tikzcd}
      \dots S_{p+1}(\X) \arrow{r}{\partial} \arrow{d}{f_\#} & S_p(\X) \arrow{r}{\partial} \arrow{d}{f_\#} & S_{p-1}(\X) \arrow{d}{f_\#} \dots \\
      \dots S_{p+1}(\Y) \arrow{r}{\partial}                 & S_p(\Y) \arrow{r}{\partial}                 & S_{p-1}(\Y) \dots
    \end{tikzcd}
  \]

\end{frame}

%--------------------------------


\begin{frame}
  \frametitle{Definiciones}
  Consideramos los grupos de los ciclos y los bordes:
  \begin{itemize}
    \item $Z_p(\X) = \{c \in S_p(\X) \mid \partial(c) = 0\}$ $p$-ciclos.
    \item $B_p(\X) = \{c \in S_p(\X) \mid \exists d \colon \partial(d) = c\}$ $p$-bordes.
  \end{itemize}

  Como $\partial^2 = 0$, $B_p \subseteq Z_p$, y podemos tomar el cociente, el $p$-ésimo grupo de homología:
  \centerline{$H_p(\X) = \frac{Z_p(\X)}{B_p(\X)} = \{[c] \mid c \in Z_p(\X)\}$} \\~\\

  Dada una aplicación continua $f \colon \X \to \Y$, es posible definir el homomorfismo inducido en
  la homología $f_* \colon H_p(\X) \to H_p(\Y)$ como $f_*([c]) = [f_\#(c)]$. Si $f$ es un homeomorfismo,
  $f_*$ será un isomorfismo.

\end{frame}

%------------------------------------------------

\section{Homología de un punto. Homología y arcoconexión}

\begin{frame}
  \frametitle{Homología de un punto}
  \begin{myproposition}
    Dado $\X$ un espacio topológico formado por solo un punto, se verifica que:
    \[H_p(\X) = \begin{cases} \Z &\quad \text{si } p = 0 \\
                              0  &\quad \text{en otro caso.} \end{cases} \]
  \end{myproposition}

\end{frame}

%------------------------------------------

\begin{frame}
  \frametitle{La homología singular y la arcoconexión}
  \begin{myproposition}
    Si $\X$ es un espacio topológico arcoconexo, entonces $H_0(\X) = \Z$.
  \end{myproposition}

  \begin{myproposition}
    Sea $\X$ un espacio topológico, $\cup_\alpha \X_\alpha$ su descomposición en componentes arcoconexas. Entonces
    \begin{itemize}
      \item[a)] La función $i_{\alpha *} \colon H_p(\X_\alpha) \to H_p(\X)$ es un monomorfismo $\forall \alpha$.
      \item[b)] $H_p(\X) = \bigoplus_\alpha i_{\alpha *}(H_p(\X_\alpha)) = \bigoplus_\alpha H_p(\X_\alpha)$.
    \end{itemize}
  \end{myproposition}
\end{frame}

%---------------------------------------

\section{Invarianza homotópica de la homología singular}

\begin{frame}
  \frametitle{Lema de Poincaré}
  \begin{myproposition}[Lema de Poincaré]
    Sea $\X \subseteq \R$ estrellado desde algún punto. Entonces
    \[ H_p(\X) = \begin{cases} \Z &\quad \text{si } p = 0 \\
                              0  &\quad \text{en otro caso.} \end{cases} \]
  \end{myproposition}

\end{frame}

%--------------------------------------

\begin{frame}
  \frametitle{Invarianza homotópica}

  \begin{mytheorem}[Invarianza homotópica de la homología singular]
    Sean $\X, \Y$ espacios topológicos, $f, g \colon \X \to \Y$ aplicaciones continuas. Si $f$ es homotópica a $g$,
    entonces $f_* = g_*$, siendo $f_*, g_* \colon H_p(\X) \to H_p(\Y)$.
  \end{mytheorem}

  \begin{mycorollary}
    Sean $\X, \Y$ espacios homotópicamente equivalentes. Entonces $H_p(\X) = H_p(\Y) \quad \forall p$.
  \end{mycorollary}

\end{frame}

%---------------------------------------

\section{Definición en pares}

\begin{frame}
  \frametitle{Homología en pares}
  Tomando un par $(\X, \A)$, con $\A \subseteq \X$, se puede definir $S_p(\X, \A) = \frac{S_p(\X)}{S_p(\A)}$,
  el grupo de $p$-cadenas del par, donde podemos volver a tomar el borde $\partial$. \\~\\

  De nuevo podemos tomar los ciclos y los bordes, que nos permiten definir la homología del par:
  \centerline{$H_p(\X, \A) = \frac{Z_p(\X, \A)}{B_p(\X, \A)}$} \\~\\

  Tomando una aplicación de pares $f \colon (\X, \A) \to (\Y, \B)$, se construyen
  \begin{itemize}
    \item $f_\# \colon S_p(\X, \A) \to S_p(\Y, \B)$, con $f_\# \circ \partial = \partial \circ f_\#$.
    \item $f_* \colon H_p(\X, \A) \to H_p(\Y, \B)$, homomorfismo inducido.
  \end{itemize}

\end{frame}

%---------------------------------

\begin{frame}
  \frametitle{Trasladando los resultados}
  Los resultados vistos de arcoconexión e invarianza homotópica tienen sus equivalentes en el caso de pares.

  \begin{myproposition}
    Si $\X$ es arcoconexo y $\A \neq \emptyset, H_0(\X, \A) = 0$.
  \end{myproposition}

  \begin{myproposition}
    Sea $\cup_\alpha \X_\alpha$ la descomposición de $\X$ en componentes arcoconexas, $\A \subseteq \X$. Entonces
    \begin{itemize}
      \item[a)] La función $i_{\alpha *} \colon H_p(\X_\alpha, \X_\alpha \cap \A) \to H_p(\X, \A)$ es un monomorfismo $\forall \alpha$.
      \item[b)] $H_p(\X, \A) = \bigoplus_\alpha H_p(\X_\alpha, \X_\alpha \cap \A)$.
    \end{itemize}
  \end{myproposition}


\end{frame}

%---------------------------------

\begin{frame}
  \frametitle{Trasladando los resultados}

  Nuevamente, si $f$ es un homeomorfismo entre $\X$ e $\Y$, verificando $f(\A) = \B$, entonces $f_*$
  es un isomorfismo entre los grupos de homología. \\~\\

  \begin{myproposition}
    Dos aplicaciones de pares $f$ y $g$ homotópicas inducen el mismo homomorfismo en la homología del par.
  \end{myproposition}


\end{frame}

%----------------------------------

\begin{frame}
  \frametitle{Sucesión exacta de un par}
  Notaremos por $\pi \colon S_p(\X) \to S_p(\X, \A)$ a la proyección. De igual forma, notaremos por
  $i_\# \colon S_p(\A) \to S_p(\X)$ a la inclusión. \\~\\

  Podemos definir la sucesión
  \[ 0 \to S_*(\A) \xrightarrow{i_\#} S_*(\X) \xrightarrow{\pi} S_*(\X, \A) \to 0 \]
  que verifica $\partial \circ i_\# = i_\# \circ \partial$ y $\partial \circ \pi = \pi \circ \partial$,
  y donde se tiene $\Ker \pi = \Img i_\# = i_\#(S_*(\A))$. A estas sucesiones las llamaremos exactas.

\end{frame}

%----------------------------------

\begin{frame}
  \frametitle{Sucesión exacta de un par}
  \begin{myproposition}[Sucesión exacta del par]
    Podemos definir un homomorfismo $\Delta \colon H_p(\X, \A) \to H_{p-1}(\A)$ tal que
    $\Img \Delta = \Ker i_*$ y $\Ker \Delta = \Img \pi_*$. \\~\\

    A la sucesión así construída la llamaremos sucesión exacta del par $(\X, \A)$
    \[ \dots H_{p+1}(\X, \A) \xrightarrow{\Delta} H_p(\A) \xrightarrow{i_*} H_p(\X) \xrightarrow{\pi_*} H_p(\X, \A) \xrightarrow{\Delta} H_{p-1}(\A) \dots \]
  \end{myproposition}

\end{frame}

%----------------------------------

\section{Teorema de subdivisión baricéntrica}

\begin{frame}
  \frametitle{Definiciones}
  Si $\X$ es un espacio topológico y $\U$ es un recubrimiento suyo, se define
  $F_p^\U(\X) = \{ \phi \in F_p \mid \exists U \in \U \colon \Img \phi \subseteq U \}$,
  que es, por definición, subconjunto de $F_p(\X)$. \\~\\

  Volveremos a tomar el grupo abeliano libre generado por $F_p^\U(\X)$, al que notaremos $S_p^\U(\X)$.
  Claramente, es subgrupo de $S_p(\X)$. La restricción del borde $\partial$ a $S_p^\U(\X)$ define un borde en este espacio.  \\~\\

  La inclusión $i \colon S_*^\U(\X) \to S_*(\X)$ induce un homomorfismo en la homología
  \[ i_* \colon H_p(S_*^\U(\X)) \to H_p(\X) \]
  donde $H_p(S_*^\U(\X)) = \frac{\Ker \partial}{\Img \partial}$


\end{frame}


\begin{frame}
  \frametitle{Teorema de subdivisión baricéntrica}

  \begin{mytheorem}[Teorema de subdivisión baricéntrica]
    Sean $\X$ un espacio topológico y $\U$ un recubrimiento suyo tal que $\interior{\U} = \{\interior{U} \mid U \in \U\}$
    también recubra a $\X$. Entonces la aplicación \[i_* \colon H_p(S_*^\U(\X)) \to H_p(\X)\] es un isomorfismo.
  \end{mytheorem}

\end{frame}

%----------------------------------

\section{Homología de las esferas}

%----------------------------------


\begin{frame}
  \frametitle{Teorema de escisión}
  \begin{mytheorem}[Teorema de escisión]
    Sea $(\X, \A)$ un par, $U \subseteq \A$ verificando $\overline{U} \subseteq \interior{\A}$.
    Entonces la inclusión
    \[ i \colon (\X - U, \A - U) \to (\X, \A) \]
    induce un isomorfismo en la homología.
  \end{mytheorem}
\end{frame}

%------------------------------------

\begin{frame}
  \frametitle{Homología de las esferas}
  Con el teorema de escisión es posible calcular los grupos de homología de las esferas. \\~\\

  \begin{mytheorem}[Grupos de homología de las esferas]
    $H_p(S^n) = \begin{cases} \mathbb Z & p = 0, n,\\
                                      0 & \text{en otro caso.} \end{cases}$
  \end{mytheorem}
\end{frame}


%-------------------------------------------------

\begin{frame}
\Huge{\centerline{Fin.}}
\end{frame}


%----------------------------------------------------------------------------------------

\end{document}
--
